% HEVEA % Copyright (c) 2015 Synrc Research Center

%\usepackage{afterpage}
\usepackage[english,russian]{babel}
%\usepackage{graphicx}
%\usepackage{tocloft}
\usepackage{fontspec}
\usepackage{polyglossia}
\usepackage{hyphenat}
\usepackage{import}

%\usepackage{caption}
%\usepackage[usenames,dvipsnames]{color}
\usepackage[top=18mm, bottom=22.4mm,
            inner=15mm,outer=18mm,
            paperwidth=142mm, paperheight=200mm]{geometry}

\hyphenation{бес-чувстве-нен cуще-ствования
 буду-щем разно-образии Дхар-му наско-лько счастли-вого ниж-них
 сохрани-лось всеведую-щие свер-нуть неза-висимости}

\fontencoding{T1}

%\setlength{\cftsubsecnumwidth}{2.5em}
%\defaultfontfeatures{Ligatures=TeX}

% include image for HeVeA and LaTeX

%\makeatletter
%\def\@seccntformat#1{\llap{\csname the#1\endcsname\hskip0.7em\relax}}
%\makeatother

\newcommand{\includeimage}[2]
{\begin{figure}[h!]
\centering
\includegraphics[width=\textwidth]{#1}
\caption{#2}
\end{figure}
}

\newcommand*{\titlePRAYER}
{
\newfontfamily{\cyrillicfont}{Geometria}
\setdefaultlanguage{tibetan}
\setmainfont{Geometria}
    \begingroup
        \thispagestyle{empty}
        \hspace*{0.15\textwidth}
        \rule{1pt}{\textheight}
        \hspace*{0.05\textwidth}
        {
        \parbox[c][][s]{0.75\textwidth}
        {
             \noindent
             \vspace{-12cm}
             \textsc{
             \setdefaultlanguage{russian}
             \setmainfont{Geometria}
             \Large
             Сборник Практик\\ [0.3\baselineskip]
             Лонгчен Ньингтик\\ [0.3\baselineskip]
             \\
             \vspace*{3cm}
             \\
             \normalsize
             Устье, 2015}
        }}
    \endgroup
    \setdefaultlanguage{russian}
    \setmainfont{Geometria}
}



% define images store

\graphicspath{{./images/}}

% start each section from new page

%\let\stdsection\section
%\renewcommand\section{\newpage\stdsection}

% define style for code listings

\lefthyphenmin=1
\hyphenpenalty=100
\tolerance=3000

%\newcommand\blankpage{
%    \null
%    \thispagestyle{empty}
%    \newpage}

\newcommand{\Vspace}[1]{\vspace{#1}}
\newcommand{\Section}[1]{\section{#1}}
\newcommand{\SectionNo}[1]{\section*{#1}}
\newcommand{\SubSection}[1]{\subsection{#1}}
\newcommand{\SubSectionNo}[1]{\subsection*{#1}}
\newcommand{\SubSubSection}[1]{\subsubsection{#1}}
\newcommand{\SubSubSectionNo}[1]{\subsubsection*{#1}}

\newcommand{\ru}{
    \setdefaultlanguage{russian}
    \setmainfont{Geometria}
}

\newcommand{\ti}{
    \setdefaultlanguage{tibetan}
    \setmainfont{DDC Uchen}
}

\newtoggle{russian@scriptlangequal}
\newtoggle{tibetan@scriptlangequal}

\newlength\tindent
\setlength{\tindent}{\parindent}
\setlength{\parindent}{0pt}
\renewcommand{\indent}{\hspace*{\tindent}}

% HEVEA \begin{document}
% nyingma_author =Maxim Sokhatsky=
% HEVEA \title{Кунзанг Монлам}

\Section{Ригдзин Годем: Кунзанг Монлам}
\SectionNo{{\ti ༄༅། །ཀུན་ཏུ་བཟང་པོའི་སྨོན་ལམ་སྟོབས་པོ་ཆེ་ཞེས་བྱ་བ་བཞུགས་སོ།}}\\
\\
{\ti
ཙིཏྟ་ཨ༔ དེ་ནས་ཐོག་མའི་སངས་རྒྱས་ཀུན་ཏུ་བཟང་པོས༔
འཁོར་བའི་སེམས་ཅན་སངས་མི་རྒྱ་བའི་དབང་མེད་པའི་སྨོན་ལམ་ཁྱད་པར་ཅན་འདི་གསུང་ངོ༔}\\
\\
ЧИТТА А: После этого изначальный Будда Самантабхадра произнёс
это особое устремление, посредством которого
чувствующие существа в сансаре не смогут не просветлиться.\\
\\
{\ti
ཧོ༔ སྣང་སྲིད་འཁོར་འདས་ཐམས་ཅད་ཀུན༔\\
གཞི་གཅིག་ལམ་གཉིས་འབྲས་བུ་གཉིས༔ \\
རིག་དང་མ་རིག་ཆོ་འཕྲུལ་ཏེ༔ \\
ཀུན་ཏུ་བཟང་པོའི་སྨོན་ལམ་གྱིས༔ \\
ཐམས་ཅད་ཆོས་དབྱིངས་ཕོ་བྲང་དུ༔ \\
མངོན་པར་རྫོགས་ཏེ་སངས་རྒྱས་ཤོག༔ }\\
\\
Хо! У всего проявленного и сущего, у сансары и нирваны,\\
Есть одна основа, два пути и два плода,\\
Проявляющиеся волшебным образом через осознавание и неведение.\\
Так пусть посредством этой молитвы Самантабхадры\\
Все достигнут явного и совершенного просветления\\
Во дворце основного пространства явлений.\\
\newpage
\\
{\ti
ཀུན་གྱི་གཞི་ནི་འདུས་མ་བྱས༔\\
རང་བྱུང་ཀློང་ཡངས་བརྗོད་དུ་མེད༔\\
འཁོར་འདས་གཉིས་ཀའི་མིང་མེད་དོ༔\\
དེ་ཉིད་རིག་ན་སངས་རྒྱས་ཏེ༔\\
མ་རིག་སེམས་ཅན་འཁོར་བར་འཁྱམས༔\\
ཁམས་གསུམ་སེམས་ཅན་ཐམས་ཅད་ཀྱིས༔\\
བརྗོད་མེད་གཞི་དོན་རིག་པར་ཤོག༔}\\
\\
Основа всего бытия не сотворена;\\
Этому самосущему, бескрайнему и невыразимому пространству \\
Неведомы названия «сансара» и «нирвана».\\
Те, кто осознают это, являются буддами,\\
А незнающие это блуждают по сансаре как наделённые ошибочным умом существа. \\
Так пусть все существа трёх миров\\
Осознают невыразимую суть основы бытия.\\
\\
{\ti
ཀུན་ཏུ་བཟང་པོ་ང་ཡིས་ཀྱང༔ \\
རྒྱུ་རྐྱེན་མེད་པ་གཞི་ཡི་དོན༔ \\
དེ་ཉིད་གཞི་ལས་རང་བྱུང་རིག༔ \\
ཕྱི་ནང་སྒྲོ་སྐུར་སྐྱོན་མ་བཏགས༔ \\
དྲན་མེད་མུན་པའི་དྲི་མ་བྲལ༔ \\
དེ་ཕྱིར་རང་སྣང་སྐྱོན་མ་གོས༔ \\
རང་རིག་སོ་ལ་གནས་པ་ལ༔}\\
\\
Я, Самантабхадра, постиг суть беспричинной \\
и необусловленной основы, \\
Которая есть самовозникающее из основы присутствие, \\
И пребываю в самоосознавании, незапятнанном \\
недостатками самопроявления, \\
Ущербностью внешних и внутренних ментальных крайностей \\
И мраком неосознанности;
\\
\newpage
{\ti
སྲིད་གསུམ་འཇིག་ཀྱང་སྔངས་སྐྲག་མེད༔\\
སྣང་སེམས་གཉིས་སུ་མེད་པ་ལ༔\\
འདོད་ཡོན་ལྔ་ལ་ཆགས་པ་མེད༔\\
རྟོག་མེད་ཤེས་པ་རང་བྱུང་ལ༔\\
གདོས་པའི་གཟུགས་དང་ཁ་དོག་མེད༔}\\
\\
Даже если разрушатся три мира, мне нечего бояться \\
В недвойственности проявлений и ума \\
Нет привязанности к пяти чувственным удовольствиям. \\
В неконцептуальном, самосущем знании \\
Нет материальных форм и пяти эмоциональных ядов.\\
\\
{\ti
རིག་པའི་གསལ་ཆ་མ་འགགས་པས༔ \\
ངོ་བོ་གཅིག་ལ་ཡེ་ཤེས་ལྔ༔ \\
ཡེ་ཤེས་ལྔ་པོ་སྨིན་པ་ལས༔ \\
ཐོག་མའི་སངས་རྒྱས་རིགས་ལྔ་བྱུང༔}\\
\\
Из непрерывного аспекта ясности осознавания\\
Появляются пять видов мудрости, обладающие единой сущностью.\\
Созревая, эти пять видов мудрости \\
Становятся пятью первоначальными семействами будд.\\
\\
{\ti
དེ་ལས་ཡེ་ཤེས་མཐའ་རྒྱས་པས༔ \\
སངས་རྒྱས་བཞི་བཅུ་རྩ་གཉིས་བྱུང༔ \\
ཡེ་ཤེས་ལྔ་ཡི་རྩལ་ཤར་བས༔ \\
ཁྲག་འཐུང་དྲུག་ཅུ་ཐམ་པ་བྱུང༔}\\
\\
Благодаря распространению этой изначальной мудрости \\
Появляются 42 мирных будды,\\
А в результате восхождения энергии пяти видов мудрости \\
Возникают 60 гневных будд, пьющих кровь эмоций.\\
\\
\newpage
\\
{\ti
དེ་ཕྱིར་གཞི་རིག་འཁྲུལ་མ་མྱོང༔ \\
ཐོག་མའི་སངས་རྒྱས་ང་ཡིན་པས༔ \\
ང་ཡི་སྨོན་ལམ་བཏབ་པ་ཡིས༔ \\
ཁམས་གསུམ་འཁོར་བའི་སེམས་ཅན་གྱིས༔ \\
རང་བྱུང་རིག་པ་ངོ་ཤེས་ནས༔ \\
ཡེ་ཤེས་ཆེན་པོ་མཐའ་རྒྱས་ཤོག༔}\\
\\
Поэтому основа, как осознавание, никогда не ведала заблуждения. \\
Я, Самантабхадра, изначальный будда\\
Возношу молитву, чтобы\\
Все существа, блуждающие по трём мирам сансары, \\
Постигли самосущее осознавание \\
И чтобы великая мудрость дошла до краёв света.\\
\\
{\ti
ང་ཡི་སྤྲུལ་པ་རྒྱུན་མི་ཆད༔ \\
བྱེ་བ་ཕྲག་བརྒྱ་བསམ་ཡས་འགྱེད༔ \\
གང་ལ་གང་འདུལ་སྣ་ཚོགས་སྟོན༔ \\
ང་ཡི་ཐུགས་རྗེ་སྨོན་ལམ་གྱིས༔ \\
ཁམས་གསུམ་འཁོར་བའི་སེམས་ཅན་ཀུན༔ \\
རིགས་དྲུག་གནས་ནས་འཐོན་པར་ཤོག༔}\\
\\
Непрерывным потоком мои эманации\\
Будут распространяться в миллиардах непостижимых образов, \\
Представая в тех или иных формах\\
Перед теми, кто нуждается в усмирении.\\
Пусть благодаря моей молитве, преисполненной сострадания,\\
Все существа трёх миров сансары\\
Исчезнут из обителей шести миров.\\
\\
\newpage
{\ti
དང་པོ་སེམས་ཅན་འཁྲུལ་པ་རྣམས༔ \\
གཞི་ལ་རིག་པ་མ་ཤར་བས༔ \\
ཅི་ཡང་དྲན་མེད་ཐོམ་མེ་བ༔ \\
དེ་ཀ་མ་རིག་འཁྲུལ་པའི་རྒྱུ༔ \\
དེ་ལ་ཧད་ཀྱིས་བརྒྱལ་བ་ལ༔ \\
དངངས་སྐྲག་ཤེས་པ་ཟ་ཟིར་འགྱུས༔ \\
དེ་ལ་བདག་གཞན་དགྲར་འཛིན་སྐྱེས༔}\\
\\
Существа заблуждаются, потому что в самом начале \\
У них не возникло осознавание по отношению к основе; \\
Из-за этого у них отсутствует какая-либо осознанность.\\
Это и есть причина заблуждения и неведения.\\
Из-за такого бессознательного состояния\\
Появляются опасения, страх и нервозность,\\
От чего возникает цепляние за себя и восприятие других в виде врагов\\
\\
{\ti
བག་ཆགས་རིམ་གྱིས་བརྟས་པ་ལས༔ \\
འཁོར་བ་ལུགས་སུ་འཇུག་པ་བྱུང༔ \\
དེ་ལས་ཉོན་མོངས་དུག་ལྔ་རྒྱས༔ \\
དུག་ལྔའི་ལས་ལ་རྒྱུན་ཆད་མེད༔}\\
\\
Постепенно такая тенденция входит в привычку, \\
И из этого возникает цепочка сансары; \\
Развиваются пять эмоциональных ядов, \\
Которые влекут за собой непрерывный поток кармы.\\
\\
\newpage
{\ti
དེ་ཕྱིར་སེམས་ཅན་འཁྲུལ་པའི་གཞི༔ \\
དྲན་མེད་མ་རིག་ཡིན་པའི་ཕྱིར༔ \\
སངས་རྒྱས་ང་ཡི་སྨོན་ལམ་གྱིས༔ \\
ཀུན་གྱིས་རིག་པ་རང་ཤེས་ཤོག༔}\\
\\
 Поэтому неведение и отсутствие осознанности \\
Является основой заблуждения чувствующих существ; \\
Так пусть благодаря моей просветлённой молитве \\
Все постигнут своё осознавание. \\
\\
{\ti
ལྷན་ཅིག་སྐྱེས་པའི་མ་རིག་པ༔ \\
ཤེས་པ་དྲན་མེད་ཐོམ་མེ་བ༔ \\
ཀུན་ཏུ་བཏགས་པའི་མ་རིག་པ༔ \\
བདག་གཞན་གཉིས་སུ་འཛིན་པ་ཡིན༔ \\
ལྷན་སྐྱེས་ཀུན་བཏགས་མ་རིག་གཉིས༔ \\
སེམས་ཅན་ཀུན་གྱི་འཁྲུལ་གཞི་ཡིན༔}\\
\\
Совозникающее неведение – \\
Это отвлечение восприятия и отсутствие осознанности. \\
Концептуальное неведение – \\
Это двойственное восприятие – деление мира на «себя» и «других». \\
Как совозникающее, так и концептуальное неведение \\
Являются основой заблуждения существ.\\
\newpage
{\ti
སངས་རྒྱས་ང་ཡི་སྨོན་ལམ་གྱིས༔ \\
འཁོར་བའི་སེམས་ཅན་ཐམས་ཅད་ཀྱི༔ \\
དྲན་མེད་འཐིབས་པའི་མུན་པ་སངས༔ \\
གཉིས་སུ་འཛིན་པའི་ཤེས་པ་དྭངས༔ \\
རིག་པ་རང་ངོ་ཤེས་པར་ཤོག༔}\\
\\
Так пусть благодаря моей просветлённой молитве \\
У всех существ в сансаре \\
Рассеется помрачающая мгла отсутствия осознанности \\
Исчезнет двойственное восприятие \\
И они узрят естественный облик осознавания.\\
\\
{\ti
གཉིས་འཛིན་བློ་ནི་ཐེ་ཚོམ་སྟེ༔ \\
ཞེན་པ་ཕྲ་མོ་སྐྱེས་པ་ལས༔ \\
བག་ཆགས་མཐུག་པོ་རིམ་གྱིས་བརྟས༔ \\
ཟས་ནོར་གོས་དང་གནས་དང་གྲོགས༔ \\
འདོད་ཡོན་ལྔ་དང་བྱམས་པའི་གཉེན༔ \\
ཡིད་འོང་ཆགས་པའི་འདོད་པས་གདུང༔}\\
\\
Двойственный рассудок порождает сомнения; \\
Начиная с появления тонких цепляний, \\
Стереотипные тенденции постепенно входят в привычку и набирают силу: \\
Еда, богатство, одежда, дом и друзья, \\
Пять чувственных удовольствий, любимые и близкие \\
Вызывают мучительную привязанность к приятным вещам.\\
\\
\newpage
{\ti
དེ་དག་འཇིག་རྟེན་འཁྲུལ་པ་སྟེ༔ \\
གཟུང་འཛིན་ལས་ལ་ཟད་མཐའ་མེད༔ \\
ཞེན་པའི་འབྲས་བུ་སྨིན་པའི་ཚེ༔ \\
བརྐམ་ཆགས་གདུང་བའི་ཡི་དྭགས་སུ༔ \\
སྐྱེས་ནས་བཀྲེས་སྐོམ་ཡ་རེ་ང༔}\\
\\
Все эти мирские вещи сбивают с пути, \\
А карма, возникающая из двойственного восприятия, не имеет ни конца, ни края. \\
Когда вызревают плоды привязанностей, \\
Существа рождаются голодными духами, \\
Которые мучаются от желаний и страдают от голода и жажды.\\
\\
{\ti
སངས་རྒྱས་ང་ཡི་སྨོན་ལམ་གྱིས༔ \\
འདོད་ཆགས་ཞེན་པའི་སེམས་ཅན་རྣམས༔ \\
འདོད་པའི་གདུང་བ་ཕྱིར་མ་སྤངས༔ \\
འདོད་ཆགས་ཞེན་པ་ཚུར་མ་བླངས༔ \\
ཤེས་པ་རང་སོར་གློད་པ་ཡིས༔ \\
རིག་པ་རང་སོ་ཟིན་གྱུར་ནས༔ \\
ཀུན་རྟོག་ཡེ་ཤེས་ཐོབ་པར་ཤོག༔}\\
\\
Так пусть посредством моей просветлённой молитвы \\
Все чувствующие существа, одержимые цепляниями и привязанностями \\
И страдающие от того, что не могут расстаться с желаниями, \\
Перестанут стяжать и цепляться. \\
Раскрепостив своё восприятие в естественном состоянии, \\
Пусть они позволят осознаванию занять своё собственное положение \\
И постигнут всеразличающую мудрость.\\
\\
\newpage
{\ti
ཕྱི་རོལ་ཡུལ་གྱི་སྣང་བ་ལ༔ \\
འཇིགས་སྐྲག་ཤེས་པ་ཕྲ་མོ་འགྱུས༔ \\
སྡང་བའི་བག་ཆགས་བརྟས་པ་ལས༔ \\
དགྲར་འཛིན་བརྡེག་གསོད་རགས་པ་སྐྱེས༔ \\
ཞེ་སྡང་འབྲས་བུ་སྨིན་པའི་ཚེ༔ \\
དམྱལ་བའི་བཙོ་བསྲེག་སྡུག་རེ་བསྔལ༔}\\
\\
По отношению к объектам, воспринимаемым как внешние, \\
Появляется тонкое ощущение опасения. \\
Тенденция к их отвержению становится всё сильнее, \\
И враждебное восприятие приводит к насилию и убийствам. \\
Когда же вызревает плод гнева и агрессии, \\
Существа рождаются в аду, мучаясь от того, что их жгут и варят.\\
\\
{\ti
སངས་རྒྱས་ང་ཡི་སྨོན་ལམ་གྱིས༔ \\
འགྲོ་དྲུག་སེམས་ཅན་ཐམས་ཅད་ཀྱི༔ \\
ཞེ་སྡང་དྲག་པོ་སྐྱེས་པའི་ཚེ༔ \\
སྤང་བླང་མི་བྱ་རང་སོར་གློད༔ \\
རིག་པ་རང་སོ་ཟིན་གྱུར་ནས༔ \\
གསལ་བའི་ཡེ་ཤེས་ཐོབ་པར་ཤོག༔}\\
\\
Так пусть посредством моей просветлённой молитвы\\
Все существа шести миров, испытывающие сильный гнев и агрессию,\\
В тот момент, когда это появляется в них,\\
смогут расслабиться в естественном состоянии, ничего не принимая и ничего не отвергая,\\
И постигнут светоносную мудрость.\\
\\
\newpage
{\ti
རང་སེམས་ཁེངས་པར་གྱུར་པ་ལས༔ \\
གཞན་ལ་འགྲན་སེམས་སྨད་པའི་བློ༔ \\
ང་རྒྱལ་དྲག་པོའི་སེམས་སྐྱེས་པས༔ \\
བདག་གཞན་འཐབ་རྩོད་སྡུག་བསྔལ་སྤྱོད༔ \\
ལས་དེའི་འབྲས་བུ་སྨིན་པའི་ཚེ༔ \\
འཕོ་ལྟུང་མྱོང་བའི་ལྷ་རུ་སྐྱེས༔}\\
\\

Когда в уме появляется высокомерие,\\
Его сопровождает желание соперничать и унижать других.\\
Такая сильная гордыня, захватывая ум,\\
Приводит к страданиям от ссор и борьбы с другими.\\
Когда вызревают плоды этой кармы,\\
Происходит рождение в мире богов, где страдают от перемен, а затем падают (в низшие миры).\\
\\
{\ti
སངས་རྒྱས་ང་ཡི་སྨོན་ལམ་གྱིས༔ \\
ཁེངས་སེམས་སྐྱེས་པའི་སེམས་ཅན་རྣམས༔ \\
དེ་ཚེ་ཤེས་པ་རང་སོར་གློད༔ \\
རིག་པ་རང་སོ་ཟིན་གྱུར་ནས༔ \\
མཉམ་པ་ཉིད་ཀྱི་དོན་རྟོགས་ཤོག༔}\\
\\
Так пусть посредством моей просветлённой молитвы \\
Все чувствующие существа, у которых появляется высокомерие, \\
В этот же самый момент раскрепостят своё сознание в естественном состоянии. \\
И когда их осознавание займёт своё естественное положение, \\
Пусть они реализуют суть равностности.\\
\\
\newpage
{\ti
གཉིས་འཛིན་བརྟས་པའི་བག་ཆགས་ཀྱིས༔ \\
བདག་བསྟོད་གཞན་སྨོད་ཟུག་རྔུའི་ལས༔ \\
འཐབ་རྩོད་འགྲན་སེམས་བརྟས་པ་ལས༔ \\
གསོད་གཅོད་ལྷ་མིན་གནས་སུ་སྐྱེས༔ \\
འབྲས་བུ་དམྱལ་བའི་གནས་སུ་ལྟུང༔}\\
\\
Когда мы мучаемся от того, чтобы возвысить себя, принизив при этом других, \\
В соперничающем уме усиливается тенденция к ссорам и борьбе; \\
По этой причине происходит рождение в мире сражающихся полубогов, \\
А это, в свою очередь, ведёт к падению в ад.\\
\\
{\ti
སངས་རྒྱས་ང་ཡི་སྨོན་ལམ་གྱིས༔ \\
འགྲན་སེམས་འཐབ་རྩོད་སྐྱེས་པ་རྣམས༔ \\
དགྲར་འཛིན་མི་བྱ་རང་སོར་གློད༔ \\
ཤེས་པ་རང་སོ་ཟིན་གྱུར་ནས༔ \\
ཕྲིན་ལས་ཐོགས་མེད་ཡེ་ཤེས་ཤོག༔}\\
\\

Так пусть благодаря моей просветлённой молитве \\
Все существа, одержимые мыслями о соперничестве, ссорах и борьбе, \\
Вместо того чтобы воспринимать других враждебно, \\
Расслабятся в естественном состоянии; \\
Пусть их восприятие займёт естественное положение \\
И они постигнут мудрость беспрепятственной активности.\\
\\
\newpage
{\ti
དྲན་མེད་བཏང་སྙོམས་ཡེངས་པ་དང༔ \\
འཐིབས་དང་རྨུགས་དང་བརྗེད་པ་དང༔ \\
བརྒྱལ་དང་ལེ་ལོ་གཏི་མུག་པས༔ \\
འབྲས་བུ་སྐྱབས་མེད་བྱོལ་སོང་འཁྱམས༔}\\
\\
В результате отсутствия осознанности, апатии, отвлечений, \\
Сонливости, притупленного восприятия и забывчивости, \\
Лени, тупости и бессознательного, отсутствующего состояния, \\
Становятся беспомощными животными.\\
\\
{\ti
སངས་རྒྱས་ང་ཡི་སྨོན་ལམ་གྱིས༔ \\
གཏི་མུག་བྱིང་བའི་མུན་པ་ལ༔ \\
དྲན་པ་གསལ་བའི་མདངས་ཤར་ནས༔ \\
རྟོག་མེད་ཡེ་ཤེས་ཐོབ་པར་ཤོག༔}\\
\\

Так пусть посредством моей просветлённой молитвы \\
У тех, кто пребывает во мраке беспроглядной тупости, \\
Взойдёт светоносное сияние осознанности \\
И они постигнут неконцептуальную мудрость.\\
\\
\newpage
{\ti
ཁམས་གསུམ་སེམས་ཅན་ཐམས་ཅད་ཀྱང༔ \\
ཀུན་གཞི་སངས་རྒྱས་ང་དང་མཉམ༔ \\
དྲན་མེད་འཁྲུལ་པའི་གཞི་རུ་སོང༔ \\
ད་ལྟ་དོན་མེད་ལས་ལ་སྤྱོད༔\\
ལས་དྲུག་རྨི་ལམ་འཁྲུལ་པ་འདྲ༔}\\
\\

Все существа трёх миров \\
Равнозначны мне, Будде вселенской основы бытия, \\
Но из-за отсутствия осознанности \\
они оказались в плену заблуждения \\
И в данный момент \\
занимаются бессмысленными делами.\\
\\
{\ti
ང་ནི་སངས་རྒྱས་ཐོག་མ་ཡིན༔ \\
འགྲོ་དྲུག་སྤྲུལ་པས་འདུལ་བའི་ཕྱིར༔ \\
ཀུན་ཏུ་བཟང་པོའི་སྨོན་ལམ་གྱིས༔ \\
སེམས་ཅན་ཐམས་ཅད་མ་ལུས་པ༔ \\
ཆོས་ཀྱི་དབྱིངས་སུ་སངས་རྒྱས་ཤོག༔}\\
\\

Шесть видов кармы – это заблуждение, которое происходит как во сне. \\
Я, изначальный Будда, \\
Присутствую здесь, чтобы усмирить шесть видов существ посредством моих эманаций. \\
Так пусть благодаря молитве Самантабхадры \\
Все чувствующие существа без исключения \\
Просветлятся в основном пространстве явлений!\\
\\
\newpage
{\ti
ཨ་ཧོ༔ ཕྱིན་ཆད་རྣལ་འབྱོར་སྟོབས་ཅན་གྱིས༔ \\
འཁྲུལ་མེད་རིག་པ་རང་གསལ་ངང༔ \\
སྨོན་ལམ་སྟོབས་ཆེན་འདི་བཏབ་པས༔ \\
འདི་ཐོས་སེམས་ཅན་ཐམས་ཅད་ཀུན༔ \\
སྐྱེ་བ་གསུམ་ནས་མངོན་འཚང་རྒྱ༔}\\
\\
A ХО! Если, начиная с данного момента, могущественный йогин, \\
Пребывающий в свободном от заблуждений, самосветоносном осознавании, \\
Произнесёт это устремление великой силы, \\
То все существа, которые услышат его, \\
Достигнут явного просветления в течение трёх жизней.\\
\\
{\ti
ཉི་ཟླ་གཟའ་ཡིས་ཟིན་པ་འམ༔ \\
སྒྲ་དང་ས་གཡོ་འབྱུང་བ་འམ༔ \\
ཉི་མ་ལྡོག་འགྱུར་ལོ་འཕོའི་དུས༔ \\
རང་ཉིད་ཀུན་ཏུ་བཟང་པོར་བསྐྱེད༔}\\
\\
Во время солнечного и лунного затмения, \\
Когда трясётся земля и гремит гром, \\
Во время солнцестояния и в Новый год \\
Представляйте себя в виде Самантабхадры.\\
\newpage
{\ti
ཀུན་གྱིས་ཐོས་པར་འདི་བརྗོད་ན༔ \\
ཁམས་གསུམ་སེམས་ཅན་ཐམས་ཅད་ཀྱང༔ \\
རྣལ་འབྱོར་དེ་ཡི་སྨོན་ལམ་གྱིས༔ \\
སྡུག་བསྔལ་རིམ་གྱིས་གྲོལ་ནས་ཀྱང༔ \\
མྱུར་བར་སངས་རྒྱས་ཐོབ་པར་འགྱུར༔}\\
\\
И если это произнесено так, что все вас слышат, \\
То, благодаря устремлению этого йогина, \\
Все существа трёх миров \\
Постепенно освободятся от страданий \\
И в конце концов достигнут просветления.\\
\\
{\ti
སྨོན་ལམ་རྒྱལ་པོ་འདི་དག་མཆོག་གི་གཙོ། །མཐའ་ཡས་འགྲོ་བ་ཀུན་ལ་ཕན་བྱེད་ཅིང་།
།ཀུན་ཏུ་བཟང་པོས་བརྒྱན་པའི་གཞུང་གྲུབ་སྟེ།
།ངན་སོང་རྒྱུད་རྣམས་མ་ལུས་སྟོངས་པར་ཤོག
།ཅེས་གསུངས་སོ༔ རྫོགས་པ་ཆེན་པོ་ཀུན་ཏུ་བཟང་པོ་དགོངས་པ་ཟང་ཐལ་དུ་བསྟན་པའི་རྒྱུད་ལས༔
སྨོན་ལམ་སྟོབས་པོ་ཆེ་བཏབ་པས༔
སེམས་ཅན་སངས་མི་རྒྱ་བའི་དབང་མེད་པར་བསྟན་པའི་ལེའུ་དགུ་པའོ༔ མངྒ་ལཾ།། །།}\\
\\
\scriptsize
Это устремление, обладающее великой силой, благодаря которому 
ни одно чувствующих существ не сможет избежать просветления, 
извлечено из 9-й главы тантры великого совершенства «Распахнутая 
реализация Самантабхадры».\\
\\
Во благо! Во благо! Во благо!\\
\\
Перевёл с тибетского Лама Сонам Дордже в День Снисхождения Будды из мира богов.
\normalsize
% HEVEA \end{document}
