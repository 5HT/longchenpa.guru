\section{Дуджом Ринпоче.\\Эликсир Достижения}

\large Сущностные наставления \\
по практике Дхармы в горном отшельничестве, \\
данные в лёгком для понимания изложении. \normalsize
\\
\\
С преданностью я принимаю прибежище и простираюсь пред
стопами своего великолепного святого Гуру, чья доброта
несравнима ни с чем. Даруй своё благословение, чтобы в
моём уме и в моих последователях быстро возникла безошибочная
реализация глубочайшего пути, и мы овладели изначальной
цитаделью в этой жизни.\\
\\
В силу предыдущих молитв и чистой кармы некоторые
достойные люди обрели чистосердечную веру в тайные
глубочайшие учения Великого Завершения и преподающих
их Учителей, стремясь практиковать до заключительного
этапа. Эти сущностные наставления по практике ключевых
пунктов наисекретнейшего Великого Завершения вручаются
таким людям при удалении в горное затворничество и изложены
для легкого понимания. Они состоят из трёх общих разделов:\\
\\
• Предварительная часть: как освободиться от пут страстей и привязанностей, повернуть свой ум к Дхарме и очистить свой поток бытия.\\
\\
• Основная часть: как устранить недопонимания в отношении воззрения, медитации и поведения и «выпрямить» практику.\\
\\
• Заключительная часть: как соблюдать обеты и самаи, объединяя жизнь с Дхармой.\\
\\
\newpage
\subsection{Предварительная часть}
\large Как освободиться от пут страстей и привязанностей,\\
повернуть свой ум к Дхарме \\
и очистить свой поток бытия.\normalsize
\\
\\
Увы! Хотя это бурлящее и беспокойное восприятие, называе\-мое «умом»,
появилось одновременно с Буддой Самантахад\-рой, живые существа
умудрились заблудиться в бесконечной самсаре из-за неведения своей
сущности, тогда как сам Самантабхадра освободился благодаря
собственному осознаванию. Количество тел шести жизнеформ,
в которых нам довелось влачить существование, не поддаётся
исчислению, и всё это было тщетно. Сейчас нам выдался единственный
шанс родить\-ся людьми, и если мы не будем практиковать методы,
позволяющие избежать рождения в низших мирах самсары, то неизвест\-но,
куда мы попадём после смерти. Но в каком бы из шести миров мы ни родились,
нам не суждено избежать страданий. Вовсе недостаточно иметь одно
человеческое тело, необходимо использовать его для настоящей духовной
практики прямо сейчас, чтобы в момент смерти вам не в чем было
раскаиваться и не было стыдно за себя. Вы должны быть как Джецун
Миларепа, который говорил: «Моя религия — это не стыдиться самого себя».
\\ \\ Но даже если вы и занялись Дхармой, вовсе недостаточно соблюдать
одно лишь религиозное обличив, — вы должны отсечь всю зависимость от
занятий и удовольствий этой жизни. Если вы её не отсечёте, начав
практиковать с колеблющимся умом, привязанным к деньгам, имуществу,
друзьям, близким и родине, то ваш ненасытный ум послужит причиной,
а объекты привязанности станут условиями для того, чтобы Мара создал
вам препятствия. А потом, попав в среду обычных мирян, дело закончится
тем, что вы бросите все благие начинания. Поэтому уменьшите свои
запросы в отношении одежды, еды и разговоров и однонаправленно
поверните ум к Дхарме без одержимости восемью мирскими заботами. Вы должны быть как Гьялва Янг Гонпа, который сказал:
\newpage
\begin{verse}[10cm]
«В уединённом месте, пронзающем сердце мыслью о смерти, \\
Практик, полностью отрешившийся от пристрастий, \\
Проводит границу, оставляя все мысли об этой жизни, \\
И его ум не трогают восемь мирских забот».
\end{verse}
\\ \\ В противном случае, смешивая Дхарму с восемью мирскими заботами, вы загубите себя таким же образом, как поедая пищу, смешанную с ядом. Эти восемь мирских забот сводятся к предвкушениям (хорошего) и опасениям (плохого), которые на самом деле являются страстью и ненавистью, если посмотреть на них изнутри. Внешне они принимают обличив духов Гьялпо и Сэнмо, с которыми вы не расстанетесь до тех пор, пока ваш ум не освободится от страсти и ненависти. В такой ситуации препятствиям никогда не будет конца. Поэтому следите за собой, вновь и вновь проверяя свой ум на наличие амбиций, восьми мирских забот и привязанностей к этой жизни, и усердно устраняйте такие пороки. Если вы держитесь за эти восемь мирских забот в глубине ума, скрывая свою личину под религиозным обличием и пытаясь ещё что-то заработать на таком коварстве, то хуже этого ничего нет.
\\ \\ Согласно поговорке, «покинув родину, вы уже реализовали половину Дхармы», поэтому оставьте вдалеке свой дом и скитайтесь по незнакомым странам. По-хорошему расстаньтесь с родными и друзьями и не слушайте их, если они будут пытаться препятствовать вашей практике Дхармы. Раздайте свои деньги и имущество в виде подаяний и живите на милостыню!
\\ \\ Развивайте в уме беспристрастие к объектам наслаждения, осознавая их как источник потенциальных препятствий. Любое удовольствие также навязчиво, как деньги и имущество, — если вы не довольствуетесь малым, то вашим запросам никогда не настанет конца, и Маре не составит труда вас совратить. Как бы вас ни хулили или восхваляли, не пытайтесь ничего опровергать или доказывать в страхе или надежде и не принимайте это за истину. Пусть говорят что хотят, как будто о давно умершем человеке.
\\ \\ Не следуйте ни за кем, кроме опытного Ламы, который доступно и грамотно объясняет необходимые вещи. Знайте себе цену и не идите на поводу у других. Гармонируйте с любой ситуацией, будьте в меру дружелюбны и не подводите других. Если кто-то создаёт серьёзные препятствия для вашей практики, будьте непреклонны как железная гора, которую пытаются сдвинуть шёлковой ленточкой. Если вы будете безвольно и податливо гнуться в любую сторону как трава от ветра, то это никуда не годится. Какую бы практику вы ни делали, с самого начала и до полного завершения соблюдайте своё обязательство и следуйте своему первоначальному обету, несмотря на удары молний сверху, наводнения снизу, камнепады со всех сторон и даже на наступающую смерть.
\\ \\ С самого начала приучайте себя к распорядку с периодически\-ми сессиями, перерывами, приёмами пищи и отдыхом, не впадая в старые привычки. Какой бы простой или сложной практикой вы ни занимались, не позволяйте себе сбиваться, прерываться и оставаться в обыденном состоянии даже на миг. Сбалансируйте темп своей практики.
\\ \\ Во время ритрита вы можете замуровать свою дверь, или же просто не встречайтесь с другими лицом к лицу, не разговаривайте и не занимайтесь никакими делами. Оставьте все отвлечения беспокойного ума, прочистите застоявшееся дыхание и примите правильную позу. Ваш ум должен находиться в осознавании своей природы, не покидая его ни на миг, как клин, вбитый в землю. Все качества и знаки практики строгого ритрита на внешнем, внутреннем и тайном уровне не замедлят появиться.
\\ \\ Вы можете подумать: «Произошло кое-что важное. Мне надо с ним встретиться и переговорить. После этого я вернусь к строгому распорядку...». В таком случае вы лишь теряете силу своей практики, которая в результате станет менее и менее дисциплинированной. Если же вы с самого начала будете решительны и непреклонны, то ваша практика будет прогрессивно улучшаться, и вас не уязвят препятствия.
\\ \\ Существует много способов исследования характеристик подходящего места для (затворничества), но если говорить в общем, самое подходящее место должно нести благословение таких мастеров прошлого, как Гуру Ринпоче или ему подобных, не посещаться нарушителями самай, находиться в удалённой местности, где легко доступны необходимые вещи и гармонировать с вашим здоровьем. Что касается практики на кладбищах и в диких местах, населяемых злыми духами, то вы можете значительно ускорить и улучшить свою медитацию благодаря внешней и внутренней взаимозависимости, если способны подчинить (негативность) силой своей реализации. Если же вы не покорите её, то возникнет много препятствий. Когда ваша реализация наберёт совершенную силу, вся негативность превратится в благоприятные условия, и на том этапе практика тайных активностей на кладбищах будет только сопутствовать вашему прогрессу.
\\ \\ Постоянно избегайте внешней занятости и внутренней суеты, находясь в состоянии, не требующем никаких действий, — это абсолютный смысл уединения.
\\ \\ Вы должны очистить свой поток бытия четырьмя общими размышлениями, отворачивающими ум (от самсары), и особыми предварительными практиками прибежища, бодхичитты, очищения омрачений и собирания накоплений (заслуги и мудрости), усердствуя в них согласно соответствующим руководствам до тех пор, пока не обретёте переживание в каждой из них.
\\ \\ В частности, вы должны положить все свои силы на гуру-йогу, считая её сердцем практики. В противном случае ваша медитация будет продвигаться медленно, и даже если слегка сдвинется с места, её преградят препятствия, так что настоящая реализация никогда не зародится в вашем уме. Поэтому молитесь с неподдельной преданностью, и через какое-то время вас без сомнений коснётся сердечная Мудрость (вашего Учителя) и невыразимая реализация взойдёт изнутри вашего ума. Как сказал Лама Щанг Ринпоче:
«Поддерживайте присутствие. Поддерживайте переживание. Поддерживайте самадхи. Хотя по этому поводу много чего сказано, самое уникальное то, что реализация появляется изнутри мощью благословения Гуру и силой преданности».
Постижение сути Великого Завершения (Дзокчен) зависит от предварительных практик, и как сказал Дже Дрикунгпа:

\begin{verse}[10cm]
«Другие учения углубляются в основную часть практики,
А мы углубляемся в предварительные».
\end{verse}
Это полностью соответствует моему собственному намерению.

\subsection{Основная часть}

\large Как устранить ошибки \\
       в воззрении, медитации, поведении \\
       и выпрямить практику
\normalsize
\\
\\
Воззрение, или взгляд — это знание реального положения вещей. Это естественное состояние абсолютной природы вашего собственного ума, которое определено в осознавании, свободном от всех интеллектуальных условностей и надуманных характеристик. Осознавание переживается в своей наготе как самосуществующая пробуждённость, оно невыразимо словами и не может быть показано на примере. Оно не ухудшается в самсаре и не улучшается в нирване. Ему неведомо возникновение и исчезновение, освобождение и заблуждение, существование и несуществование. Оно не впадает ни в крайности, ни в ограничения. Поскольку у него никогда не было материальной субстанции с характерными признаками, то его изначально чистая сущность — это всеобъемлющая пустотность. Его спонтанно присутствующая природа обладает качествами пробуждённости и не пуста как стерильный вакуум, потому что непрерывная энергия пустотности естественно проявляется во всех аспектах самсары и нирваны, подобно лучам, исходящим из солнца. Это осознавание единства пустоты и явлений воплощает в себе три Каи, будучи естественным состоянием фундаментальной природы бытия испокон веков. Взгляд Великого Завершения, лежащий за пределами интеллекта, заключается в осознавании этого реального положения вещей «так как есть». Как сказал великий Ачарья (Падмасамбхава):
\begin{verse}[10cm]
«Таковость Дхармакаи находится за пределами рассудка».
\end{verse}
Какое счастье, что мы действительно можем ухватить мудрость Самантабхадры! Это самая сердцевина 6 400 000 тантр Великого Завершения, являющихся абсолютной сутью всех 84 000 учений Победоносного Будды. Вы должны определиться, что дальше этого вам некуда идти.
\\ \\ Медитация в данном контексте — это непрерывное присутствие в этом взгляде без сомнений и недопониманий. Мы не занимаемся концептуальной медитацией, к которой относятся все интеллектуальные тренировки с ориентиром сосредоточения.
\\ \\ Не теряя стабильности вышеописанного взгляда, оставьте восприятие всех пяти органов чувств в естественном присутствии и полностью расслабьтесь. Не внушайте себе: «Вот оно!», пытаясь медитировать над этим. Если вы медитируете, то это лишь работа вашего рассудка, а на самом деле там не над чем медитировать вообще. Но и не отвлекайтесь от этого ни на миг. Настоящее заблуждение и есть отвлечение от присутствия в естественном состоянии, поэтому вы не должны отвлекаться. Позвольте возникать любым появляющимся мыслям, не пыта\-ясь их блокировать и не следуя за ними.
\\ \\ Вы можете спросить: «А что же тогда делать?» Какие бы объекты ни появлялись в вашем восприятии, оставайтесь в свежести переживания, не вовлекаясь в цепляния за воспринимаемое, подобно ребёнку, рассматривающему боголепие в храме. Оставляя все  вещи в их естественном положении, они не изменяют своего цвета и формы и не утрачивают своей яркости. Все осознаваемые явления будут переживаться в ясной и пустотной пробуждённости без ваших концептуальных цепляний.
\\ \\ Некоторые люди с посредственным интеллектом морочат себе голову множеством так называемых «глубоких и обширных» учений, не понимая их сути. Для них я укажу главный пункт, ткнув пальцем в самую сущность.
\\ \\ Между исчезновением прошлой мысли и возникновением следующей имеется небольшой промежуток. Разве в этом промежутке нет свежего присутствия сознания, — ясного и обнажённого осознавания, нетронутого никакими мысленными искажениями? Это и есть ригпа, — естественное присутствие осознавания.
\\ \\ Однако оно не находится в этом состоянии постоянно, — разве потом не возникают следующие мысли? Эти мысли являются собственным выражением или энергией осознавания.
\\ \\ Если вы не осознаете эти мысли сразу же по их возникновению, то они возвращаются в своё прежнее русло. Этот процесс именуется «цепной реакцией заблуждения», что и есть корень самсары.
\\ \\ Если же вы осознаёте сущность (мыслей) по возникновению и не следуете за ними, расслабляясь в собственной природе, то любые появляющиеся мысли равностно освобождаются в пространстве Дхармакаи осознавания. Это и есть основная практика, совмещающая воедино взгляд и медитацию Трекчо. Как сказал Гараб Дордже:
\begin{verse}[9cm]
«Это мгновенное вспоминание осознавания, \\
Появляющегося из состояния \\ \indent изначально чистого пространства,\\
Подобно находке драгоценности в глубинах океана. \\
Никто и никогда не подделает и не создаст Дхармакаю».
\end{verse}
Именно в этом вы и должны усердствовать изо всех сил, медитируя днём и ночью без отвлечений. Не оставляя пустотность в виде интеллектуального понятия, наполните её своим осознаванием.
\\ \\ Теперь кое-что о том, как медитация может извлечь пользу из вашего поведения, другими словами, как вы должны выпрямить практику.
\\ \\ Как я объяснил ранее, главное не забывать ни на миг воспринимать своего Гуру как Будду и молиться ему непрестанно из глубины сердца. Преданность названа «единственной панацеей» и лучше всего ускоряет и усиливает практику любого духовного пути, а также устраняет препятствия.
\\ \\ Что касается потенциальных изъянов в медитации, то если вы чувствуете вялость и сонливость, вам следует освежить своё осознавание. Если вы слишком возбуждены, расслабьте своё осознавание изнутри. Не приковывайте своё внимание к постоянному отслеживанию медитации, просто помните о том, чтобы не отвлекаться от осознавания собственной сущности. Поддерживайте его непрерывно как во время медитации, так и после медитации в любых делах, — во время еды, ходьбы, сна или отдыха. Какие бы мысли, эмоции и ощущения удовольствия и боли ни возникали в вашем уме, никогда не пытайтесь подавлять их противодействиями, не присваивайте и не отвергайте их в надежде или в страхе. Оставьте эти ощущения счастья и страдания в их собственной сущности «так как есть», — обнажёнными, непосредственными и одинокими. Всё это сводится к одному и тому же принципу, поэтому не морочьте себе голову множеством мыслей. Вам не нужно отдельно медитировать на пустотность, противодействуя сво\-им эмоциям и мыслям, от которых следует избавиться. Осознавая сущность того, от чего следует избавиться, эти мысли самоосвобождаются в тот же момент как свернувшаяся в узел змея.
\\ \\ Это абсолютный тайный смысл Ваджрной Сути Ясного Света. Многим известны эти слова, но они не знают как практиковать, поэтому их болтовня не отличается от слов попугая. Наша заслуга воистину велика.
\\ \\ Теперь вдумайтесь как следует, ибо вы должны понять ещё кое-что. Двойственное цепляние было вашим заклятым врагом на протяжении бесчисленных жизней, именно оно и сковывает вас в самсаре до этого момента. Сейчас ваш добрейший Учитель открыл присутствующую в вас Дхармакаю, и эта двойственность сгорает как птичий пух в огне, не оставляя и следа. Разве это не повод для радости? Если, получив такие глубокие наставления по кратчайшему пути, вы не будете их практиковать, это тоже самое, как засунуть всеисполняющую драгоценность в рот покойника, — какая потеря! Не позволяйте сгнить сущности этих наставлений, — используйте их на прак\-ти\-ке!
\\ \\ В самом начале практики ваша осознанность будет легко теряться под воздействием «чёрного растворения» мысленного отвлечения. Цепочки незамеченных мыслей будут внезапно всплывать и размножаться до тех пор, пока к вам не вернётся пробуждённая осознанность. Вы будете думать с досадой: «Я опять отвлёкся». В этот момент не надо себя укорять или исследовать причины происхождения предыдущих мыслей, — просто поддерживайте естественное присутствие той осознанной пробуждённости, которая к вам вернулась. Этого вполне достаточно.
\\ \\ Всем известны слова: «Не отвергайте мысли, разглядите в них Дхармакаю». Однако до тех пор, пока вы не развили силу ясного видения Випашьяны и остаётесь в отсутствующем состоянии Шаматхи, думая, что это Дхармакая, вы рискуете попасть под влияние инертного безразличия. В таком летаргическом состоянии нет никакого понятия о том, что происходит. Поэтому с самого начала практики вы не должны оценивать, исследовать или раздумывать над появляющимися мыслями,- просто смотрите на них в своём осознавании как старик, наблюдающий за детскими играми, — без оценок и осуждения.
\\ \\ Когда вы находитесь в естественном присутствии без мыслей, ясная и обнажённая пробуждённость, запредельная уму, восходит в тот момент, когда вы внезапно срываете с неё ощущения удовольствия и прочего. Вы не сможете избежать переживаний блаженства, безмыслия и ясности на своём пути практики, однако не считайте их чем-то важным. Если вы будете присутствовать без малейших предвкушений и опасений, то сможете устранить потенциальные ошибки.
\\ \\ Самое главное постоянно избегать отвлечений и медитировать с однонаправленной осознанностью. Если вы ограничитесь лишь интеллектуальным пониманием и непоследовательной практикой, то за исключением ощущения спокойствия у вас не появится никаких настоящих переживаний. А ваши заумные изречения не принесут вам никакой пользы. Мастера Дзогчена говорили:
\begin{verse}[10cm]
«Теории как заплатки, — они изнашиваются и отпадают»,
\end{verse}
а также: 
\begin{verse}[10cm]
«Переживания как туман, они испаряются и исчезают».
\end{verse}

Именно поэтому даже усердные практикующие порой заблуждаются из-за малейших благоприятных или неблагоприятных обстоятельств. Многие ошибались в таких ситуациях. Несмот\-ря на то, что медитация может быть коснулась вашего ума, если вы не будете практиковать постоянно, то все глубочайшие наставления останутся в книжках, а ваш ум, практика и дхарма останутся непроницаемыми. Так вы никогда и не дождётесь появления настоящей медитации. Некоторым старым практикующим с младенческой медитацией стоит быть очень осторожными, чтобы не помереть позорной смертью со слеза\-ми на лице.
\\ \\ Практикуя непрерывно в течение продолжительного времени, в силу преданности и прочих условий ваши переживания перерастут в реализацию. Вы увидите обнажённое осознавание в ослепительной ясности, переживая его открытость и блаженство, будто пелена спала с ваших глаз. Это называется «высшим видением», но там нечего видеть. С того момента ваши мысли будут появляться в качестве медитации. Как мысленное движение, так и спокойствие будут равнозначно освобождаться. В начале мысли будут освобождаться в момент их осознавания, подобно встрече знакомого человека. На следующем этапе практики мысли будут самоосвобождать\-ся как свернувшаяся в узел змея. В конце мысли будут освобождаться, не принося пользы и не причиняя вреда, подобно вору, который забрался в пустой дом. Эти три этапа будут происходить постепенно. У вас появится глубокая внутренняя уверенность и определённость в том, что все явления- это иллюзия единого самоосознавания, и вместе с этим вас переполнят волны сострадания, неотделимого от пустотности. Вы постигнете недвойственность самсары и нирваны, а также отсутствие разницы между Буддами и обычными существами. Что бы вы ни делали на этом этапе, ваш ум будет беззаботно счастливым, не покидая того самого состояния природы явлений. Эта реализация будет распространяться на день и ночь. Мастера Дзогчена говорили:
\begin{verse}[10cm]
«Реализация как небо, — она неизменна».
\end{verse}
Хотя такой йоги может казаться обычным человеком, его ум пребывает в реализации Дхармакаи, свободной от напряжён\-ных усилий, ибо он прошёл все духовные пути без нужды в каких-либо действиях. В конце концов наступает этап «истощения рассудка и явлений», аналогичный (слиянию внешнего пространства) с пространством внутри разбившегося сосуда, когда тело распадается на атомы, а ум сливается с природой явлений. Это называется стадией, когда «основное пространство изначальной Основы бытия сворачивается в юное Тело сосуда внутренней светоносности». Такое абсолютное заключение взгляда, медитации и поведения именуется «реализацией недостижимого плода». Этапы этих переживаний и реализации могут возникать как в определённой последовательности, так и без всякого порядка вообще, или же могут произойти все сразу одновременно. Всё это зависит от индивидуальных способностей людей. Однако сам плод просветления не подлежит никаким условным разделениям.

\newpage
\subsection{Заключительная часть}
\large Как соблюдать самаи, \\ объединяя Дхарму с повседневной жизнью. \normalsize
\\
\\
Несмотря на ваше усердие в практике взгляда, медитации и поведения,
незнание правильного подхода к повседневным занятиям может привести
вас к нарушению обетов и самай. В результате этого у вас сразу
появятся препятствия в практике, а в будущем вы неизбежно попадёте
в невыносимый ад Авичи. Поэтому крайне важно никогда не терять
осознанности и не ошибаться в том, что следует и не следует делать.
Как сказал великий Ачарья (Падмасамбхава):
\begin{verse}[10cm]
«Хотя моё воззрение выше неба, \\
Мои действия также скурпулёзны, как измельчённая мука».
\end{verse}
Прекратите вести себя грубо и необдуманно и будьте внимательны
к причинам и следствиям своих поступков. Соблюдайте свои самаи
и обеты без погрешения, избегая малейших нарушений и проступков.
В Тайной Мантре существует множество самай, которые сводятся к
самаям Тела, Речи и Ума коренного Гуру. Сказано, что секунда
восприятия Учителя обычным человеком отодвигает духовные
достижения на месяцы и годы. Если вам интересно почему,
то этот ключевой принцип объяснён так:
\begin{verse}[10cm]
«Духовные достижения практикующих Ваджраяну \\
Зависят только от Учителя».
\end{verse}
Кем бы вы не были до тех пор, пока настоящий Учитель
не принял вас в ученики, вы предоставлены самим себе
в полное распоряжение. Но после того, как вы создали
связь с Учителем посредством получения посвящений и
наставлений, вы не имеете права пренебрегать своими
самаями. По завершению четырёх посвящений мы традиционно
кланяемся Гуру как божественному Владыке Мандалы, говоря:
\begin{verse}[10cm]
«С этого момента я подношу себя в услужение. \\
Пожалуйста, прими меня в ученики И делай со мной всё что хочешь».
\end{verse}
У нас может быть высокое положение и статус.
Но разве, произнося такое обещание, мы не отдаём себя в распоряжение
Ламы целиком и полностью? Мы также произносим:
\begin{verse}[10cm]
«Что бы Владыка ни повелел, \\
Я исполню каждую его команду».
\end{verse}
После таких слов вы должны делать всё, что вам сказано,
разве не так? За неисполнение своих обетов вы заслужите
прозвище «нарушитель самай», как бы неприятно оно ни звучало.\\
\\
Помимо этого вам стоит знать, что не существует таких текстов,
где бы объяснялось, что самаи следует соблю\-дать чрезвычайно
прилежно только с важными, богатыми и влиятельными Ламами,
а со скромными, бедными и неизвестными йогинами можно не
соблюдать самаи вообще. Кем бы не был (ваш Учитель), важно,
чтобы вы сами понимали свою пользу и вред (от соблюдения и
нарушения самай), а иначе вы будете такими же глупыми как
старый мерин. Для чьей пользы вам нужно соблюдать эти самаи,
— для своей или для пользы Ламы? Если только для пользы Ламы,
то вы можете бросить их прямо сейчас, но если это для вашей
пользы, то не пудрите себе мозги. Думайте над этим основательно
и постоянно, как больной, регулярно принимающий лекарства.\\
\\
Что касается самай с братьями и сестрами по Дхарме, то в
общем вам следует уважать всех идущих по пути учения Будды,
тренироваться в чистом видении и избегать предвзятых воззрений
и критики. В частности, вы должны избегать всякого соперничества,
зависти, презрения, обмана и критики своих ваджрных друзей,
получавших вместе с вами те же посвящения от одного и того
же Учителя, относясь к ним как к родным.\\
\\
\newpage
Все живые существа без исключения были нашими добрейши\-ми
родителями. Как жаль, что сейчас они мучаются от страшных
страданий в бесконечной самсаре! Если мы не защитим их,
то кто ещё их спасёт? Развивайте в своём уме сострадание
с помощью таких мыслей. Стремитесь приносить им пользу телом,
речью и умом как можете и посвящайте всю добродетель на благо других.
Постоянно держите в уме только три вещи:\\
\\
\indent 1) Дхарму,\\
\indent 2) своего Учителя,\\
\indent 3) живых существ.\\
\\
Не практикуйте и не думайте о чём-то другом.\\
\\
Не соревнуйтесь и не соперничайте с разными монахами
или реализованными практикующими, держите свой ум под контролем,
а рот закрытым. Это очень важно, поэтому не валяйте дурака!
Если вы разглагольствуете о Дхарме, думая лишь о личной
выгоде в будущих жизнях, то эту Дхарму в первую очередь
необходимо практиковать вам самим. Вы конечно можете
надеяться на то, что после вашей смерти вам помогут
другие посредством добродетельных практик, но вовсе
не факт, что вы получите от этого пользу.
\\
\\
Поэтому занимайтесь внутренней работой над собой.
Прежде всего развивайте в уме глубокое отречение,
осознанно приравнивайте жизнь к практике и формируйте
базис из большого усердия. В основной части практики
концентрируйтесь на ключевых принципах глубокого
взгляда и медитации. В дополнение к этому ведите
себя в соответствии с принятыми обетами и самаями,
не путаясь в том, что надо усваивать и отвергать.
Тогда все просветлённые качества неизбежно возникнут в вас изнутри.
Великое Завершение (Дзогчен) — это такой путь, на котором даже
грешники бесповоротно достигают просветления.\\
\\
\newpage
В силу чрезвычайной глубины этих учений вы должны
быть готовы к препятствиям, которые следует понимать
как большой риск, сопровождающий большую прибыль.
Они объясняются тем, что вся негативная карма,
накопленная вами в прошлом, активизируется силой
глубоких наставлений. В знак этого возникают внешние
демонические препятствия и привидения, например:\\
\\
\small
\begin{tabular}{ll}
— & в месте вашей практики \\
  & будут появляться духи и (самсарные) боги;\\
— & они будут звать вас по имени;\\
— & они будут принимать облики Лам и давать вам пророчества;\\
— & разнообразные ужасающие видения \\
  & будут происходить в ваших снах и переживаниях;\\
— & вас могут атаковать, ограбить и обворовать другие люди;\\
— & вы можете заболеть или попасть в любые приключения;\\
— & в своём уме вы будете горевать без особой на то причины, \\
  & а также впадать в депрессию вплоть до слёз;\\
— & вас будут одолевать бурные эмоции;\\
— & ваша преданность, сострадание и бодхичитта будут уменьшаться;\\
— & ваши мысли восстанут против вас, доводя вас до сумасшествия;\\
— & вы будете превратно понимать полезные советы;\\
— & вам опостылет находиться в ритрите \\
  & и захочется нарушить свой обет;\\
— & у вас будут появляться ошибочные мысли об Учителе;\\
— & вас будут одолевать сомнения по поводу Дхармы;\\
— & вас могут обвинить в чужих грехах;\\
— & о вас могут злословить;\\
— & ваши друзья могут обернуться врагами;\\
— & у вас могут возникнуть разнообразные нежелательные \\
  &обстоятельства как внешнего, так и внутреннего плана.
  \end{tabular}
\normalsize
\\
\\
\\
Итак, вы должны понимать эти препятствия как тест для себя. Это тот самый рубеж, на котором вы либо выигрываете, либо проигрываете. Если вы справитесь с этими препятствиями посредством ключевых пунктов практики, то они превратятся в ваши достижения. Если же вы попадёте под их влияние, то они станут непреодолимой преградой для прогресса в вашей практике.
\newpage
Во всех этих ситуациях безукоризненно соблюдайте свои самаи и направьте всю веру и помыслы к своему Учителю, молясь ему с полной отдачей из глубины сердца и веря, что он знает все ваши испытания. Если вы будете относиться к неблагоприятным обстоятельствам как к желанным и практиковать с полным усердием, то через какое-то время эти обстоятельства самопроизвольно утратят свой напор, вместе с чем ваша практика значительно улучшится. Явления будут казаться рассеянными и разрозненными, и у вас возникнет ещё большая уверенность в своём Учителе и его наставлениях. Даже если в дальнейшем вам выпадут такие испытания, вы будете сохранять уверенность и душевное спокойствие, думая: «Какие проблемы?! Всё в порядке!»
\\ \\ Это и есть единственный выход из всех ситуаций. Если вы способны использовать любые обстоятельства на пути практики, то вам известен выход из всех испытаний. Как чудесно! Это как раз то, что мы, старики, вам желаем. Поэтому укрепляйте силу воли и не ведите себя подобно шакалу, которому хочется поживиться мертвячиной, но хватает храбрости лишь на тявканье у коленок покойника!
\\ \\ Некоторые индивидуумы с ничтожной заслугой, чрезвычайно ошибочными воззрениями, множеством сомнений и запущенными обетами и самаями произносят впечатляющие обещания, однако не утруждают себя практикой. Такие люди с гнилыми душами испрашивают наставления у Лам лишь для того, чтобы забыть их на своих книжных полках. Они подцепляют негативные обстоятельства и идут у них на поводу, становясь лёгкой добычей Мары, который ведёт их в направлении низших миров. Какое горе! Молитесь Учителю, чтобы не опуститься до этого.
\newpage
Можно сказать, что испытания неблагоприятными обстоятельствами проходят относительно легко. Гораздо сложнее пройти тест приятными условиями. Тут есть большая опасность, что вы возомните из себя высокореализованных и отвлечётесь на пристрастия к одному лишь величию в этой жизни. Вы должны быть очень осторожны, чтобы не попасть в услужение Маре Девапутре. Поймите, что этот рубеж разделяет движение вниз и вверх. Именно в таких ситуациях появляются гомчены, — возомнившие из себя великих практиков.
\\ \\ До тех пор, пока ваши качества внутренней реализации не набрали совершенную силу, вы должны держать рот под замком и не рассказывать истории о своих переживаниях. Более того, не распространяйтесь о количестве лет и месяцев, проведённых в затворничестве, а стремитесь посвятить всю свою жизнь практике. Не пренебрегайте добродетелью, обусловленной причинно-следственной связью, внушая себе идеи о пустотности. Не находитесь долго в людных местах, зарабатывая себе на пропитание разными популярными ритуалами покорения духов и прочим.
\\ \\ Сведите к минимуму бессмысленные занятия, ненужные разговоры и бесполезные размышления.
\\ \\ Не морочьте людям голову плутовством и обманом, что полностью противоречит Дхарме. Не зарабатывайте себе на жизнь превратным образом, льстя, подхалимничая и намекая на желаемые вещи, к которым вы привязаны. Не общайтесь с плохими друзьями, а также с теми, чьи воззрения и поведение несовместимы с вашими. Следите за своими недостатками и не обсуждайте скрытые недостатки других.
\\ \\ Вы должны бросить курить, потому что табак является проделкой демонов Дамси, нарушивших свои клятвы. Алкоголь следует умеренно употреблять в качестве субстанции самаи (во время ганапуджи), но не напивайтесь до невменяемого состояния, будьте осторожны.
\newpage
Верующие люди могут относиться к вам с почётом и уважением, тогда как неверующие могут вас оскорблять, злословить и вовлекать в распри. Используйте все хорошие и плохие отношения с другими как практику и помогайте всем без разбора своими чистыми молитвами.
\\ \\ Не падая духом, поддерживайте своё внутреннее осознавание в просторном и незаземлённом состоянии при любой ситуации, тогда как внешне ведите себя как можно скромнее. Одевайтесь в простую, поношенную одежду. Уважайте не только тех, кто выше вас по положению, но и равных себе, и даже нижестоящих. Живите без излишеств, на самом насущном, и оставайтесь в своём затворничестве в горах. Дайте своему уму установку на нищету, беря в пример мастеров прошлого. Не обвиняя свою прошлую карму, сделайте практику Дхармы чистой и безупречной. Не жалуясь на нынешние обстоятельства, практикуйте непреклонно в любых условиях.
\\ \\ Вкратце, следите за своим умом, посвятите всю свою жизнь Дхарме, и тогда в момент смерти вам не будет стыдно за себя и не останется чувства чего-то незавершённого. В этом заключается смысл всей практики. Когда подойдёт время смерти, раздайте все свои деньги и имущество и не цепляйтесь даже за такую малость, как швейную иглу. В этот момент наилучшие практикующие наполняются радостью, умеренные практикующие не испытывают страха перед смертью, а посредственные практикующие умирают без сожаления. Если реализация ясного света распространяется на день и ночь, то в этом случае нет посмертного состояния бардо, (так как просветление наступает сразу же после) разрушения физической оболочки тела. В противном случае, если вы уверены, что освободитесь в бардо, то с вами всё в порядке. Если нет, то вам следует заранее тренироваться в практике переноса сознания (пхова) и во время умирания пустить её в дело, перенеся сознание в желаемую чистую землю, где вы пройдёте остаток пути и достигнете просветления.
\newpage
В нашей драгоценной линии практикующие до сих пор достигают полной реализации на путях Трекчо и Тогал, исчезая в радужном теле ясного света. Это не сказки из прошлого. Поэтому не выкидывайте эту драгоценность, охотясь за побрякушками. Вам очень повезло, что вы встретили такие глубочайшие наставления, — сердечную кровь Дакинь. Так что возрадуйтесь, воспряньте духом и практикуйте с таким настроением.
\\ \\ Мои ученики, берегите этот текст как драгоценность, и он принесёт вам большую пользу. Я составил его в помощь медитирующим в ритрите Огмин Пема О Линг по просьбе усердного практика Рикзанг Дордже, владеющего сокровищем неразлучной веры и преданности.
\\ \\ Этот сердечный совет дал Джигдрэл Йеше Дордже в форме сущностных наставлений. Пусть он послужит причиной возникновения мудрости реализации в умах достойных людей!
\\
\\
\scriptsize
Его Святейшество Дуджом Ринпоче, дал это учение в Непале в 1983 году.
Спасибо Тулку Палсанг Церинг Ринпоче, любезно предоставившему мне этот текст.
\normalsize

