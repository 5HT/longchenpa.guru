\section{Тулку Ургьен Ринпоче.\\
Наставления по практике Дзокчен\\
в затворничестве.}
\subsection{Смысл и цель затвора}
Тибетское слово ЦАМ (mtsams), переводимое как «ритрит» или «затворничество», означает проведение границы (ЦАМ ЧАД). Эта граница имеет три аспекта — внешний, внутренний и тайный. Внешний аспект или принцип ритрита означает, что никому не позволяется пересекать территориальную границу и вторгаться в место ритрита. Тот, кто находится в затворничестве, в свою очередь не может покидать территории ритрита, оставаясь там как вкопанный. Внутренний ритрит означает полное прекращение всех мирских активностей тела, речи и ума, то есть обыденных дел, обычных разговоров и нерелигиозных мыслей. Находясь в затворничестве, вы не должны заниматься такими делами. Тайный ритрит подразумевает неотвлечение от осознавания природы ума.
\\ \\ Смысл затворничества состоит в устранении отвлечений и получении возможности целиком и полностью посвятить тело, речь и ум достижению просветления. Будда Шакьямуни провёл шесть лет в ритрите в джунглях Бодгаи. Махасиддхи Индии проводили в затворничестве по много лет. Множество великих мастеров Тибета прожили в ритрите всю жизнь.
\\ \\ Наша практика воззрения, медитации и поведения страдает от различных отвлечений. Даже если вы осознали природу ума, тем не менее вам необходимо провести много времени в ритрите для того, чтобы достичь стабильности в этом осознавании. Достижение полного просветления за одну жизнь без ритрита просто невозможно. Мастер Дзокчена 19 века Палтрул Ринпоче определил это так:
\newpage
\begin{verse}[10cm]
«До тех пор, пока вы не достигли стабильности, \\
Важно избавиться от отвлечений и практиковать. \\
Вы должны разделить медитацию по сессиям.»
\end{verse}
В соответствии с традицией Дзокчена практикующий должен оставить девять видов активности. По этой причине йоги и йогини удаляются в горы и проводят жизнь в уединении. Они полностью прекращают говорить, потому что разговоры — это основная форма общения в этом мире. Прекратив разговаривать, вы отсекаете главный источник отвлечения. Если вы хотите сделать ритрит по Дзокчену, обратите внимание на то, как Мастер Кхарак Гомчунг (11 век н.э.) описал принцип затворничества:
\begin{verse}[10cm]
«В уединённом месте, пронзающем сердце мыслью о смерти, \\
Практик, полностью отрешившийся от пристрастий, \\
Проводит границу, оставляя все мысли об этой жизни, \\
И его ум не трогают восемь мирских забот.»
\end{verse}
\\ \\ Настоящее место ритрита — это глубоко осмысленное понимание непостоянства. Мысль «я могу умереть в любой момент» должна пронзить сердце, и именно эта мысль должна стать вашей настоящей ритритной хижиной. Если у вас имеется такое сильное чувство непостоянства, то ваше отречение разовьётся самопроизвольно. Вы отрешитесь от привязанности к самсаре, видя, насколько бессмысленны и тщетны все самсарные занятия. Вы отвергнете привязанность к самсаре таким же образом, как будто вас стошнило. После того как вас стошнило, вы же не возвращаетесь, чтобы проглотить свою рвоту, — только псы могут есть блевотину! Это настоящее отношение йогинов. Отрешившись от самсарных занятий, йоги не возвращается назад.
\\ \\ Настоящий йоги думает так: «Если я умру, это нормально; если я заболею, это тоже нормально; что бы со мной ни случилось, — просто здорово. Я буду только практиковать! У меня нет времени, чтобы тратить его впустую! Мне больше ничего не осталось, как сидеть и практиковать». Именно таким образом он или она проводит границу, оставляя все заботы этой жизни. Практикующий — это не тот, кто отсчитывает месяцы или годы своего ритрита, а тот, кто сидит, полностью сосредоточив ум на практике. Только благодаря такой устремлённости Сиддхи прошлого достигали просветления. Вам тоже необходима такая решительность, если вы хотите достичь просветления за одну жизнь.
\\ \\ Если вы будете так практиковать, то вас не уязвят восемь мирских забот, а именно обретения и потери, слава и бесславие, похвала и хула, удовольствие и несчастье. Эти четыре строки Кхарак Гомчунга описывают абсолютный, внутренний ритрит.
\subsection{Десять Козырей}
Древние Мастера традиции Кадампа оставили наставления по практике истинной Дхармы в свободе от всех забот этой жизни. Эти наставления именуются «Десятью Козырями» и включают Три Ваджры, Четыре Наметки и Три Достижения. Индийский Мастер Атиша донёс эти учения до Тибета:
\begin{verse}[10cm]
«Помести перед собой ваджру необескураженности.\\
Помести сзади себя ваджру непосрамления.\\
Иди по жизни с ваджрой чистой мудрости.\\
Наметь свой ум на Дхарму.\\
Наметь практику Дхармы на нищету.\\
Наметь нищету вплоть до смерти.\\
Наметь смерть в пустой пещере.\\
Оставь общество людей.\\
Примкни к компании собак.\\
Достигни уровня богов.»
\end{verse}
\\ \\ Первый из этих наказов — «помести перед собой ваджру необескураженности» Не падайте духом, даже если ваш отец, мать, родственники или друзья пытаются отговорить вас от практики Дхармы. Даже если они вынуждают вас заниматься мирскими делами, даже если они угрожают вашей жизни, никогда не отчаивайтесь. Практикующий Дхарму должен иметь такую решимость и силу воли. Даже если ваш учитель будет пытаться препятствовать практике, вы должны знать, что такой случай является единственным исключением, когда вы можете проигнорировать команду учителя. Будьте непоколебимыми как ваджра в своей решимости практиковать Дхарму. Имейте в виду, что истинная практика привлекает много препятствий.
\\ \\ «Помести сзади себя ваджру непосрамления». Не опускайтесь до того, как заканчивают свой путь многие практикующие. Многие в начале очень усердны в практике, однако позже становятся усердными бизнесменами. Они создают тесные связи с друзьями и родными и сражаются со своими неприятелями. Они ведут себя хуже обычных обывателей. Все свои усилия они вкладывают в зарабатывание денег, создание капитала и славы. Такие люди умирают в стыде и позоре. Одна поговорка гласит: «Начинающий практик пренебрегает даже золотом. Старый практик подбирает даже ошмётки с дороги».
\\ \\ Начинающий буддист полон энтузиазма, думая: «Всё непостоянно. Я могу скоро умереть. Что толку от золота, денег и богатства? Я лучше отдам это другим». Через пару лет он осознаёт, что не развил истинного отречения. Он чувствует, что в его практике не произошло сдвига и решает вернуться к мирским занятиям. Так как у него было много времени на размышления, он становится более сообразительным и практичным в заработке денег, чем раньше. Он превращается в так называемого «старого практикующего». Такой старый практик думает: «Хм, этот кусок кожи, что валяется на обочине, ещё можно использовать для чего-нибудь. Пожалуй я его подберу, приведу в подобающий вид и выгодно продам...». Такие практикующие подбирают даже то, что не трогают и псы. К сожалению, такое случается со многими. Такие практикующие не продвигаются по пути, потому что не развили глубокого отречения, не чувствуют отвращения к самсаре и не зародили в себе понимания непостоянства. С самого начала практики Дхармы истинный буддист должен иметь сильное и непоколебимое намерение не «кончить за упокой». Это намерение должно быть неизменным как ваджра. Развивайте, отречение и отвращение к самсаре, примите в сердце ощущение непостоянства.
\\ \\ «Иди по жизни с ваджрой чистой мудрости». Ваджра чистой мудрости — это ничто иное, как пробуждённость осознавания. Пробуждённость осознавания лишена всякой нечистоты, она неизменна как ваджра. Настоящий практикующий, устремлённый осознавать природу ума непрерывно, не будет тратить даже мига на что-то другое. Он или она будет просто сидеть и практиковать, абсолютно непоколебимый в своей решимости достичь просветления в этой жизни посредством практики пробуждённости осознавания. У такого практикующего нет других интересов.
\\ \\ «Наметь свой ум на Дхарму». Раз уж вы обрели драгоценное человеческое тело, приложите все свои силы и интерес к Учению. Не тратьте свою энергию и время на другие вещи. Решите, что будете практиковать всю жизнь. Никогда не бросайте практику Дхармы, что бы ни случилось.
\\ \\ «Наметь практику Дхармы на нищету». Не становитесь богаты\-ми людьми. Не прожигайте своё время, скапливая деньги, оберегая и преумножая капитал. Не превращайтесь в рабов своего богатства. Если вы будете раболепствовать деньгам и богатству, у вас никогда не будет шанса достичь реализации в практике. Это довод в пользу бедного образа жизни, который способствует практике без множества отвлечений. Нищета в данном контексте означает, что у вас есть место для ритрита, одежда, чтобы укрыться и достаточно еды и питья, — не больше и не меньше.
\\ \\ «Наметь нищету вплоть до смерти». Оставайтесь бедными практикующими всю свою жизнь. Не превращайтесь позже в бизнесменов. Не возвращайтесь к отвлекающему образу жизни. Будьте простыми и скромными до конца.
\newpage
\\ \\ «Наметь смерть в пустой пещере.» Умирайте в одиночестве, — в пещере, в ущелье, на гребне горы или в другом подобном месте, где умирают настоящие йогины. Место смерти йоги или йогини не является объектом строительства, оно не требует ухода и не нуждается в правах на владение недвижимостью. Поэтому у йоги или йогини нет таких волнений типа: «Что случится с моим телом после смерти? Что произойдёт с моим богатством, когда я умру? Кому я должен передать своё имущество?» Никто ничего не даёт йогинам. Никто ничего от йогинов не получает. Они полностью свободны и независимы. Время смерти — это самый важный момент в жизни, и в этот момент существенно важно быть независимым, так как любая привязанность или беспокойство может повредить. Когда Джамьянг Кхьенце Вангпо (1819—1892) посетил Лхасу, он произнёс необычное посвящение перед Джово, — главной статуей Будды в Тибете:
\begin{verse}[10cm]
«Без хозяина надо мной, без слуг подо мной, \\
\indent без врагов там и без друзей здесь\\
Пусть я умру в уединённом горном затворничестве!»
\end{verse}
\\ \\ Вы должны считать это посвящение самой высшей из молитв.
\\ \\ «Оставь общество людей». Таких йогинов больше не считают нормальными людьми. Миларепа как-то сказал: «Когда я смотрю на людей, мне кажется, что они сумасшедшие, а когда люди смотрят на меня, они думают, что это я сошёл с ума.» Не поддерживая связей, йоги предпочитает покинуть людское общество.
\\ \\ «Примкни к компании собак». Такой йоги оставил все понятия о чистом и нечистом. Он может питаться вместе с собаками. Он не радуется, когда с ним вежливо общаются. Если его игнорируют, он не расстраивается. Он придерживается низше\-го положения, нося оборванную одежду и прося милостыню на пропитание. Ему безразлично, что о нём думают другие, а мирской успех, хорошую одежду, вкусную пищу и репутацию он считает пороками. Он вышел за пределы надежд и опасений. Он реализовал взгляд равностности.
\\ \\ «Достигни уровня богов». Этот йоги отбросил восемь мирских забот, и хотя он имеет тело обычного человека, его ум уже пробуждён. Он достиг уровня богов.
\\ \\ Таковы «Десять Козырей» древних Мастеров Кадампа. Эти учения, данные для настоящих практикующих, определяют стандарты для достижения просветления в течение одной жизни.
\subsection{Бродячий йоги}
\\ \\ Бродячий йоги — это тот, кто оставил все мирские занятия. Он никогда не путешествует ни на лошади, ни на машине, ни на другом транспорте. Он странствует пешком, используя длинный посох для ходьбы. Свои скудные пожитки он носит на спине. Он живёт на то, что ему подносят люди. Он не скапливает состояния, у него нет ни слуги, ни помощника. Он сам готовит себе еду. Не имея дома, он не задерживается надолго ни в одном месте. Если ему подвернётся небольшая хижина, он может пожить в ней какое-то время. Или же он может провести несколько месяцев в пещере или у подножья горы. Он передвигается от одного уединённого места к другому и не строит планов насчёт того, где и как долго он будет находиться. Если вы спросите такого йоги, где он будет через месяц, у него не найдётся для вас расписания. Он странствует непредсказуемо как парящая птица.
\\ \\ Настоящий бродячий йоги внешне свободен от стереотипов мирского поведения, а внутренне лишён мыслей, связанных с восемью мирскими заботами. Бродячий йоги также называется «скрытым йоги», так как он никогда не раскрывает свои медитативные переживания и реализацию и держит в тайне своего коренного Гуру и личную практику. Бродячего йоги также именуют «йоги простоты», ибо он странствует без религиозных причиндалов, нося с собой чётки и пару текстов, не больше.
\newpage
\subsection{Скрытый йоги}
\\ \\ Скрытый йоги — это тот, о ком не знают даже местные люди и соседи. Такой йоги никогда не рассказывает о своих качествах, ибо его качества — как раз то, что он скрывает. Раскрывая свои качества и хвастаясь о духовных достижениях, вы сможете «практиковать» лишь восемь мирских дхарм. Скрытый йоги никогда не подумает: «Так как я практикую уже много лет, настало время учить других. Мне следует поискать учеников...». Не питая ни малейшей надежды стать учителем, он естественно осуществляет свои активности на благо других существ. Такой Мастер привлекает к себе учеников среди людей и нелюдей, как ароматный цветок притягивает пчёл.
\begin{verse}[10cm]
«Тело скрыто, находясь на уединённой горе. 
\\ Речь скрыта и полностью лишена разговоров. 
\\ Ум скрыт от концептуальных понятий. 
\\ Такого именуют «скрытым йоги».»
\end{verse}
\\ \\ Если вы хотите быть скрытым йоги, то для этого имеется две традиции практики, а именно «стиль нищего» и «стиль антилопы». Что касается стиля нищего, то вы должны оставить свою страну. На родине у вас есть как друзья, так и враги, и вам следует их оставить. Отправляйтесь в другое место, чтобы никто не ведал вашего имени, откуда вы пришли, чем занимаетесь и так далее. Не заводите себе друзей. Прикиньтесь обычным человеком, не наряжайтесь и не стройте из себя йоги. Носите оборванную одежду, выброшенную другими людьми. Полностью отбросьте привязанности к еде, славе и даже к своему стилю. Вы можете поселиться на краю небольшой деревни или на отшибе в маленьком доме, не давая другим знать, чем вы занимаетесь, и не общаясь с людьми вообще. Держитесь вдали от взора и живите очень скромно, будьте никем. Если вы способны жить как нищий йоги, это значит, что вы преодолели желание показывать свои духовные качества и не нуждаетесь в дальнейших наставлениях от других учителей, обретя несомненную уверенность в наставлениях своего коренного Гуру.
\newpage
\\ \\ Если вы предпочитаете практиковать в стиле антилопы, то вам придётся отправиться в чрезвычайно удалённое место, типа склона горы, пещеры в скале, лесной чащи, вершины горы, безлюдной долины или снежных гор. Прекратите все дела. Не занимайтесь торговлей, строительством, земледелием, не ходите никуда, не делайте даже простирания и обхождения святых мест. Сидите как вкопанные и не разговаривайте ни с кем. Полностью оставьте все размышления. Не отвлекайтесь от взгляда сущностного раздела Упадеша Великого Завершения (Дзокчена). Пусть вас навещают только дикие звери. Прекратите все контакты с людьми.
\\ \\ Истинный йоги никогда не будет основывать и содержать монастыри или Дхарма центры. Тем не менее, он или она естественно осуществляет невообразимую пользу живым существам. Где бы ни находился такой Мастер, все духи, демоны и люди успокаиваются сами собой. В умах духов и демонов самопроизвольно развивается мотивация сострадания, и все их вредные намерения исчезают естественным образом. Такой Мастер является держателем «просветлённого Ума Победоносных». Он реализовал нерождённую Дхармакаю. Поскольку он никогда не отвлекается от осознавания природы ума, то и в умах духов и демонов автоматически появляется самадхи. Вся местность становится спокойной и приятной. Исчезают болезни, ссоры, недовольства и воздействия негативности. Бесчисленные существа бардо, пребывающие в пространстве, обретают покой и счастье.
\\ \\ Предвкушения и надежды стать учителем Дзокчена заведомо содержат в себе восемь мирских забот. Некоторые практикующие, пробывшие несколько лет в ритрите, внезапно приходят в восторг от своего усердия и знаний. Они начинают думать: «Теперь мне следует передать свои знания другим и стать учителем. Мне есть что преподать, и люди вероятно захотят послушать. Я должен оставить затворничество и сделать кое-что на благо других». Такие мысли появляются, когда в сердце отсутствует настоящее отречение. Истинным йоги или йогиней не становятся по количеству лет в ритрите. Сколько бы ни длился ритрит, всё равно что-то упущено, если не достигнуто прогресса в осознавании, если не появилось истинной преданности и сострадания и не родилось отречение и отвращение к самсаре.
\\ \\ Проверьте свой ум! Есть ли в нём глубокое чувство отречения? Осознали ли вы в своём сердце тщетность всех самсарных занятий? Питаете ли вы до сих пор к ним привязанность? Всегда ли вы помните о непостоянстве? Относитесь ли вы к каждой минуте как к последней в своей жизни? Есть ли у вас истинное отвращение к самсаре? Чувствуете ли вы в своём сердце, что все существа на самом деле подобны вашему отцу и вашей матери? Наворачиваются ли на ваши глаза слёзы в тот миг, когда вы вспоминаете своего коренного Учителя? Всегда ли вы присутствуете в пробуждённости осознавания без отвлечений? Имеется ли у вас чистое восприятие всего окружающего?
\\ \\ Когда достигнута стабильность в осознавании природы ума, то мудрость, любовь и просветлённые активности проявляются спонтанно и без усилий. В этом и заключается смысл фразы: «Знание одного освобождает всё». Это состояние за пределами надежд и опасений, присвоения и отвержения. Тот, кто достиг стабильности, не будет сражаться с другими людьми, видеть нечистоту и потворствовать восьми мирским дхармам, — это просто невозможно.
\subsection{Скромность}
\\ \\ Скрытый йоги, следующий примеру Мастеров прошлого, не будет учить вообще, пока не достиг стабильности в пробуждённости осознавания. Даже если его попросят до того момента, он откажется давать учения. Когда он больше не отвлекается от осознавания ни на миг, его активности на благо других разворачиваются спонтанно. Большая опасность состоит в прекращении йогического образа жизни до тех пор, пока не завершён Путь.
\\ \\ Говоря в общем, человек может учить Дзокчену после того, как достигнут уровень, называемый «определённостью в естественном состоянии». Это минимальное требование. Всякий так называемый «учитель Дзокчена», не достигший этого уровня, является фальшивкой. Тогдэн Шакья Шри (Мастер Дзокчена 1853—1913) настаивал, что начинать преподавать Дзокчен следует только после достижения третьего видения Дзокчена, именуемого «Достижением Кульминации Осознавания».
\\ \\ Уровень «определённости в естественном состоянии» подразумевает способность объединять практику осознавания с повседневными делами. На этом этапе практикующий не подвержен ни негативным, ни позитивным обстоятельствам, и не теряет осознавания даже в крайних ситуациях. Все сомнения по поводу осознавания полностью исчезли, и все обстоятельства только способствуют практике ригпа. На этом уровне единое непрерывное присутствие без усилий в обнажённом осознавании длится минимум двадцать минут, в течение которых нет ни малейшего отвлечения. В этом случае йоги действительно реализовал неизменную природу пробуждённости осознавания.
\begin{verse}[10cm]
«Даже если бы отвлёкся, сущность всё равно неизменна.\\
Хотя нет отвлечений, \\ \indent ты свободен от ориентации и сосредоточения.»
\end{verse}
\\ \\ Настоящий йоги поддерживает осознавание сущности, занимаясь повседневными делами. Выражение осознавания способно полноценно функционировать в любой ситуации без потери осознавания своей сущности. Благодаря поддерживанию осознавания сущности отсутствует какое-либо сосредоточение и ориентация. Когда вы достигнете этого уровня в своей практике, тогда вы будете носить имя «йоги». На тибетском слово «йоги» звучит НАЛДЖОРПА и означает того, кто тренирует свой ум в подлинной реальности (НАЛМАЙ ДОНРАНГ ГИ ГЬЮЛА ЙОРПА).
\newpage
\\ \\ Йоги справляется с негативными ситуациями относительно легко. Но даже хорошие йоги, которые определились в естественном состоянии, могут оплошать в положительных обстоятельствах. Кажется, что ваша практика осознавания продвигается, у вас много учеников, которые вас обожают и засыпают подношениями, и вы чувствуете, что приносите большую пользу живым существам. Если вы не будете поддерживать осознавание во всех этих положительных обстоятельствах, вы можете легко возгордиться собой. Даже не заметив, вы попадёте под влияние Мары. Если такое случится, немедленно бросьте свои активности. Удалитесь в уединённое место, где вас никто не знает и делайте лишь практику простоты.
\\ \\ В наши дни некоторые Тулку и Ламы попадают в зависимость от своего положения учителей и глав монастырей. Они начинают учить Дзокчену ещё до того, как достигли завершения Пути. Их просят дать учение, и они думают, что уже достаточно реализовали Дзокчен для этого. Они не осознают, что попадают под влияние отвлечений. Древние Мастера нашей линии никогда не учили преждевременно. Они оставались скрытыми йогинами до тех пор, пока не достигли стабильности. Берите этих древних Мастеров в пример и не обращайте внимания на то, что творится сейчас.
\subsection{Независимость от других}
\\ \\ Обычно мы зависим от других, рассуждая так: «Если я не пойду к нему, он может рассердиться. Если я не дам ей что-то, она расстроится...». Мы считаем хорошим тоном поддерживать отношения. Но великие Мастера типа Миларепы, Лонгченпы, Джигме Лингпы или Палтрула Ринпоче вовсе не считали важными ни отношения, ни дружбу, ни богатство, ни одежду, ни еду. Они не ходили на вечеринки, не ублажали друзей подарками и не заботились о том, как о них будут думать в обществе. Они не были оторваны от своей практики, так как не зависели от других. Они полагались только на практику осознавания. У них не было другого интереса, кроме достижения просветления в течение одной жизни.
\\ \\ Внешне избегайте отвлекающих самсарных занятий. Внутрен\-не не позволяйте своему уму отвлекаться от практики. Если вы избавитесь от внешних и внутренних отвлечений, ваш ум станет очень ясным. В таком ясном уме легко появятся качества отречения, сострадания, преданности и медитации. Чтобы достичь этого, находитесь в затворничестве. Лучше раскрепоститься в истинном осознавании одну минуту, чем провести год в относительной медитации. Поэтому очень важно быть вне всяких сомнений в отношении истинного взгляда (Трекчо). Если вы действительно осознали природу ума, будьте уверены в наставлениях своего коренного Гуру и не бегайте за каждым учителем и на каждое учение. Живите сами по себе и практикуйте.
\\ \\ Практикуйте до тех пор, пока не достигнете стабильности. Вначале вам нужно осознать сущность, затем усовершенствовать силу этого осознавания, и в конце концов достичь в нём стабильности. Стабильность означает отсутствие малейшего отвлечения, даже на секунду. Когда достигнута стабильность в осознавании, это и есть полное Просветление, и невообрази\-мые качества мудрости, любящей доброты и просветлённых активностей возникнут спонтанно и одновременно с этим. Не спешите преподавать до тех пор, пока не достигнете этого уровня. Не берите в пример современных учителей.
\subsection{Линия благословения}
\\ \\ До тех пор, пока вы полностью не определились во взгляде (Трекчо), пока у вас имеются тонкие сомнения, делайте только короткие ритриты. Встречайтесь со своим коренным Гуру вновь и вновь. Проверяйте свой ум: «Есть ли у меня какие-нибудь предвкушения или опасения? Есть ли в моём воззрении цепляния и привязанности?»
\newpage
\\ \\ Вы должны достичь уровня, называемого «определённостью в естественном состоянии». Это уровень весомой стабильности в осознавании природы ума. Когда исчезнут все сомнения, ваша преданность к коренному Гуру станет безграничной.
\\ \\ Тем не менее, не находитесь слишком близко к своему коренно\-му Гуру. Поговорка гласит: «Гуру подобен огню. Не подходи слишком близко.» Если вы будете слишком близки с ним, вы можете развить превратные идеи и начать сомневаться в нём, испортив тем самым свои самаи. Если же вы будете вдалеке от него какое-то время, делая ритрит, то ваша преданность и чистое восприятие Гуру только увеличится. Если вы действительно постигнете обнажённое состояние осознавания, вы будете воспринимать своего коренного Гуру как Будду. Природу Будды можно осознать только благодаря доброте коренного Гуру. Поэтому сказано: «Доброта Гуру больше доброты Будды». Однажды Миларепа сказал Гамбопе:
\begin{verse}[10cm]
«Настанет время, когда в силу своей великой преданности
\\ \indent ты увидишь во мне настоящего Будду. \\
В тот миг в твоём уме
\\ \indent взойдёт истинная реализация неподдельной Махамудры.»
\end{verse}
\\ \\ Редко встретишь такого чистого йоги.
\\ \\ Учение Великого Завершения (Дзокчен) именуют «линией благословения». Эта линия содержит благословение передачи просветлённого Ума Победоносных. Когда благословение Гуру касается ума ученика, тот открывает в себе обнажённое осознавание, то есть Дхармакаю. Такая передача целиком и полностью зависит от преданности ученика. Если ученик с чистой преданностью встречает квалифицированного Мастера, передача просветлённого Ума Победоносных безусловно состоится. Но если нет стабильной и чистой веры, то результат передачи будет под вопросом. Это ключевой момент в Дзокчен. В учении Великого Завершения благословение Гуру должно войти в ум ученика. Чтобы это произошло, ученик должен иметь безграничную преданность и чистое видение Гуру.
\newpage
\\ \\ В линии истинных Мастеров Дзокчена благословение ничуть не уменьшилось сквозь века. Так как линия никогда не прерывалась, то ученик до сих пор может получить такое же благословение, какое Прахеваджра (Гараб Дордже, 4 век до н.э.) получил от Ваджрасаттвы. Линия Дзокчена напоминает длинный водопровод. Если труба в целости, свежая вода из источника дойдёт до её конца. Что способствует целости линии? Совпадение «трёх совершенных условий»: совершенный учитель, совершенные наставления и совершенный ученик.
\\ \\ Дрикунг Кьобгон Ринпоче (1143—1217) сказал:
\begin{verse}[10cm]
«До тех пор, пока солнце преданности не озарит 
\\ Снежные вершины четырёх Кай Гуру, 
\\ С них не хлынет поток благословения. 
\\ Поэтому усердно развивайте преданность в уме.»
\end{verse}
Осознание сущности, совершенствование силы осознавания и достижение стабильности происходит лишь благодаря благословению Гуру. Как я уже сказал, благословение в линии Дзокчена абсолютно цело и невредимо, начиная с Будды Самантабхадры и заканчивая вашим коренным Гуру. Эта линия подобна жемчужинам, нанизанным на золотой цепочке. Силой искренней преданности вы собираете накопления (заслуги и мудрости) и очищаете омрачения. Если накопления не собраны, а омрачения не очищены, истинное осознание приро\-ды ума невозможно. В тантре говорится:
\begin{verse}[12cm]
«Вы должны знать, что идиотизм –
\\ Это использование любых методов,
\\ Кроме практик собирания накоплений, очищения омрачений
\\ И благословения святого коренного Гуру.»
\end{verse}
\\ \\ Также там сказано:
\begin{verse}
«Лучше вспомнить своего Гуру на секунду, 
\\ Чем медитировать сто тысяч кальп
\\ На божество, обладающее главными и второстепенными признаками. 
\\ Лучше один раз помолиться своему Ламе, 
\\ Чем читать миллионы мантр Приближения и Достижения.»
\end{verse}
Говоря в общем, есть четыре типа преданности: восхищённая преданность, устремлённая преданность, уверенная преданность и необратимая вера. Если вы пришли в восторг, которого не испытывали раньше, узнав о невообразимых качествах своего коренного Гуру, это называется восхищённой преданностью. Если вы стремитесь избавиться от страданий низших миров и желаете подражать качествам просветлённых Масте\-ров, это именуется устремлённой преданностью. Если вы глубоко убедились в сверхъестественном благословении свое\-го коренного Гуру, и в вашем уме появилась прочувствованная уверен\-ность, это уже уверенная преданность или «вера со знанием причины». Эти три типа преданности могут появляться и исчезать. Но если вы обрели преданность, которая никогда не покидает ваш ум, это называется необратимой верой, нерушимой верой или неразлучной верой.
\\ \\ Поймите, что ваше блуждание в самсаре в течение бесчисленных кальп происходит лишь из-за неведения обнажённого состояния осознавания. Силой благословения и наставлений Гуру сейчас вы должны осознать свою просветлённую сущность вне сомнений.
\\ \\ Это Дхармакая, просветлённый Ум всех Будд. Осознавание также называется «самодостаточным царём» и «единым знани\-ем, освобождающим всё». Истинный коренной Гуру — это тот, кто ввёл вас в обнажённое состояние осознавания, после чего у вас не осталось никаких сомнений. Если вы действительно осознали неподдельную, обнажённую природу ума, вы никогда не потеряете преданности к своему коренному Гуру и не ограничьтесь в своей практике. Если вы глубоко постигли непостижимую доброту своего коренного Гуру, у вас появится необратимая вера.
\\ \\ Каждый имеет природу Будды, но необходимо ещё научиться её осознавать. Даже если вы имеете представление о том, как её осознавать, одно лишь такое знание вас никуда не приведёт. Вы должны определиться в ней, и для этого вам необходим коренной Гуру, который поможет углубить уверенность в просветлённой природе. Он объяснит вам разницу между (подсознанием) все основы и Дхармакаей, различие между двойственным умом и осознаванием, а также разницу между сознанием и пробуждённостью. Без наставлений и благословения истинного мастера линии вы никогда не осознаете и не определитесь в этом. Вот почему каждому нужен Гуру.
\\ \\ Учителя бывают трёх видов: «учителя, дающие посвящения», «учителя разъясняющие тантры» и «учителя, передающие устные наставления». Все эти мастера должны восприниматься более особенными, чем Будда, или по крайней мере равными Будде. Вы должны считать учителя Четвёртой Драгоценностью. Служите своему Гуру телом, речью и умом, следуя его командам без обмана. Никогда не допускайте злых намерений и плохих мыслей в его сторону. Никогда не вредите ему, не оскорбляйте его и не противодействуйте его наставлениям даже в малом. Цените мастера больше, чем своих родителей, и дороже, чем свои глаза. Служите ему как самому Будде и угождайте ему тремя усладами — лучше всего своей практикой, затем услужением, и после этого материальными подношениями. Воспринимайте своего коренного Гуру как воплощение всех мастеров линии. Когда тысяча звёзд отражается на поверхности одного озера, нет нужды завлекать каждую из них по отдельности. В этом примере озеро символизирует коренного Гуру, тогда как все мастера линии сравниваются со звёздами. Поддерживайте чистое видение и преданность к своему коренному Гуру всё время.
\\ \\ Чем больше вы будете практиковать, тем больше станет ваша вера к своему коренному Гуру. Когда вы достигнете такого уровня, что увидите в Гуру самого Будду собственной персоной, у вас появятся такие мысли: «Он даровал мне глубочайшие наставления, которые я должен использовать на практике и обрести опыт и реализацию. Он передал мне безупречную Дхармакаю, просветлённый Ум Будды, и большей доброты просто нет. Только благодаря благословению и доброте Гуру я смог осознать Дхармакаю. Эту реализацию не могут дать ни родители, ни друзья, ни деньги, ни власть. Во всей вселенной нет большей доброты и нет более глубоких наставлений». Когда вы думаете так о коренном Гуру, слёзы преданности должны потечь сами собой.
\\ \\ Когда великие мастера объявляют имя своего Гуру в начале посвящения, они порой не могут продолжить чтение. Они сидят и рыдают перед учениками, думая о коренном Гуру. Такое посвящение очень драгоценно, так как подобная переда\-ча — это царское посвящение.
\\ \\ В момент сильной преданности и большого сострадания вы должны реализовать своё осознавание. В этот миг осознавание абсолютно чисто, это и имел в виду Кармапа Рангджунг Дордже (1284—1339), сказав:
\begin{verse}
«В момент любви \\ \indent пустотная сущность восходит в наготе.»
\end{verse}
\\ \\ Под словом «любовь» здесь подразумевается как преданность, так и сострадание. «Момент» означает время, когда ваша преданность встречается с благословением Гуру. Не остаётся даже самых тонких понятий. Все мысленные движения полностью исчезают, также как при произнесении слога «ПХЭТ!». В этот миг остаётся лишь одна обнажённая пустотность, без ошибок и омрачений, как полностью чистое небо. Преданность и сострадание- это наилучшие средства для улучшения практики. Древние мастера Кагью говорили: «Преданность — единственная панацея». К такой практике невозможно подойти путём философии.
\subsection{Непостоянство и отречение}
\\ \\ Все мы знаем, что должны умереть, и всё же уповаем на долгую жизнь. Имейте в виду, что время смерти неопределено. Люди умирают во время ходьбы, разговоров, сна, приёмов пищи, сидя, стоя и так далее. Вы можете умереть в любой момент, примите эту правду всем сердцем. Поэтому нечего тратить время. Каждая ваша минута может оказаться последней. Думайте так: «Вот и подошла моя смерть, моё время истекло. Мне больше ничего не осталось как практиковать». Если у вас есть такое отношение, то вы реализуете Учение. В противном случае нет. Где бы вы ни были, вы всегда расстаётесь с любимыми, встречаете врагов, болеете, и в конце концов умираете. Каждый рождённый должен умереть! Всё, что скоплено, должно расточиться. Всё, что соединено, должно разъединиться. Все, что построено, будет разрушено.
\\ \\ Если вы хотите, чтобы у вас появилось глубокое отречение, думайте о бесполезности самсарного существования. Размышляйте так: «Мирская жизнь тщетна, друзья бесполезны, враги ещё более бесполезны, богатство и добро бессмысленны, слава и удача тщетны. Куда ни глянь, везде одна лишь бессмысленность самсары». Заблуждение означает видеть смысл в том, что бессмысленно.
\\ \\ Некоторые практикующие испытывают отречение и теряют привязанность к самсаре. Но позже они вовлекаются в буддийские дела и политику. Как сказал Гамбопа:
\begin{verse}[10cm]
«Если вы не практикуете Дхарму в соответствии с Дхармой,
\\ \indent то эта же дхарма послужит причиной перерождения в низших мирах».
\end{verse}
Если у вас есть истинное отречение и чистая мотивация Бодхичитты,
то ваша практика соответствует Дхарме. Однако, если вы практикуете,
надеясь на получение еды, одежды и репутации, и намереваясь поиметь
приход от Дхармы, то такой вид практики «дхармы» станет причиной для
рождения в трёх низших мирах. Как сказал великий Мастер Сакья
(Сачен Кунга Ньингпо, 1092—1158):
\begin{verse}[10cm]
«Если ты привязан к этой жизни, ты не практикуешь Дхарму.
\\ Если ты привязан к самсаре, у тебя нет отречения.
\\ Если ты привязан к личной выгоде, у тебя нет Бодхичитты.
\\ Если у тебя имеются цепляния, то отсутствует взгляд».
\end{verse}
Любой, кто привязан к этой жизни или имуществу, кто ходит на работу, содержит семью или сражается с врагами, не может практиковать Дхарму по настоящему, и его/её нельзя назвать практикующим. Если вы не отсекли привязанности к самсарной жизни изнутри, вы никогда не испытаете истинного отречения. Тот, кто думает: «Мне надо жениться. Я хочу иметь детей. Мне нужен хороший дом. Ещё мне нужна хорошая работа...», проведёт всю свою жизнь в оковах самсарных привязанностей. Если человек думает только о личной выгоде и целях, не заботясь о нуждах других людей, то у него отсутствует сострадание. У него нет мотивации Бодхичитты. Если практикующий сохраняет тончайшие цепляния в медита\-ции, то он не осознал исконный взгляд, который свободен от цепляний как чистое небо. Цепляние — это мыслительный процесс, а посредством концептуального мышления вы никог\-да не реализуете взгляд.
\begin{verse}
«Сказано, что прекращение привязанностей — это стопы медитации».
\end{verse}
Осознавание непостоянства жизни ведёт к утомлению от неё. Вы должны понять, что блуждали в самсаре жизнь за жизнью в течение бесчисленных кальп. Тогда вы отвернётесь от привязанностей к самсаре и вашем уме появится истинное отречение. Размышляйте над самсарой таким образом: «Я блуждал в самсаре жизнь за жизнью, и что мне это дало? Что толку было от всех моих занятий? Разве все мои дела не тщетны? Может всё-таки будет умнее увидеть их бессмысленность?» Если вы будете думать таким образом, то вас можно назвать практикующими.
\\ \\ Когда эти мысли станут частью вас самих, тогда ваш ритрит принесёт успех. Если же вы проведёте в затворничестве недели, месяцы или годы, не обретя понимания непостоянства, усталости от самсары, отречения, истинной преданности и сострадания, то ваше время пройдёт впустую. Имейте в виду, что осознавание природы ума — это единственный способ освободиться из круговорота самсары. Как я уже сказал, вначале осознайте сущность ума, затем тренируйтесь в этом осознавании и в конце концов достигните стабильности.
\begin{verse}[10cm]
«Неотвлечение названо основной частью медитации».
\end{verse}
До тех пор, пока вы не достигли стабильности в осознавании, вам следует практиковать «неотвлечение от немедитации» короткими промежутками, повторяемыми много раз. Когда вы достигнете стабильности, не отвлекаясь от осознавания ни на миг, то вы освободитесь от кармы и беспокоющих эмоций. Но покуда вы на пути, где короткие отрывки осознавания чередуются с периодами отвлечения, ваши эмоции не искоренены до конца. До тех пор, пока полностью не устранены беспокоющие эмоции, у вас сохраняются зачатки самсарных рождений.
\\ \\ Практикующему необходимы три вещи: неподдельная преданность, «направленная вверх»; неподдельное понимание непостоянства, «направленное вниз»; и между ними йога непрерывного потока, то есть усердие в неотвлекаемой немедитации. Такое непрерывное усердие может быть достигнуто только тогда, когда понимание непостоянства стало частью вас самих. Думайте про себя: «Я могу сегодня умереть. Это может быть мой последний день, последний час. И до сих пор я не достиг стабильности. Как я могу тратить время на что-либо, кроме осознавания природы ума?» Практикуйте миг за мигом, не теряя ни секунды.
\\ \\ Большинство людей наполовину мирские и наполовину духовные, а практика Дхармы для них — нечто типа хобби. Они считают важными все свои дела, видя смысл в том, чем занимаются. Такие люди не могут подняться до уровня «йогинов простоты». Непреодолимые привязанности к мирс\-ким делам и сложностям придают полной простоте йогической жизни устрашающий вид. Поэтому они ищут компромисс, подгоняя свою практику Дхармы и образ жизни под свой уровень отречения. Вначале они должны собрать всё необходимое для того, чтобы посвятить жизнь практике, — достаточно денег, ритритный дом в уединённом месте и прочее. Затем, получая наставления от опытного мастера и храня их в сердце, они должны применять их в ритритной практике, время от времени возвращаясь к учителю за разъяснениями. Таким образом они смогут медленно освободиться от мирских привязанностей, никогда не забывая молиться учителю:
\begin{verse}
«Благослови меня,
  \\ \indent чтобы я смог отсечь все привязанности 
        \\ \indent \indent одним взмахом».
\end{verse}
\newpage
Именно так и сделал Будда Шакьямуни, равно как и другие великие Мастера прошлого, — отрубив единым взмахом все привязанности.
\\ \\ Если у практикующего нет истинного отречения, но он пытается подражать стилю бродячего йоги или йоги простоты в силу энтузиазма новичка, то есть шанс, что он не выдержит. Через какое-то время он может вернуться к самсарной жизни. Самсарные связи должны быть отсечены изнутри, а не подражанием чьему-то стилю. Поэтому дайте своей практике осознавания улучшиться с годами и вы разовьёте сильное отречение естественным образом. Вот тогда и настанет время для йогической жизни. Это безопасный путь. Когда вы повзрослеете, вы сможете стать хорошими практикующими. Невозможно достичь просветления за одну жизнь, находясь в отвлечении. Рано или поздно вам придётся оставить все отвлечения. Всегда держите при себе мысль о непостоянстве. Это мой совет.
\\ \\ Если учитель с самого начала будет давать суровые наставления, то ученики могут испугаться. Поэтому искусный учитель использует «четыре метода привлечения». Вначале он предоставляет помощь и дарит подарки. Затем учитель рассказывает истории о достижениях и пользе от практики Дхармы. Потом учитель показывает ученику хорошее место, чтобы тот чувствовал себя удобно, и руководит его практикой. В конце учитель вводит ученика в абсолютное состояние и объясняет, что необходимо для достижения просветления в этой жизни.
\\ \\ Когда вы практикуете осознавание природы ума в ритрите, время от времени напоминайте себе: «Это может быть мой последний шанс для практики. Я могу умереть прямо во время этой сессии». Без таких постоянных напоминаний о непостоянстве вы далеко не продвинетесь в своей практике Дзокчена. Дзокчен без отречения, преданности и сострадания — это извращение Дзокчена. Если ваша практика осознавания слишком слаба, вам следует преднамеренно напоминать себе о непостоянстве. Целенаправленно развивайте преданность и сострадание. Говорится, что «надуманное ведёт к ненадуманному». Вдохновляйте себя такими размышлениями и совмещай\-те их с практикой осознавания. Не теряя присутствия осознавания, позволяйте выражению осознавания рассматривать эти моменты. Если бы осознавание ограничивалось одним лишь пустотным аспектом, то конечно же, без потери осознавания было бы невозможно рассматривать эти положения. Но поскольку осознавание неограниченно, вы можете поддерживать присутствие осознавания в то время, как его естественное выражение постигает смысл этих моментов.
\\ \\ Впечатляющие видения, получения пророчеств от Дакинь, сверхестественное ясновидение, полёты в небесах и прочее — всё это ничто по сравнению с истинным пониманием непостоянства, истинным отречением, преданностью и состраданием. Йоги, обладающий этими качествами, никогда не ошибётся в практике. Это настоящие результаты ритрита. Тот, кто бахвалится годами практики, медитативными переживаниями и высоким воззрением, уже одержим демонами. Если вас восхищают собственные достижения, вы сбились с пути. Неподдельная преданность, сострадание и отречение не дадут заблудиться истинному йоги.
\subsection*{Структура затворничества}
\\ \\ Существует три типа ритрита: рецитация (мантр) по их количеству, рецитация по времени и рецитация до появления знаков. Рецитация по количеству подразумевает, что вы находитесь в затворничестве до тех пор, пока не начитаете определённого количества мантр. Рецитация по времени значит, что вы находитесь в строгом ритрите установленный срок — неделю, два месяца или три года. Рецитация до знаков подразумевает пребывание в ритрите до появления знаков достижения. Знаком может быть встреча с божеством в ясном сне, в видении или лучше всего — собственной персоной.
\\ \\ В ритрите вы определяете время для сессий медитации и для перерывов между ними. В некоторых традициях день разделяется на две, три, четыре или даже шесть сессий. В традиции Ньингтик мы разбиваем день на четыре сессии:
\\
\indent 
\begin{tabular}{ll}
1 & сессия до рассвета;\\
2 & утренняя сессия;\\
3 & послеобеденная сессия\\
4 & и вечерняя сессия.\\
\end{tabular}
\\
\vspace{0.5cm}
\\
Вставайте рано, чтобы начать практику в 03.00. Завершайте эту сессию в 06.00 до восхода солнца. Затем следует перерыв, во время которого вы завтракаете. Вы можете прочесть вслух «Сокровищницу Дхармадхату» Логченпы («Дхармадхату Коша») и расслабиться на утреннем солнце. С 08.00 до 11.00 вы делаете утреннюю сессию. Затем обедаете. Если хотите, почитайте книги по Дхарме и отдохните. С 14.00 до 17.00 вы делаете послеобеденную сессию. После неё сделайте подношения Защитникам (Дхармапалам) и прочтите молитвы и посвящения. Затем вы ужинаете. День заканчивается после вечерней сессии, длящейся с 19.00 до 22.00. Если вы распределите день таким образом, то привыкнете к практике, а ваш ум обретёт необходимую дисциплину.
\\ \\ Начинающие практикующие могут делать четыре сессии по два часа каждая: ранним утром с 05.00 до 07.00, утром с 09.00 до 11.00, после обеда с 14.00 до 16.00 и вечером с 19.00 до 21.00. В будущем вы сможете постепенно довести свои сессии до трёх часов. В общем говорится, что начинающий должен отдыхать во время перерывов между сессиями, изучать тексты Дзокчена и развивать преданность и отречение, читая истории из жизней великих мастеров. Не читайте обычную, литературу в ритрите. Ключевой момент в практике Дзокчена — это «многократное повторение коротких промежутков, подобно тому, как протекает крыша в старом доме». Эта практика осуществляется как во время сессий, так и во время перерывов.
В «Трёх Словах, Бьющих в Точку» (ЦИКСУМ НАДДЕК) Палтрула Ринпоче говорится:
\begin{verse}[10cm]
«Нет разницы между медитацией и пост-медитацией.
\\ Нет различия между сессиями и перерывами.
\\ Оставайтесь непрерывно в неделимом состоянии».
\end{verse}
\newpage
\subsection{Подготовка к практике}
\\ \\ Перед началом сессии прочистите нос и горло, умойтесь, откройте окно и проветрите свою комнату. Затем закройте дверь, убедившись, что сделали всё необходимое, чтобы больше не покидать места медитации и не прерывать практику. Потом расположитесь на своём месте. Расслабьте изнутри как тело, так и ум. Примите твёрдое решение не прерывать свою сессию, что бы ни случилось. Подумайте так: «В этой сессии я не буду поддаваться отвлечениям. Я не буду отвлекаться от немедитации». Несмотря на то, что вы так думаете только один раз в начале каждой сессии, эта решимость должна распространяться на всю практику осознавания (во время сессии).
\subsection{Прочищение застоявшегося дыхания}
\\ \\ Вы должны прочищать застоявшееся дыхание в начале каждой сессии. Из-за своей кармы и беспокоящих эмоций вы приобре\-ли обычное тело, речь и ум. У вас также есть нечистые (энергетические) каналы, праны и бинду. Комбинация вашего ума и праны способствовали цепной реакции ваших омрачений, кармических тенденций и грехов с безначальных времён. После того, как очищен ум и прана, пробуждённость осознавания восходит естественным образом, и вместе с ней проявляют\-ся каналы, праны и бинду мудрости.
\\ \\ В начале каждой сессии мы очищаем нечистые каналы, праны и бинду, омрачённые эмоциями. Примите семичленную позу Вайрочаны или позу «расслабления в природе ума» (которая отличается от позы Вайрочаны тем, что ладони покоятся на коленях, а взгляд направлен в небо). Затем прочистите застоявшийся воздух три раза через правую ноздрю, три раза через левую ноздрю и три раза через обе. Это называется «девятикратным прочищением застоявшегося дыхания». Также можно делать и трёхкратное прочищение застоявшегося дыхания, выдыхая один раз через правую ноздрю, один раз через левую и один раз через обе ноздри.
\\ \\ Начинайте прочищение с правой ноздри, так как омрачённая прана правой стороны (тела) сильнее левой. Зажимая левую ноздрю указательным пальцем левой руки, мощно выдохните застоявшийся воздух через правую ноздрю. Представляйте, что через правую ноздрю устраняются эмоциональная грязь, карма, клеши, грехи, омрачения, болезни, нарушенные самаи, привычные тенденции и препятствия медитации, как то сонливость и возбуждение, то есть все негативные элементы, содержащиеся в омрачённых каналах, пранах и бинду. В то же время визуализируйте, как препятствия для практики этого глубокого пути покидают ваше тело в виде чёрного дыма и исчезают далеко в небе.
\\ \\ После устранения всех негативных элементов делайте вдох, думая, что мудрость, любовь и просветлённые качества и активности всех Будд входят в вас в виде радужного света. Представляйте, что ваши каналы, праны и бинду мудрости наполняются благословением ваджрного Тела, Речи и Ума всех Будд.
\\ \\ Повторите эту процедуру, зажимая правую ноздрю своим правым указательным пальцем. В конце прочистите дыхание через обе ноздри: выдыхая, распрямляйте пальцы рук, покоящиеся на коленях; вдыхая, некрепко сжимайте пальцы в кулаки, слегка притягивая кисти к телу.
\subsection{Мотивация}
\\ \\ Прекратите все отрицательные мысли и выведите свой ум из безразличного состояния, развивая мотивацию Бодхичитты. Думайте следующим образом:
\\ \\ «В этой сессии я буду практиковать глубокие наставления Великого Завершения на благо бесчисленных существ. Пусть все существа освободятся от причин и следствий страдания. Пусть они достигнут совершенного просветлённого состояния».
\newpage
Развивайте эту мотивацию Бодхисаттв в начале каждой сессии. Сущность Бодхичитты, или пробуждённого ума, — это осознавание пустотности, тогда как её проявление — это сострадание.
\\ \\ Затем развивайте тантрическое отношение, думая: «Всё видимое, слышимое и мыслимое — это проявление осознавания. Всё это — безграничная чистота».
\subsection{Призывание коренного Гуру}
\\ \\ Над своей головой вы представляете белый лотос с сотней тысяч лепестков, поверх которого лежит солнечный диск и лунный диск. На этом восседает ваш великолепный коренной Гуру. Вы можете визуализировать его как в нормальном облике в монашеских или тантрических одеяниях, так и в форме Самантабхадры, Ваджрасаттвы, Гуру Ринпоче, Лонгченпы или другого мастера или божества. В любом случае, всегда воспринимайте его в своей сущности как настоящего Будду.
\\ \\ В Дзокчене молитва к коренному Гуру осуществляется с пятикратным отношением: коренной Гуру воспринимается в качестве Дхармакаи Будды; его активности воспринимаются как деяния Будды; его доброта к вам считается большей, чем доброта Будды; Гуру считается воплощением всех объектов прибежища; молясь Гуру с таким отношением, реализация зарождается в уме без использования других методов. Вы можете произносить традиционную молитву:
\begin{verse}
\small
ПАЛДЭН ЦАВЭЙ ЛАМА РИНПОЧЕ\\
ДАГИ ЧИВОР ПЕМЕЙ ДЭН ЩУГ ЛА\\
КАДРИН ЧЕНПО ГОНЭЙ ДЖЕ СУНГ ТЭ\\
КУ СУНГ ТУК КЬИ НГОДРУБ ЦОЛ ДУ СОЛ
\end{verse}
\normalsize
\begin{verse}
«Драгоценный и великолепный коренной Учитель,\\
Снизойди на лотосовый трон поверх моей головы. \\
Прими меня в силу своей великой доброты,\\
И даруй сиддхи Тела, Речи и Ума».
\end{verse}

\newpage
После этого вы можете повторить 3, 21 или 108 раз «А ЛАМА КХЬЕН НО»,
что означает «Гуру, думай обо мне!» В конце этой рецитации представляйте
получение четырёх посвящений, умоляя Гуру следующим образом:
\begin{verse}[10cm]
\small
ДУСУМ САНГЬЕ ТАМЧЕЙ КЬИ НГОВО КУЩИ ДАГНЬИ \\
ПАЛДЭН ЛАМА ДАМПА ЛА \\
СОЛВА ДЭБСО.\\
ЗАБЛАМ ГЬИ ТОКПА КХЬЕПАР ЧЕН ГЬЮ ЛА КЬЕВАР \\
ДЖИН ГЬИ ЛАБТУ СОЛ.\\
КАДАГ НЭЛУК КЬИ ТОКПА ДЖИН ГЬИ ЛАБТУ СОЛ.\\
ЦЕ ДИ НЬИ ЛА О СЭЛ \\
ДЗОГПА ЧЕНПО ЛАМ ЧОК ТАРЧИН ПАР \\
ДЖИН ГЬИ ЛАБ ТУ СОЛ.
\end{verse}
\normalsize
\begin{verse}[10cm]
«Сущность всех Будд трёх времён, владыка четырёх Кай, \\ \indent великолепный святой Гуру, я молюсь Тебе.\\
Даруй своё благословение, чтобы в моём бытие родилась \\ \indent особая реализация глубокого пути.\\
Благослови, чтобы я постиг естественное \\ \indent состояние изначальной чистоты.\\
Благослови, чтобы в этой самой жизни я завершил \\ \indent высший путь светоносного Великого Завершения».
\end{verse}
Читая эти строки, представляйте, как из белого слога ОМ во лбу Гуру исходит белый свет и входит в чакру вашего лба. Посредством этого вы получаете посвящение Сосуда. Из красного слога Ах в горле Гуру исходит красный свет и входит в вашу горловую чакру. Этим самым вы получаете Тайное посвящение. Из синего слога ХУМ в сердечной чакре Гуру исходит синий свет и входит в вашу сердечную чакру. Силой этого вы получаете посвящение Мудрости Праджни. Из оранжевого слога ХО в пупочной чакре Гуру исходит оранжевый свет и входит в вашу пупочную чакру. Посредством этого вы получаете посвящение Драгоценного Слова.
\\ \\ После того, как вы сильно и искренно помолились, Гуру растворяется в свет с великой добротой и любовью, а свет растворяется в вас. Просветлённый Ум коренного Гуру сливается воедино с вашим умом. Оставайтесь какое-то время в осознавании природы ума.
\\ \\ Сказано, что сессия медитации, начинающаяся с визуализации коренного Гуру, получения от него четырёх посвящений и объединения с его умом, намного превосходит сессию без гуру-йоги. Каждая сессия должна начинаться с гуру-йоги.
\subsection{Во время сессии}
\\ \\ Поддерживайте присутствие осознавания от прибежища до конца садханы, воспринимая всех существ и весь мир в безграничной чистоте. Если вы будете практиковать таким образом, вы быстро реализуете эту садхану. Название данной практики переводится как «Владение Печатью Бинду». Это бинду — единая сфера Дхармакаи, включающая все явления самсары и нирваны. Поскольку всё заключено в этой сфере, то её печать стоит на всём.
\\ \\ Поддерживайте осознавание Дхармакаи в течение всей практики. Если вы окажетесь в западне надежд и опасений, вы не реализуете цель этой садханы.
\\ \\ Во время основной части читайте мантру и тренируйтесь в практике Дзокчена. Делайте садхану во взгляде Трекчо. Рецитируйте мантры, пребывая в раскрепощённом состоянии этого взгляда. Не искажайте визуализацию, цепляясь за детали. Не сводите рецитацию мантр к насчитыванию определённого количества. Не вовлекайтесь в предвкушения особых медитативных переживаний. Осуществляйте садхану в состоянии изначальной чистоты, лишённом надежд и опасений.
\\ \\ Читайте мантры правильно и медленно. Чистая рецитация увеличивает вашу заслугу в сто раз. Чтение мантр в состоянии самад-хи увеличивает заслугу в сто тысяч раз. Правильно произнесённая мантра соответствует по заслуге ста мантрам, произнесённым слишком быстро, нечётко или неполностью. Заслуга от одной мантры, произнесённой в состоянии осознавания, равна заслуге от ста тысяч правильно произнесённых мантр без осознавания. Поэтому, делая ритрит, не будьте одержимыми количеством.
\subsection{Посвящение заслуг}
\\ \\ В конце сессии посвящайте заслугу, молясь следующим образом: «Пусть вся моя заслуга, накопленная в прошлом, настоящем и будущем, а также недвойственная заслуга всех Будд и Бодхисаттв и двойственная заслуга всех людей достанется всем живым существам; пусть все они освободятся от причин и следствий страдания и достигнут уровня совершенного просветления». Это относительное посвящение должно быть опечатано посвящением трёхкратной чистоты посредством пребывания в пробуждённости осознавания.
\subsection{Проверка своей сессии}
\\ \\ В конце каждой сессии производите самооценку. Спросите себя: «Отвлекался ли я большинство времени? Провёл ли я сессию, грезя наяву? Практиковал ли я с преданностью и состраданием или просто механически насчитывал мантры?» Если ваша сессия прошла хорошо, не гордитесь собой, а подумайте так: «Своему прогрессу в практике я обязан только доброте и благословению моего Учителя». Если сессия была неудачна, скажите себе: «В силу своего безначального блуждания в самсаре я провела эту сессию под влиянием неведения и омрачений. В следующей сессии я буду усердно применять наставления коренного Гуру и приложу все силы, чтобы преодолеть закоренелые привычки». Таким образом, относитесь к своей практике самокритично, не гордясь собой и зная свои недостатки.
\\ \\ Практикующие наивно избегают самокритики, думая: «Моя практика и вправду хороша. Я не отвлекалась в течение всей сессии. Так как я действительно хорошо практикую, мне пора начинать учить других». Такие практикующие не знают разницы между медитативными переживаниями и реализацией. Если вы начали гордиться своей практикой, то вы в ней не продвинетесь. Если вы думаете, что пришло время учить других, осознайте эту мысль как влияние демонов. Просветлённым существам нет нужды заранее программировать свои действия. Они проявляют непостижимую пользу другим спонтанно. Помните пример скрытого йоги, который превосходит других в практике, не раскрывая своих качеств. Настоящий йоги утаивает свои качества также, как другие люди прячут сокровища. Хвалите достоинства других, а свои собственные скрывайте!
\subsection{Три идеальных аспекта}
\\ \\ Какую бы практику вы ни делали, всегда украшайте её тремя идеальными, или совершенными аспектами: идеальным началом Бодхичитты, идеальной основной частью неконцептуальной практики и идеальным завершением посвящения заслуги. Комбинируя практику с этими тремя аспектами, вы собираете два накопления. Посредством Бодхичитты и посвящения вы собираете накопление заслуги, а посредством неконцептуальной практики вы собираете накопление мудрости.
\\ \\ Одно лишь использование этих трёх идеальных аспектов очищает двойное омрачение основы, совершенствует двойное накопление пути и способствует достижению двух Кай плода. Двойное омрачение — это омрачение эмоций и омрачение восприятия. Две Каи — это Дхармакая и Рупакая. Если эти три идеальных аспекта отсутствуют в вашей практике Дзокчена, такая практика называется «хватанием пустоты голыми руками». Это также тщетно, как попытки ухватить небо.
\subsection{Вдохновляющие напоминания}
Когда вы находитесь в равностности, там не о чем думать. И всё же, во время сессии вам может прийти в голову: «Надо выпить чаю. Пора сходить в туалет...». В этом случае расслабьтесь и подумайте: «После блуждания в самсаре с безначальных времён в отсутствии драгоценных учений я наконец получил это особое человеческое тело, которое так тяжело обрести. Я использую эту возможность для истинной практики. Клянусь вернуться к практике, не тратя ни минуты». Воодушевите себя таким образом и продолжайте свою сессию.
\newpage
\\ \\ В следующий раз подумайте о непостоянстве. Все материальные вещи меняются с каждой секундой. Смерть — самое ужасающее проявление непостоянства. Подумайте: «Если я сегодня умру, готов ли я к этому? Как велика моя уверенность в практике?». Если вы не уверены, что осознаете пробуждённость в момент смерти, думайте так: «Если я сегодня уйду, то я ведь вовсе не готова, в моей практике нет стабильности. Пусть благословение моего добрейшего коренного Гуру и Трёх Драгоценностей помогут мне в бардо».
\\ \\ В другой раз поразмыслите о тщетности самсары. Когда вы действительно постигнете это, у вас не останется к ней доверия и больше не появится самсарных предвкушений. До сих пор вы постоянно уповали на самсарное счастье. Но такое счастье содержит зачатки страдания. Мысли типа: «Я чувствую себя так хорошо! Мне здесь нравится! Я люблю свою работу! Мой ритрит идёт здорово! У меня так много денег!» ведут лишь к дальнейшему страданию. Все эти обстоятельства изменчивы. Чем больше вы привязаны к благополучным обстоятельствам, тем больше вы будете страдать. Поймите, что в самсаре нет вечного счастья. Увеличивающаяся стабильность в осознавании взгляда Дзокчена способствует уменьшению надежд на преходящие самсарные радости. Пробуждённость осознавания не тронута ни страданием, ни счастьем, она за пределами всяких самсарных условностей. Вы окажетесь вне страданий только тогда, когда достигнете стабильности в пробуждённости осознавания.
\\ \\ До тех пор, пока в уме имеются даже тонкие мысли, вы будете испытывать страдание, так как самая тончайшая фиксация на субъекте и объекте, на наблюдателе и наблюдаемом, всё же содержит самсарные зачатки. Истинное отречение — это понимание того, что цепляющиеся мысли производят самсару. Если вы действительно осознаёте это, то вы станете абсолютно независимыми и не будете нуждаться ни в чём. Вы будете как старик, наблюдающий детские игры. Вам будет ясно видна бессмысленность всей самсары.
\newpage
\\ \\ Самсарные обыватели проводят всю жизнь, чередуя пристрастия к приятным вещам с отвержением неприятных, присваивая и избегая что-то в надеждах и опасениях. Йоги ясно видит глупость и ребячество такого отношения. Он или она воспринимает самсару как океан страданий и знает из него выход.
\begin{verse}
«Взгляни по настоящему в настоящее.\\
Увидев настоящее, ты полностью освободишься».
\end{verse}
\\ \\ «Настоящее» — это пробуждённость осознавания. «Взглянуть по настоящему в настоящее» значит осознать пробуждённость ригпы, свободную от наблюдателя и наблюдаемого. Это осознавание лишено двойственности. Созерцатель и созерцаемое становятся единым целым. Сказано, что двойственный ум не может осознать свою природу. Так как ригпа недосягаема двойственному уму, она может быть осознана только сама собой. В момент осознавания природы ума все мысли освобождаются самопроизвольно.
\\ \\ Скажите себе: «Будда действительно показал абсолютную истину. Он на самом деле Просветлённое Существо. В силу великого сострадания он поделился с нами своим постижением. Благодаря доброте моего коренного Гуру я понял эту драгоценную истину. Какое невероятное благословение!» Сострадание появляется из понимания того, что хотя все существа наделены совершенной природой Будды, они мучаются от невообразимых страданий из-за её неведения. Когда осознана пустотность, вам открывается бессмысленность самсары, и у вас естественно появляется отречение, преданность и сострадание. Постижение того, что страдание и его причины происходят из двойственных цепляний концептуального ума ведёт к состраданию. Подумайте так: «Пусть все существа освободятся от страдания и его причин. Пусть я приведу их к состоянию полного просветления». Обет Бодхичитты — это наивысшая из мыслей.
\newpage
\begin{verse}
«Воззрение пустоты, лишённое сострадания,\\
Не соприкасается с высшим Путём».
\end{verse}
\\ \\ Понимание непостоянства, отречение, преданность, любовь и сострадание — это качества осознавания. Они показывают, что вы на верном пути.
\\ \\ Подумайте также о кармическом законе причинно-следствен\-ной связи. Если вы способны смотреть на несколько шагов вперёд, то будете иметь прогрессивный взгляд на жизнь. Если же вы мыслите негативно, то и ваши жизненные воззрения будут такими же. Имейте в виду, что все ваши мысли, слова и действия имеют последствия. Негативные мысли весьма влиятельны. До тех пор, пока длится ваша злость, весь мир будет казаться злым и враждебным. Смерть в состоянии сильной ненависти безусловно приведёт к очень плохому перерождению. Только осознавание пустотности растворяет мысли трёх эмоциональных ядов. Если вы что-то любите, то это страсть; если вам что-то не нравится, — это ненависть; отсутствие всякого мнения или оценки — это тупость. Все положительные, отрицательные и нейтральные мысли загрязнены тремя эмоциональными ядами и несут в себе самсарные зачатки. До тех пор, пока у вас имеются мысли, вы неразлучны с кармическими причинами и следствиями. Ваша карма перестанет накапливаться только тогда, когда вы достигнете стабильности в пробуждённости осознавания. Изначально свободное осознавание находится за пределами кармы. Но до тех пор, пока вы от него отвлекаетесь, вы продолжаете создавать свою карму.
\\ \\ Иногда практикуйте аналитическую медитацию и лично убедитесь в пустотности явлений и своего «Я». Удостоверьтесь в том, что они только кажутся существующими, а на самом деле не существуют. Поразмыслите над появлением, пребыванием и исчезновением вещей и ума.
\\ \\ Если вы не осознали пустотность и тщетность самсары, вам будет трудно развить истинное усердие «без усилий», требуемое в Дзокчене. Это практика неотвлекаемой немедитации. Подумайте следующим образом: «Я получил самые глубокие и тайные наставления всех Будд. Как замечательно! Как восхитительно! Я отплачу невообразимую доброту моего коренного Гуру, истинно практикуя эти наставления. Это единственный шанс из ста тысяч жизней. Пусть моя практика будет идти в радости. Пусть я не потеряю ни минуты!» Поймите, что очень мало кто может истинно постичь и практиковать глубочайший тайный раздел самого сущностного Дзокчена. Истинное понимание подразумевает прекращение мирских занятий ради практики этих Учений. Обычные люди слишком одержимы мирской жизнью, чтобы смочь по настоящему оце\-нить их.
\\ \\ Дзокчен означает «Великое Завершение».
      Все явления самса\-ры и нирваны завершены (ДЗОГ) в едином состоянии величия (ЧЕН).
      Это состояние — настоящий коренной учитель, в котором мы принимаем прибежище,
      — Будда Дхармакая. Эта пробужденность осознавания не является чем-то внешним,
      будучи сущностью вашего ума. Когда вы осознаёте свою собственную пробуждённость,
      просветлённый Ум коренного Гуру уже слит с вашим умом воедино.
      Если вы видите в своём коренном учителе Дхармакаю Будды,
      то он и есть природа вашего ума.
\\ \\ Сущность ума, осознающая взгляд пустотности, является высшей формой знания.
      Её также называют праджняпарамитой, — «запредельным знанием».
      Это знание неподдельно и не смешано с мыслями. Когда вы истинно осознаете
      настоящий взгляд, у вас появится великое знание, неведомое до этого.
      Эта мудрость именуется «знанием, возникающим из медитации».
      Когда естественное состояние постигнуто «так как есть»,
      все омрачения очищаются и самопроизвольно раскрывается ведение всего,
      что есть. Вы узнаёте ранее неизвестные вещи без всякого обучения
      и постороннего объяснения.
\addtocontents{toc}{\protect\newpage}
\newpage
\subsection{Белая торма и торма для препятствующих сил}
\\ \\ Чтобы начать затворничество, вы должны «договориться» с нелюдью, обитающей в данной местности. Поднесите белую торма (КАР ТОР) местным духам, спрашивая их разрешение на использование этого места для практики Дхармы. Вначале поднесите СЕРЧЕМ (кубок с чаем или алкоголем), а затем белую торма (конусообразное мучное изделие), подбросив их в направлении неба снаружи двери.
\\ \\ Поднесите торма (ГЕК ТОР) препятствующим силам, — (восьми) классам негативности, которые препятствуют практике Дхармы и враждебно относятся к практикующим. Визуализируя себя в облике Маха Шри Херука (ПАЛ ЧЕН ХЕРУКА), прикажите им покинуть данное место. Вынесите наружу (красную) торма и швырните её на землю.
\subsection{Шесты для Великих Королей}
\\ \\ Когда вы уходите в затворничество, вам следует установить «шесты для королей» снаружи вашего ритритного дома, рядом с дверью. На шестах, (которых обычно бывает четыре), прикрепляется изображение Ваджрапани с Четырьмя Велики\-ми Охраняющими Королями, или просто пылающий (синий) слог ХУМ на лотосе, солнце и луне. Это изображение также может быть одной картинкой Четырёх Охраняющих Королей. Если у вас нет этих изображений, вы можете насыпать (рядом с дверью) кучу камней и воткнуть в них несколько (сухих) веток, — это самая простая форма шестов для Королей.
\\ \\ Изображение должно быть повёрнуто наружу, а не в сторону ритритного дома. В строгом ритрите вы не можете выходить за черту, помеченную шестами. Если шест водружён прямо за дверью, не выходите за порог в течение всего ритрита. Если вы хотите иметь небольшую территорию для передвижений, то вполне нормально прикрепить изображение к ближайшему дереву или к дворовой калитке. Четыре Охраняющих Короля поклялись Будде Шакьямуни защищать его Учение и последователей. Считается, что эти Четыре Короля постоянно обитают в четырёх сторонах горы Сумеру. Когда вы водружаете для них шесты и приглашаете их на это место, они будут охранять вас от негативных сил и демонов во время вашего затворничества.
\\ \\ Внешний ритрит означает проведение границы, которая определяется шестами для Королей. Внутренний ритрит значит, что вы не покидаете медитативную позу в течение сессий. Вы сидите на своей кровати и не разговариваете. Тайный ритрит подразумевает, что вы оставляете свой ум в осознавании его природы.
\\ \\ Говоря в общем, ритритную границу может пересекать только определённый «ритритный помощник», приходящий в установленное время и приносящий необходимое продовольствие, и никто другой. Перед началом ритрита выберите себе одного человека, который будет ритритным помощником. Если какому-то другому человеку крайне необходимо посетить вас в ритрите, то ещё до начала затворничества напишите его имя на камне, именуемом «ритритным камушком», или просто на листе бумаги, и положите его под своим алтарём. Во время вступительной церемонии думайте, что этот человек находится вместе с вами в ритрите. Если вы поступите таким образом, то его или её посещение не создаст для вас препятствий. Если же вы встретитесь с людьми, которых не ждали, то у вас может появиться неприятное чувство, болезнь или проблемы в практике. На это есть две причины. Во-первых, нарушая свой ритритный обет не встречаться с людьми, вы обижаете Защитников. Во-вторых, вторгающиеся в ваш ритрит несут негативную энергию, которая может вам повредить. Лучше всего прекратить всяческое общение с внешним миром, — ни писем, ни телефонных разговоров, ни посетителей (кроме ритритного помощника). Не шпионьте за тем, чем занимаются другие люди. Говорится, что подсматривание за другими пробивает дыры в вашем ритрите, а вторгающиеся люди его нарушают.
\\ \\ Когда вы делаете «строгий ритрит» (ЦАМ ДАМ ПО), только ваш коренной Гуру или очень близкие братья/сёстры по Дхарме, с кем нет ни малейшей погрешности в самаях, могут быть желанными гостями. Когда они придут, вы будете воодушевле\-ны и рады. Все другие пришельцы считаются вторгающимися, и несут негативную энергию. Сила этой негативной энергии варьируется от человека к человеку, причём самой негативной энергией обладают нарушители самай. Никогда не допускайте, чтобы такой человек попал в ваш ритрит, так как встретив его/её, все достижения и благословения божества, мантры и самадхи исчезнут мгновенно. Также остерегайтесь людей с плохим характером, которые вторгаются, чтобы распустить порочные сплетни и слухи. Они лишат вас комфорта и снизят вашу энергию. Йогины в ритрите очень восприимчивы и легко уязвимы.
\\ \\ Вторгающиеся пришельцы могут навредить только тем йогинам, чья практика не достигла уровня, именуемого «определённостью в естественном состоянии». Когда йоги определился в естественном состоянии, встречи с вторгающимися ничуть не повредят практике. Исключение из этого — нарушители самай, они могут уязвить даже великого мастера. Из предосторожности, подымите на всех посетителей горящим ладаном (тиб. ГУГУЛ), побрызгайте на них очищающей водой из «сосуда активности» и дайте им выпить оттуда немного воды. Делайте это с каждым, кто приходит к вам в ритрит, за исключением коренного Гуру и ритритного помощника.
\\ \\ Строгое затворничество устраняет внешние, внутренние и секретные препятствия. Внешние препятствия — это проблемы четырёх элементарных энергий природы, то есть оползни, наводнения, пожары и ураганы. Внутренние препятствия — это проблемы с (энергетическими) каналами, пранами и бинду (в теле), проще говоря болезни. Секретные препятствия — это препятствия двойственного цепляния, фиксации на созерцателе и созерцаемом, субъекте и объекте, что и есть корень самсары. До тех пор, пока в вашей практике сохраняет\-ся двойственное цепляние, вы не имеете истинного взгляда. Поэтому секретные препятствия называют «препятствиями в самадхи». Когда практика подвержена влиянию двойственных цепляний, йоги будет иметь проблемы с сонливостью и возбуждением. После того, как йоги определился в естественном состоянии, присутствие его осознавания будет неуязвимо для сонливости и возбуждения.
\\ \\ Начинайте ритрит ближе к вечеру. Традиционно, шесты для Королей водружаются
      после поднесения торма препятствующим силам, но до возведения защитного круга.
      В нашей садхане поднесение торма препятствующим силам объедине\-но в одной
      строфе с защитным кругом, поэтому устанавливайте шесты для Королей после
      возведения защитного круга. На этом этапе литургии вынесите стол к месту,
      где вы будете водружать шесты для Королей, и поставьте на стол сосуд активности,
      колокольчик и дамару. Расставьте на нём внешние подношения (то есть 7 чашек с
      водой и прочим) и поместите торма для Королей прямо перед шестами. Либо сидя,
      либо стоя у стола, прочтите раздел литургии, называемый «Кратким (ритуалом)
      Шестов Королей» (ГЬЯЛ ТХО ДУ ПА). Посредством этого четыре Короля займут
      места в четырёх направлениях вокруг вашего ритрита и будут охранять вас
      в течение всего затворничества.
\\ \\ По завершению ритуала опустошите чашки с подношениями, оставьте (четыре) торма для четырёх Королей у подножья шестов (или прикрепите к шестам) и занесите внутрь дома свои ритуальные принадлежности.
\newpage
\subsection{Клятва в затворе}
\\ \\ В начале ритрита вы должны принять строгий обет в отноше\-нии продолжительности
      своего завтворничества. Обет означа\-ет такую клятву: «Я не покину свой ритрит,
      даже если горы посыпятся мне на голову, даже если меня затопит наводнением,
      даже если огонь будет бушевать со всех сторон». Вы принимае\-те такое твёрдое
      обещание для того, чтобы увеличить усердие. Если вы его не примите,
      то можете облениться. Принятие клятвы — это средство самоконтроля.
      Если вы хотите сделать долгий ритрит, обязуйте себя вначале на шесть месяцев.
      Когда полгода прошли успешно, примите обет на следующие шесть месяцев.
      Таким образом вы сможете сделать трёхлетний или даже более долгий ритрит.
      Если вы примите слишком большое обещание в самом начале, у вас могут появиться
      препятствия, задерживающие продвижение в практике, и вам не удастся пробыть
      в ритрите запланированный срок. Нарушение вашего обета обернётся для вас
      пагубными последствиями. Поэтому не преувеличивайте свои возможности и обещания.
      Много людей объявляли о трёхлетнем или четырёхлетнем ритрите, однако даже
      не смогли начать. Увеличивая время своего затворничества в небольших
      нарастающих пропорциях, вы успешно выполните своё обещание и будете
      удовлетворены исполнением своего обязательства. Это гораздо лучше,
      чем создавать себе проблемы в связи с нарушением большого обещания.
      Если вы досрочно прекратите свой ритрит, чтобы кому-то помочь,
      вы конечно сможете утешить себя тем, что действуете с благородной мотивацией.
      Но обет то ваш нарушен. Вы можете обоснованно оставить затворничество,
      только когда умирает ваш учитель или родители и зовут вас.
      Кроме коренного Гуру, только наши родители проявили великую доброту.
      За исключением этих случаев и серьёзной болезни, нет никаких оправданий
      для прекращения ритрита. Если вы сильно заболели, вы можете посетить
      врача (если нет возможности пригласить его на дом и обойтись без стациона\-ра).
      Не поддерживая своё тело в здоровом состоянии, вы просто умрёте и лишитесь возможности практиковать. Если болезнь явно угрожает жизни, вы вправе покинуть затворниче\-ство.
\\ \\ Опытный йоги или йогини спит не больше трёх-четырёх часов. Для начинающих нормально спать шесть часов в начальной стадии ритрита. С постепенным улучшением практики потребность во сне становится меньше.
\\ \\ Лучше всего, если бы вы смогли полностью прекратить говорить. Если вам очень неудобно избегать разговоров в дневное время, в частности во время сессий, то можете общаться с ритритным помощником во время вечернего перерыва. Не разговаривайте ни с кем кроме него/неё.
\\ \\ Оставляя свой ум в осознавании, вы отсекаете все усложнения, где бы вы ни жили. Осознавание природы ума — это главный пункт.
\\ \\ Что касается отречения, то если ваш желудок не так предрасположен к отречению как ваши глаза, то йогический образ жизни вам надоест, и рано или поздно вы вернётесь в мирскую жизнь. Многие практикующие в начале обладают сильным отречением. Через несколько лет они теряют интерес к практике и вместо этого становятся хорошими бизнесменами. Что может быть хуже для практикующего?! Поэтому лучше оставить мирские занятия медленно, но верно. Устройтесь в небольшом, но удобном ритритном домике и занимайтесь своей практикой без аскетизма. Сперва соберите достаточно денег. Когда вам хватает на скромную и умеренную жизнь йоги, прекратите мирские дела.
\\ \\ Я вам дам один важный совет: «не сражайтесь с врагами и не пекитесь о близких».
\\ \\ Никогда не вовлекайтесь в злобные отношения с другими. Не тратьте время на судебную тяжбу. С другой стороны, не углубляйтесь чересчур в связи с родными и друзьями. Не сближайтесь с людьми до такой степени, что это начнёт вас беспокоить. Золотая середина — не быть чересчур близким и слишком недружелюбным. Просто соблюдайте нейтралитет. Не давайте людям повода вас сильно любить или сильно недолюбливать, — придерживайтесь именно такой позиции. Помимо этого, не стремитесь стать самым удачным бизнесменом, прекратите занятия бизнесом как можно скорее. Нет сомнений, что близкие, враги и бизнес отвлекают практикую\-щих больше всего.
\\ \\ Развивайте сильное ощущение непостоянства.
      Чёткое понимание непостоянства автоматически приводит к отречению,
      и вам будет легко освободиться от восьми мирских забот.
      Такое отношение является сущностью затворничества.
      Помните, что время смерти полностью непредсказуемо.
      Хотя это легко понять, мы всё же дурачим себя, думая: «Конечно я умру,
      но не сейчас же, — может через несколько лет».
      Вы должны помнить о непостоянстве всегда.
      С этой мыслью вы сможете стать усердными практикующими и прожить так всю жизнь.
      В данный момент мы ещё можем решать, что хотим делать и куда хотим идти.
      Когда же наша жизнь оборвётся, нами будут управлять кармические силы.
      На том этапе мы не сможем контролировать ход событий,
      если только не были хорошими практикующими.
      Поэтому очень важно использовать всё оставшееся время для практики,
      начиная именно с этого момента. Готовьте себя к смерти.
\\ \\ Никогда не оставляйте практику. Не выдумывайте, что вы ей займётесь, когда станете старше. Вы можете подумать: «Сейчас неподходящее время для затворничества. Вначале я должна посетить то да это. Я ещё не отработала пенсионный стаж. Слишком рано уходить в ритрит. Только что мой близкий родственник заболел. Я начну, когда он/она выздоровеет...». Мара, который создаёт отвлечения и отсрочки для практики, именуется «Изощрённым Отпрыском Богов». Этот Мара удерживает вас от практики. Можно найти сотни оправданий, чтобы отложить ритрит. Не тратьте время! Начинайте прямо сейчас. Имейте в виду, что вы не сможете отложить свою смерть. Миг за мигом она подходит всё ближе. Внезапно дыхание покинет вас и больше не вернётся.
\subsection{Еда и питье в затворе}
\\ \\ Никакой особой диеты для ритрита не требуется. Просто питайтесь тем, к чему привыкли. Однако сказано, что чеснок оказывает отрицательное воздействие на рецитацию мантр. Поэтому старайтесь не есть чеснок. Вдобавок к этому полезно кое-что знать о потреблении мяса. Согласно традиции Сутры, перед потреблением мяса вы должны прочесть определённые мантры-дхарани, видья-мантры и подуть на мясо. В тантрической традиции вы должны иметь мотивацию, что посредством поедания плоти живых существ вы создаёте с ними связь. Создавая такую связь, вы даёте им возможность направиться к пути освобождения. Говоря в общем, все существа были нашими родителями. Потребление мяса подобно поеданию плоти вашего отца или матери. Потребляя мясо, вы должны думать: «Я подношу мясо этого существа, не менее близкого мне, чем отец и мать, естественной мандале божеств моего тела. Пусть очистятся все грехи и омрачения этого существа». Ешьте в манере (внутреннего) огненного подношения. Не поедайте мясо беспечно, а следуйте вышеупомянутым наставлениям. Таким образом во время еды вы будете собирать накопления (заслуги и мудрости). Конечно же, хорошо быть вегетарианцем, но если вы не можете обойтись без мяса, делайте эту практику во время приёмов пищи.
\\ \\ Когда вы пьёте алкоголь, представляйте, что пьёте амриту, — субстанцию подношения. Но никогда не напивайтесь. Если вы опьянеете, то станете безмозглыми и не сможете различить между правильным и неправильным. Перед употреблением алкоголя насыпьте в него немного мендруба (святого лекар\-ства), и это автоматически превратит его в амриту. Когда пьёте, представляйте, что амрита подносится естественной мандале мирных и гневных божеств внутри вашего тела. Мендруб следует добавлять только в алкоголь, а не в еду, за исключением ганапуджи.
\\ \\ Великие Мастера прошлого отрекались от всего своего имущества и жили в нищете. У нас нет особой необходимости практиковать аскетизм, вполне нормально делать практику в небольшом уютном доме. В традиции Ваджраяны чувственные удовольствия используются на пути. Йоги может наслаждаться чувственными удовольствиями, ибо он знает как поддерживать осознавание. Хорошая одежда и вкусная еда — это подношение естественной мандале тела. Следуйте наставлению Ладаг Пема Чогьял: «Тогда как воззрение и реализация зависят от ума, телу делаются подношения». Это значит, что надо хорошо практиковать и хорошо питаться, так как еда — подношение мандале божеств внутри тела. Если вы не будете есть как следует, то заболеете. Если вы заболеете, то можете умереть и потерять возможность практиковать в этой жизни. Поэтому содержите своё тело в порядке.
\subsection{Место для затвора}
\\ \\ Место для строительства ритритного дома должно быть удалённым.
      Оно должно находиться вдали от деревень, чтобы вас не посещали ни люди, ни собаки.
      Оно должно быть безопасным, недосягаемым для хищных зверей, а вода и топливо должны
      быть легко доступными.
\\ \\ Слово «удалённое» также относится к трёхкратному уедине\-нию тела, речи и ума. Уединение тела подразумевает, что тело оставлено в покое от любых мирских или религиозных занятий. Уединение речи значит полное прекращение как мирских, так и религиозных бесед, включая чтение литургий и рецитацию мантр. Уединение ума означает пребывание в осознавании его природы без мирских и духовных мыслей. Посредством такой практики вы быстро реализуете ваджрное Тело, Речь и Ум.
\\ \\ Очень важно, чтобы ритритный дом находился в красивом и привлекательном месте. Это может быть лес, снежные горы, холмистая местность или святое место, где практиковали мастера пошлого. Луше всего святое место, где жили великие мастера или даже сам Будда. В силу их великой реализации эти места пронизаны их благословением и насыщены добродетелью. В наши дни йогины могут получить благословение, просто практикуя в таком месте.
\\ \\ Наилучшее место затворничества для йоги или йогини абсолютной простоты — это либо горная пещера, либо впадина под нависшей скалой, либо яма в земле. Такие примитивные объекты недвижимости не требуют ухода или ремонта, и поэтому не создают привязанностей. Люди с меньшими способностями могут построить небольшой простой дом для затворничества.
\\ \\ Ритритный дом должен располагаться на южном склоне горы. Ему не должны угрожать оползни ни сверху, ни снизу. Сам дом должен быть повёрнут к югу. Не стройте его на самой вершине горы, он должен располагаться на горном склоне, как будто на коленях медитирующего человека, чьи руки запечатлены по обе стороны невысокими грядами гор. Место должно освещаться солнцем весь день, причём лучше всего такое место, где солнце выходит рано и заходит поздно, — это считается самым благоприятным и удлиняет жизнь практикующего в затворничестве. Ритритный дом должен иметь панораму с открытым небом, не заслоняемым ни деревьями, ни горами. Высокое место над открытым океаном идеально для практики «тройного пространства» (НАМ КХА СУМ РУК), так как при взгляде вверх открывается синее пространство неба, а при взгляде вниз — синий простор океана. Лучше всего, если вы сможете жить в одиночку в горах, делая практику простоты. Если это невозможно, то всё равно не приглашайте в свой ритрит более одного или максимум двух хороших друзей по Дхарме.
\\ \\ Выясните, насколько благоприятно выбранное вами место для ритрита. Узнайте, не случалось ли там сражений, голода, эпидемий или других бедствий. Такое место считается порочным и неблагоприятным. Никогда не стройте свой дом в месте со злополучным прошлым. Такие неблагоприятные места хороши только для практикующих Чод, так как они практикуют наставления по «трансформации плохих знамений в благополучие». Подберите себе новое и нетронутое место, незагрязнённый участок земли, на котором никто не жил до этого.
\subsection{Дом для затвора}
\\ \\ Никакого стандартного дизайна для ритритного дома на самом деле нет. Тем не менее, вы можете следовать параметрам ритритных хижин, которые использовали монахи самого Будды во время летних дождей. Рекомендуемая длина стен хижины подсчитывалась так: одна сажень + один локоть + одна ладонь, что в пересчёте из старинных мер даёт почти три метра в общей сложности. Самое большое окно имеет размер 100 см на 180 см и располагается на южной стене. Небольшие окна должны также находиться на восточной и западной стенах для проветривания. Дверь смотрит на запад и находится наискосок от алтаря. На северной стене окон нет. Простая кухня пристраивается к стене снаружи. Алтарь размещается в северо-восточном углу комнаты, на восточной стене. Кровать располагается вдоль северной стены, и когда вы спите, ваша голова направлена на восток, в сторону алтаря. Спать следует на правом боку в позе спящего льва, лицом на север. Ваша спина должна быть повёрнута к югу, а ступни на запад. Когда вы практикуете, вы сидите на кровати, глядя в сторону большого окна на южной стене. Ваша кровать должна иметь толстый, прочный и соразмерный матрас, лежащий на простой деревянной раме, обеспечивающей его сухость. Рама не должна иметь спинок, в частности по бокам, находясь в пятнадцати сантиметрах от пола. Ваша кровать является также вашим местом для медитации, так что традиционные медитативные коробки не пригодны для Дзокчена. Вам нужно достаточно места для некоторых практик. Перед тем, как закладывать фундамент своего ритритного дома, заройте там маленькую бутылочку с мендрубом (священным лекарством), чтобы устранить вредные воздействия.
\\ \\
\subsection{Начало затвора}
\\ \\ Перед тем, как уходить в затворничество, запаситесь основным необходимым продовольствием. Чтобы избежать болезней, запаситесь лекарствами. Чтобы защититься от воров и грабителей, вам необходимо оружие. Запаситесь достаточным количеством еды и питья, чтобы предотвратить голод и жажду. Чтобы не мёрзнуть, не забудьте взять тёплую одежду и одеяла.
\\ \\ Начинайте ритрит в любой благоприятный день во второй половине лунного месяца, то есть во время убывающей луны. Заканчивайте ритрит в любой благоприятный день первой половины лунного месяца, когда луна прибывает. Следуйте тибетскому календарю.
\\ \\ Утром первого дня сделайте Санг, — очищающее подношение благовоний. После этого прочтите несколько раз молитву «Устранение Препятствий с Пути» («Барчей Ламсэл»). Утром или в полдень подметите и вымойте полы и соберите на алтаре все необходимые принадлежности, включая подношения для пиршества ганапуджи (ЦОК).
\\ \\ Под своим местом медитации вы должны начертить мелом правостороннюю свастику. Это древний буддийский (а не фашистский символ, олицетворяющий благополучие и неизменную суть. В четырёх углах свастики начертите по «кругу радости» (похожего на символ инь-янь, только без отверстий). Прочтите три раза мантру четырёх ХУМ и подуйте на свастику. Поверх рисунка свастики положите три веточки травы куша. В связи с тем, что Будда Шакьямуни достиг просветления, сидя на траве куша, эти веточки создадут благоприятную взаимозависимость для того, чтобы и вы достигли просветле\-ния. Над своим медитативным сидением на потолке прикрепи\-те конверт с волосами, клочком одежды или ногтями своего коренного Гуру, а также с мендрубом (священным лекарством). Эти вещи несут благословение Гуру и означают, что ваш коренной учитель всегда с вами. После того, как вы подготови\-ли таким образом своё место для медитации, на нём никто не должен сидеть кроме вас. По завершению ритрита сотрите свастику и уберите траву куша.
\\ \\ Начинайте ритрит ближе к вечеру. Очистите свой ум посредством предварительных практик из Лонгчен Ньингтик или по крайней мере прочтите короткий вариант этих практик, написанный мастером Джамьянг Кхьенце Вангпо, — от прибежища до гуру-йоги. Затем прочтите молитву к мастерам линии вместе с молитвой к линии вашей садханы. После этого поднесите белую торма местным духам, визуализируя себя в форме Ваджра Маха Шри Херука (ДОРДЖЕ ПАЛ ЧЕН ХЕРУКА). Затем прочтите строфы прибежища и Бодхичитты из садханы, поднесите торма препятствующим силам и установите шесты для Королей. Продолжайте садхану, прочтите мантры и поднесите ганапуджу. Завершая свой первый день ритрита, произнесите пожелания, молитвы и посвящения заслуги. В первый день вы делаете только одну сессию, а затем отдыхаете. Ритрит фактически начинается следующим утром.
\subsection{Завершение затвора}
\\ \\ В последний день ритрита встаньте пораньше, в 01.00—02.00 ночи. Умойтесь и оденьтесь в чистую одежду. Начните с молитв к мастерам линии и вашей садханы, затем сделайте саму садхану, прочтите исповедь и прочее. После этого следует ганапуджа (ЦОК), по завершению чего вы принимаете сиддхи в первые минуты рассвета, когда становятся видны линии на ваших ладонях. Это время считается благоприятным для получения достижений, так как Будда Шакьямуни достиг просветления именно в это время суток. Приложите к своему лбу, горлу и солнечному сплетению чашу из черепа (капалу), имеющего три секции, красную гуру-торма (ЛА ТОР), статую Лонгченпы и другие освящённые принадлежности (ДРУБ ДЗЭ), которыми вы пользовались в ритритной практике. Принятие сиддхи устраняет внутреннюю границу ритрита, то есть завершает ваше затворничество на внутреннем уровне. Дочитайте литургию до конца и завершите ритрит молитвами и пожеланиями благополучия. Вынесите стол наружу, поставив его перед шестами для Королей. Установите на нём внешние подношения и (четыре) торма для Королей. Поместите на стол сосуд активности, дамару и колокольчик. Прочтите часть литургии, называющуюся «снятием шестов Королей» (вся литургия переведена в конце книги), поднесите торма четырём Королям в знак благодарности и попросите их удалиться в свою резиденцию на горе Сумеру. Занесите шесты Королей внутрь дома и уберите их с глаз долой. Эта процедура завершает внешний ритрит, так как устраняется внешняя граница.
\\ \\ Если вы делали долгий ритрит, то согласно традиции вы остаётесь внутри своего ритритного дома даже после того, как устранили внутреннюю границу. Вы отдыхаете и ждёте несколько дней, прежде чем снимать внешнюю границу. Если вас поджимает время, то конечно, вы можете убрать внешнюю границу сразу же. Когда вы убрали внешнюю границу, избегай\-те контактов с большим количеством людей. Вначале встречайтесь только с учителями и друзьями по Дхарме.
\subsection{Проверьте себя}
\\ \\ Поняли ли вы, что раз вы родились, вам предстоит и умереть? Поняли ли вы, что после смерти вам придётся переродиться вновь?
\\ \\ Поняли ли вы, что родившись вновь, вы будете испытывать страдания? А понятно ли вам, что вашим страданиям не будет конца? Пожалуйста, осмыслите это всем своим сердцем.
\\ \\ Вы должны стремиться именно в этой жизни достичь состояния совершенного просветления, запредельного рождению и смер\-ти. Если нет, то старайтесь достичь просветления в самый момент смерти. Если нет, то вы можете достичь просветления в бардо (посмертном состоянии). Пока вы живы, вы должны по крайней мере обрести уверенность, что не переродитесь в трёх низших мирах самсары.
\\ \\ Совершенному йоги нет нужды умирать. Ещё до смерти он реализует радужное тело великой трансформации (ДЖА ЛУ ПХО ЧЕН) и приносит пользу существам в своём теле. Или же он может реализовать («простое») радужное тело и манифестировать чистую землю Будды Самбхогакаи. Либо атомы его тела просто исчезают в реализации Дхармакаи. Возможно и то, что он отправится в чистые земли в своём собственном теле или просто станет невидимым, — это два случая реализации Нирманакаи. Когда умирает йоги, он освобождается от оболочки своего тела в самый момент смерти, подобно гаруде, рождающейся из яйца. Также как гаруда взмывает в небо, как только выбралась из скорлупы, так и йоги достигает просветления Дхармакаи в самый момент смерти. Он не проходит через посмертное состояние бардо. Его уход подобен дню полнолуния, когда солнце встречается с луной без сумерек между ними. Если же йоги осознает себя во время бардо Дхарматы (посмертного состояния природы явлений), то он достигнет просветления Самбхогакаи. Если он осознает себя в бардо Становления, он достигнет просветления Нирманакаи.
\\ \\ Чтобы достичь этого, существует три метода: преданность и чистое восприятие, направленные «вверх», сострадание и отречение, направленные «вниз», а между ними осознавание природы ума. Так как ваша жизнь коротка и насыщена неблагоприятными обстоятельствами, которых гораздо боль\-ше, чем благоприятных, вы должны применять эти методы непрерывно. Имейте в виду, что эти три метода находят истинное применение только тогда, когда в вашем сердце появилось настоящее понимание непостоянства. Поймите, что вы приближаетесь к смерти с каждой секундой. И остановить этот ужасающий процесс просто невозможно. Допустим, что вы живёте сто лет, и сорок уже прожиты. Эти сорок лет вы не сможете вернуть назад, даже если привлечёте все силы этого мира.
\\ \\ Спросите себя, — когда вы на самом деле занимаетесь практикой наставлений своего коренного Гуру по преданности, чисто\-му восприятию, отречению, состраданию и осознаванию приро-ды ума? Взгляните на себя со стороны. Как вы проводите день? Семь часов вы спите, три часа вы тратите на еду и питьё, два часа просто перемещаетесь и восемь часов тратите на работу и разговоры с другими. Ну а после всего этого, посвящаете ли вы практике оставшиеся четыре часа? Как вы продвинулись в своих переживаниях и реализации? Насколько уменьшаются ваши эмоции с каждым днём? Практикующий с высшими способностями продвигается в своей практике изо дня в день. Подходите ли вы к этой категории? Или же вы соответствуете практикующим со средними способностями, которые измеряют свой прогресс по месяцам? Насколько ваши цепляния за «Я» и «моё» уменьшаются с каждым месяцем? Насколько улучшается ваше осознавание пробуждённости с каждым месяцем? А может вы просто практикующий с низшими способностями, чей прогресс заметен лишь по прошествию года? Если вы сравните свою практику в этом году с прошлым годом, заметно ли какое-нибудь продвижение? Увеличилась ли ваша преданность и сострадание?
\\ \\ Если вы не продвинулись за целый год, то какой практикой вы там вообще занимаетесь? Чьему примеру вы следуете? Смотрите, не ошибитесь с примером своего подражания. Будда сказал: «Если вы хотите быть такими же как Я, следуйте за мной». Если вы хотите обрести обычные и высшие сиддхи, то в первую очередь вы должны чётко знать, как сделать Дхарму своим главным приоритетом и отвести мирским занятиям второстепенное место. Проверьте себя, когда кто-то приглаша\-ет вас в ресторан или на вечеринку во время практики. Прекратите ли вы свою практику ради одного лишь хорошего обеда или будете поддерживать усердие и завершите свою сессию? Разберитесь в своих настоящих ценностях и предпочтениях.
\\ \\ Если вы действительно хотите достичь просветления в этой самой жизни, то вы должны практиковать днём и ночью всю свою жизнь до тех пор, пока не станете просветлёнными. В данном случае речь идёт не о годах и месяцах. Достижение просветления за одну жизнь возможно только для людей с высшими способностями. Высшие способности означают великую веру, великое усердие, великую осознанность, великое самадхи и великое знание и Благодаря великому знанию вы понимаете истинный смысл учений о воззрении, медитации и поведении. Посредством такого знания вы обретаете великую преданность, ибо у вас будет повод для такой веры.
\\ \\ Это вдохновит вас на практику великого самадхи. Такая практика приведёт вас к великой осознанности, и вы никогда не забудете о воззрении, медитации и поведении. Таким образом, благодаря великому усердию вы будете непрерывно практиковать вплоть до достижения полного просветления. Ваше усердие должно быть непрерывным как поток реки, или подобным тетиве лука, — без утомления и послабления. С такой практикой вы в конце концов достигнете стабильности в осознавании и станете «йогинами непрерывного потока», которые никогда не отвлекаются от осознавания.
\\ \\ Вы должны развивать эти пять качеств до тех пор, пока не обретёте их в совершенстве. Читайте биографии великих мастеров и вы разовьёте преданность. Осознавайте природу ума и вы обретёте веру в гуру своего собственного осознавания. Внешний гуру — это ваш коренной учитель, внутренний гуру — ваш собственный ум, тайный гуру — это пробуждённость вашего осознавания. Постигнете единство этих трёх гуру. Осознайте, что вся вселенная и все существа в ней — это гуру пробуждённости осознавания.
\\ \\ Развивайте усердие, размышляя о непостоянстве. Воодушевляйте себя на усердную практику, думая о преимуществах Дхармы. С ясным и свежим умом радуйтесь усердной практике. Чтобы обрести такую ясность, находитесь в уединении. Не обращайте внимание на то, что говорят, делают или думают остальные. Смотрите только на неподдельную пробуждённость, присутствующую в этот миг.
\\ \\ Развивайте знание посредством слушания учений, размышления над ними и их практического применения. Благодаря интенсивной практике вы обретёте знание, происходящее из медитации. Вы познаете Дхарму Будды без обучения.
\\ \\ Развивайте самадхи, тренируясь в шаматхе и випашьяне. В Дзокчене абсолютной формой шаматхи считается аспект стабильности осознавания. Абсолютная форма випашьяны — это постигающий аспект осознавания.
\\ \\ Тренируйтесь в осознанности, общаясь с духовными друзьями и учителями. Присутствуйте в осознавании природы ума в любой ситуации. Практикуйте наставления «коротких промежутков, повторяемых много раз».
\\ \\ Если человек с высшими способностями встретит учение Великого Завершения, он или она быстро достигнет просветления. Этот человек оставит все мирские занятия и посвятит всю свою жизнь практике. Этот йоги проведёт жизнь, прекратив «девятикратные активности», — три активности тела, три активности речи и три активности ума.
\\ \\ Три активности тела — это 1) внешне — все омрачённые мирские занятия; 2) внутренне — второстепенные религиозные занятия типа простираний, обхождений (святых мест) и прочие; 3) секретно — даже малейшие шевеления тела. Сидите на своём месте медитации без малейших движений как труп, брошенный на кладбище.
\\ \\ Три активности речи — это 4) внешне — омрачённые мирские разговоры; 5) внутренне — рецитация мантр и чтение молитв; 6) секретно — вдохи и выдохи. Полностью прекратите разговаривать, позволив своему дыханию успокоиться в естественном состоянии, где оно едва заметно.
\\ \\ Три активности ума — это 7) внешне — заблуждающееся мирское мышление; 8) внутренне — идеи об учении и сосредоточение на визуализациях; 9) секретно — все умственные проекции и концентрации. Пребывайте в неконцептуальном состоянии, не покидая присутствия осознавания.
\\ \\ Практикуя эти наставления по прекращению девятикратных активностей, начинайте тренировку с получасовой неподвижности в каждой из четырёх сессий. Привыкнув к этому, переходите к часу в каждой сессии и так далее. Продолжительность неподвижности тела, речи и ума должна соответствовать возможностям тела и практики.
\subsection{Разочарование старых практикующих}
\\ \\ Некоторые старшие практикующие, начавшие практиковать Дхарму с чрезмерными предвкушениями, через какое-то время разочаровываются и осознают, что после двадцати лет практики у них не развилось отречение. Они оказываются слишком привязанными к миру, но без продвижения на духовном пути. Они начинают думать, что просветление за одну жизнь вообще невозможно.
\\ \\ Таким людям следует подумать о том, что они находились в заблуждённом состоянии бесчисленные жизни. В этой жизни они стремятся достичь просветления, но хотят получить его легко и быстро. Они думают только о плодах, не утруждая себя практикой пути. Просветление за одну жизнь без отречения просто невозможно. Вы должны посвящать себя практике днём и ночью. Просветление за одну жизнь подразумевает достижение уровня неотвлечения от осознавания. Если ученик получил наставления по Дзокчену и усердствует в практике днём и ночью, то он или она без сомнений достигнет просветления.
\\ \\ Люди читают истории, что практикующие с высшим усердием могут достичь реализации за три года и семь месяцев. Люди читают, что практикующие со средними способностями достигают реализации за пять лет и пять месяцев, а практикующие с меньшими способностями могут достичь просветления за семь лет и одиннадцать месяцев. Но практика Дзокчена не является запрограммированным расписанием, которое зависит от времени. Такой механический подход к практике приводит к разочарованию. Йоги Дзокчена не должен измерять свою практику сроками ритритов и количеством мантр. Однако её можно оценить по продолжительности присутствия в осознавании без отвлечений. Если и когда вы сможете пребывать днём и ночью в неотвлекаемой немедитации, то тогда вы выясните для себя, к какой из вышеупомянутых категорий вы относитесь.
\newpage
\\ \\ Чтобы достичь такого уровня практики, вы должны использовать метод «коротких промежутков, повторяемых многократно». Это единственный путь. Это именно та практика, в которой надо усердствовать. Не ищите других практик. Практикуйте без предвкушений и опасений. Практикуйте осознавание природы ума без устали. Но не утомляйте себя механической практикой Дхармы. Дзокчен невозможно реализовать в состоянии предвкушений и опасений. Это ведёт только к разочарованию и к риску полного прекращения практики Дхармы. Дзокчен преследует только осознавание.
\\ \\ Йоги, не получавший наставлений по Дзокчену, может провести много времени в интенсивных ритритах по чтению мантр, надеясь реализовать какое-нибудь божество. Если он не реализует это божество через долгие годы практики, то скорее всего он разочаруется. Йоги никогда не должен питать надежд на результаты практики и особые знаки. Дзокчена должен быть осторожным в том, чтобы не перенапрячься в практике природы ума. Он должен быть расслабленным и раскрепощённым изнутри. Если вы на самом деле расслаблен\-ны, то вам просто не от чего будет переутомиться. Практикую\-щий немедитацию без отвлечений не может устать от своей практики. Но если он будет напрягаться в практике неотвлекаемой немедитации, то конечно он утомится. Говоря вкратце, если йоги практикует с двойственным умом, он в конце концов устанет от такой практики. Если же он практикует, присутствуя в осознавании, то он будет продвигаться, не утомляясь. Суть истинного раскрепощения можно выразить словами Джигме Лингпы:
\begin{verse}[10cm]
«Когда осознавание прибывает в основное пространство, 
\\ Аналитическая медитация исчезает естественным образом; 
\\ Методичные усилия осознанности исчезают из вида. 
\\ Как легко и свободно состояние неделимого пространства и осознавания!»
\end{verse}
Главный момент в практике — это неотвлечение, независимо от того, где вы находитесь,
— в пещере или деревенском доме. Когда в вашей практике нет отвлечений,
то всё прекрасно. Если ваша практика достигнет высокого уровня неотвлечения,
то у вас отпадёт желание ходить в кино или на вечеринки. Вы будете глубоко
чувствовать бессмысленность этих развлече\-ний. Чем больше ваша стабильность
в осознавании, тем дальше вы будете от ошибочного пути. Поговорка «зрячий
не прыгнет в пропасть» относится к йогинам, которые осознанно избегают
бессмысленных занятий. Не цепляясь ни за что внутри, они осознают полную
бессмысленность внешних явле\-ний. Это настоящее отречение высших практикующих.
Без такого естественного отречения и неутомимого усердия просветление
за одну жизнь невозможно. Восемь мирских забот полностью уничтожатся
в уме такого йоги. Но его сострадание, альтруизм и забота о других
никогда не исчезнут. Даже если такой йоги не учит никого другого,
каждый увидивший, услышавший его или соприкоснувшийся с ним будет
освобож\-дён. Когда люди встречаются с настоящим йоги, присутствующим
в осознавании, то без единого слова его реализация касается их умов,
и они оказываются в неконцептуальном состоянии без усилий. Таково
благословение самадхи настояще\-го йоги. Оно мгновенно устраняет мысли
окружающих его людей. Ему ничего не нужно для этого делать.
Это происходит само собой.
\vspace{1cm}
\subsection{Преподавание \\— препятствие для Просветления}
\\ \\ Если йоги полностью избегает преподавательской активности до того, как он достиг восьмого буми реализации, ему будет легче достичь просветления в этой жизни. Говорится, что если вы хотите реализовать радужное тело, то вам не следует брать учеников вообще. Наличие учеников создаёт отвлечения, которые становятся препятствием для реализации радужного тела. Первый Джамгон Конгтрул (1812—1899) имел предсказание, что ему суждено реализовать радужное тело, если у него не будет учеников. Но поскольку он преподавал Дхарму тысячам людей, то даже он не смог достичь такой реализации. Практическое правило гласит: «Отсутствие учеников более благоприятно для достижения радужного тела, чем их наличие». Только Гуру Ринпоче и Вималамитра были исключениями из этого правила. Несмотря на бесчисленное количество учеников, их преподавательская активность не помешала достижению радужного тела великой трансформации.
\\ \\ Поэтому йогины простоты не берут учеников. Они скрываются в горах и практикуют всю свою жизнь. Мы должны следовать примеру йогинов простоты, не пытаясь имитировать Вималамитру или Гуру Ринпоче.
\subsection{Принятие подношений \\за преподавание и в затворе}
\\ \\ Независимо от того, какая национальность у учителя, ученики вначале должны попросить его дать учение. Учитель не должен напрашиваться и бегать за учениками. Никогда не преподавайте учения Дхармы людям, не желающим их слушать. Испрашивая наставления, ученики должны в любом случае поднести учителю мандалу. Мандала, — деньги или золото, должна быть поднесена по причине величия учений Дхармы. Учитель может взять себе всё, что ему поднесено. Когда вы испрашиваете ваджрного мастера (дордже лопон), духовного учителя (геше) или духовного товарища (гевей дрокпо) объяснить драгоценные учения Будды, вам следует доставить им удовольствие посредством подношений. Когда учитель ублажён подношениями, ученик уже обрёл заслугу.
\\ \\ Учения Будды бесценны. Сколько бы денег вы ни поднесли, они не окупят истинную стоимость учений. Когда вы испрашиваете глубочайшие учения Великого Завершения, на мандале дол\-жны быть поднесены все ваши деньги и золото. Вы не можете купить эти учения. Сколько бы вы ни поднесли, это лишь символический взнос. Мастера прошлого собирали золото годами, чтобы только испросить учения. Они подносили всё, что у них имелось. Деньги и золото не были для них желанными вещами. Они жаждали лишь драгоценные учения, которые позволили бы им достичь просветления за одну жизнь. Если ученики предварительно не испросят мастера преподать драгоценную Дхарму, им будет неведома ценность учений. Можете быть уверены, что настоящим Мастерам не нужны деньги учеников.
\\ \\ Материальные подношения — это самый незначительный среди путей ублажения учителя. Самое большое удовольствие учите\-лю доставит ваша практика, затем служение телом, речью и умом, а уж потом материальные подношения.
\\ \\ Если иностранный учитель уполномочен давать учения по предварительным практикам, шаматхе или випашьяне другим иностранцам, то ученики должны формально испросить учение с подношением мандала. Никогда не начинайте учить случай\-но. Настоящий ученик к тому же сделает вначале простирания. Посредством простираний и подношения мандала ученик вы\-ражает почтение учению Будды и тому, кто их объясняет. Преподаватель должен быть квалифицирован и уполномочен давать учение. Он должен использовать принятые подношения на благотвори\-тельные цели. Получая подношения, он должен всегда подносить их Трём Драгоценностям и посвя\-щать заслу\-гу на благо учеников. Если учитель преподаёт лишь для того, чтобы получить приход от Дхармы, не посвящает заслугу и не использует поднесённые деньги на благотвори\-тельные цели, то он попадёт под негативное воздействие «грязных денег».
\\ \\ Настоящие мастера используют поднесённые деньги на практические нужды своих учеников, — на постройку ритритных центров, монастырей и библиотек (с книгами по Дхарме), они подают деньги бедным практикующим, нищим и больным. Другие мастера, типа Палтрула Ринпоче, принимали подношения, а затем их выкидывали, не беря себе ничего. Если вы сделаете подношение настоящему мастеру, то заработаете большую заслугу. Не думайте, что учителю нужны ваши деньги.
\\ \\ Палтрул Ринпоче (1809—1882) скитался по пещерам, живя в стиле нищего. Он никогда не присваивал никаких подношений. Он их выбрасывал за спину вглубь пещеры, говоря: «Я подношу это моему коренному Гуру и Трём Драгоценностям!» Какие бы подношения он ни получал, — золото, деньги, парчу, шёлк и прочее, — он просто бросал это за спину. Через какое-то время пещера наполнялась подношениями, и Палтрул Ринпоче, оставшийся без свободного места, просто покидал её и уходил практиковать в другую уединённую местность. Местные люди называли эти отвергнутые подношения «Сокровищницами Старого Палтрула». Так как он был великим и известным мастером, вначале местные люди боялись трогать подношения в течение года или двух, опасаясь наказания со стороны Защитников Дхармы. Но со временем жулики прихватили немного безнаказанно, и вскорости вся пещера опустела вновь. Когда великий Джамьянг Кхьенце Вангпо (1819—1892) услышал эти истории о Палтруле Ринпоче, он сказал: «Этот безмозглый Палтрул не знает как правильно использовать деньги. Вместо того, чтобы увеличить свою заслугу, пустив те подношения на благотворительные цели, он позволяет ворам растащить всё поднесённое. Что за придурок!» Чуть позже Палтрул Ринпоче прослышал те слова Джамьянга Кхьенце Вангпо и с того момента стал использовать подношения на сооружение огромной «китайской» стены из камней, инкрустированных мантрами ОМ МАНИ ПАДМЕ ХУМ. Он нанимал тридцать или сорок семей в течение многих лет, чтобы завершить этот проект. Сам Палтрул Ринпоче не имел ни малейшей привязанности к деньгам и богатству.
\\ \\ Учитель должен всегда принимать подношения, чтобы позволить ученику или спонсору приобрести заслугу. Ученик должен посвятить заслугу, зарабатываемую в результате своего подношения. Учитель в свою очередь должен поднести получен\-ное Трём Драгоценностям и произнести посвящения и пожела\-ния. Более того, учитель должен использовать подношения на благотворительные цели. Использовав подношения таким образом, учитель должен опять посвятить заслугу от этого на благо всех существ. Действуя таким образом, вы увеличите накопление заслуги до невероятной степени.
\newpage
\\ \\ Когда вы получаете подношения от спонсоров в ритрите, сразу же подносите их Трём Драгоценностям в форме мандалы. Вы вправе использовать эти деньги на ритритные нужды. Если у вас имеется излишек денег, потратьте их на благотворительные цели, поднеся часть своему коренному Гуру или бедным людям. Йоги, получивший подношения во время ритрита, должен думать так: «Этот спонсор обеспечивает мой ритрит. Я же практикую с намерением достичь просветления в этой жизни. Пусть в силу этой благоприятной взаимозависимости между мной и этим добрым спонсором мы достигнем просветления одновременно». В одной из песен Мила-репы говорится:

\begin{verse}[9cm]
«Между усердным практиком, медитирующим в горах,\\
И тем кто обеспечивает его всем необходимым,\\
Есть взаимозависимость \\ \indent для одновременного просветления.\\
Посвящение заслуги \\ \indent является сущностью этой взаимосвязи».
\end{verse}

\subsection{Хватает ли вам учений}
\\ \\ Когда вы уверены в том, что действительно и правильно осознали обнажённую природу ума и точно знаете, что все многочисленные учения Дзокчена говорят об одном и том же, тогда вы почувствуете, что этого достаточно. Если у вас нет никаких сомнений, то зачем вам нужны другие учения? Когда кто-то испрашивает учение за учением, это лишь показывает, что он ещё не освободился от сомнений. Когда все ваши сомнения исчезнут, вы почувствуете, что получили достаточно наставлений. У вас не будут течь слюни на другие учения.
\\ \\ Вначале рекомендуется провести несколько лет со своим коренным Гуру, чтобы понять разницу между плохим и хорошим, а также научиться осознавать сущность своего ума и делать практики Дзокчена. После этого отправляйтесь в ритрит и оставайтесь в нём. Если исчезли все сомнения по поводу вашей практики, вам больше не нужно находиться рядом со своим коренным Гуру. Вы будете неотделимы от него силой вашей преданности и практики природы ума. Как сказано: «Вначале полагайтесь на своего коренного Гуру, затем полагайтесь на письменные наставления вместо учителя (но не наоборот!), а в конце концов полагайтесь на гуру пробуждённости своего осознавания».
\subsection{Уединение}
\\ \\ Если у вас появилось непреодолимое желание практиковать в, затворничестве и вы сделали всего семь шагов в сторону уединённого места, то вы обрели больше заслуги, чем от подношения всего золота в мире тысяче Буддам. Уединённое место означает горную местность, не посещаемую людьми. Это также может быть место в снежных горах, в скалах, в предгорьях, в лесистых холмах или на острове. Таковы места для настоящей практики Дхармы, в них можно достичь просветления за одну жизнь. К уединённым местам также относятся места, где практиковали и достигли просветления мастера прошлого. Места, которые способствуют достижению просветления, несут большое благословение. В удалённом месте нечем больше заниматься, кроме практики Дхармы. Там нет внешних отвлечений. Если у вас действительно появится глубокое желание сделать ритрит в таких местах, и вы пройдёте семь шагов в направлении этого места, вы обретёте невероятную заслугу. Будда Шакьямуни сказал:
\begin{verse}[10cm]
«Семь шагов в сторону уединённого места \\
Гораздо лучше подношения всего золота тысяче Буддам».\\
\end{verse}
Но из-за одной лишь заслуги вы не достигните просветления. Так или иначе, вы должны отправиться в одно из этих мест и провести в затворничестве всю оставшуюся жизнь. Если вы считаете себя практикующим и не уходите практиковать в такие уединённые места, то наверное ваши ноги так парализованы, что вы не можете добраться до туда. Или же у вас имеются большие проблемы со зрением, что вы не можете даже увидеть эти места. Либо вы не можете поверить, что такие места несут огромное благословение, и поэтому у вас нет веры в практику среди этих мест. В этом случае вы действительно сошли с ума. Как говорится:
\begin{verse}[10cm]
«Если ты не уходишь в уединённые места, \\ \indent ты наверное хромой. \\
Если твои глаза не видят уединённых гор, \\ \indent ты наверное слепой. \\
Если у тебя нет веры в уединение, \\ \indent ты просто сумасшедший».
\end{verse}
Трёхкратное уединение описано следующим образом:
\begin{verse}[10cm]
«Уединение тела обретается в удалённых горах. \\
Уединение речи обретается в прекращении разговоров. \\
Уединение ума обретается в прекращении всех мыслей».
\end{verse}
\subsection{Свободное и строгое затворничество}
\\ \\ В так называемом «свободном ритрите» вы разбиваете день на четыре сессии и строго их соблюдаете, не выходя наружу и не принимая посетителей (кроме помощников). Но во время перерывов между сессиями вы можете покинуть свой ритритный дом и выйти (недалеко), чтобы встретиться с людьми. В строгом ритрите вы не встречаетесь ни с кем (кроме помощника и учителя) и не выходите наружу ни во время сессий, ни во время перерывов.
\\ \\ Строгое затворничество гораздо более благоприятно для практики, чем свободное, в котором вы подвержены внешним отвлечениям. В строгом затворничестве вы устанавливаете шесты для Королей и никогда не покидаете ритритный дом, не встречаясь с посетителями. В свободном ритрите вы также устанавливаете шесты для Королей. Хотя вы можете покинуть ритритный дом, в него не может войти никто другой. Если вы позволяете посторонним людям входить в ваш дом, то тогда нет смысла устанавливать шесты для Королей. Если вы приглашаете людей в свой ритритный дом во время перерывов, то ваш ритрит называется «чрезвычайно свободным».
\\ \\ Говоря в общем, если вы хотите сделать строгий ритрит, всегда начинайте его в свободной манере. В течение 10—15 дней постепенно ограничивайте свою свободу. Когда вы в конце концов установите шесты для Королей, не встречайтесь ни с кем, за исключением избранного ритритного помощника. Не начинайте свой ритрит в строгой манере, постепенно расслабляясь ближе к завершению. Это считается неверным.
\subsection{Будьте дышащим трупом}
\\ \\ Никогда не прекращайте практику. Вам не нужно уходить в пещеру. Просто живите в маленьком доме или ритритной хижине. Всегда развивайте преданность, сострадание и осознавайте природу своего ума. Практикуйте вплоть до момента смерти. Посвятите оставшуюся жизнь одной лишь практике.
\\ \\ Некоторые практикующие хотят практиковать, но не имеют необходимых условий или средств, чтобы провести остаток своей жизни в ритрите. Другие имеют на это деньги, но не заинтересованы в практике вообще. Третьи хотят заняться практикой после увольнения, но не знают чувства меры. Имея тысячу долларов, им надо десять тысяч. Когда они получают десять тысяч, им уже нужно сто тысяч. А когда у них оказывается сто тысяч, они хотят миллион долларов. Не секрет, что некоторые умирают, всё ещё пытаясь собрать самые благоприятные условия для практики. 
\\ \\ Лучше всего, если вы сможете отдать практике всю оставшуюся жизнь, имея на это материальные средства. Без денег вы не сможете обеспечить себя в ритрите. Когда у вас появились средства, имейте чувство меры. Людей, не заинтересованных в практике Дхармы, также много, как звёзд в ночном небе. Люди, которые действительно практикуют Дхарму, также редки, как дневные звёзды. Вам нужна крыша над головой, одежда, еда и питьё. Если вы собрали всё это, у вас уже имеются благоприятные условия для ритрита.
\\ \\ Помните, что жизнь коротка и насыщена болезнями. Вам не дано знать, когда вы точно умрёте. Нет никакой гарантии, что в следующей жизни вы родитесь в благоприятной ситуации с такими же сопутствующими условиями для практики. Поэтому устройтесь в небольшом ритритном домике или хижине. Хотите — готовьте себе еду сами, хотите — дайте готовить ритритному помощнику, своей жене или другу. Живите скромно, чтобы другие о вас не знали. Не зарабатывайте себе известность и славу йогина. Настоящий практикующий не должен иметь ни славы, ни имени, ни известности.
\\ \\ Мастера прошлого говорили, что тело следует считать ритритным местом, а ум — практикующим в ритрите. Все мы обладаем присутствующей пробуждённостью. Если бы у нас не было присутствующей пробуждённости, мы были бы трупами. Прорвитесь к ней сквозь усложнения прошедших, настоящих и будущих мыслей. Прошедшие мысли уже ушли. Будущие мысли ещё не появились. Взгляните в сущность нынешней мысли, и она исчезнет в осознавании природы ума. Оставьте свой ум в неподдельном естестве. Если ваш ум присутствует в этом состоянии, его можно назвать «абсолютным отшельником».
\\ \\ Абсолютный ритрит включает три метода: преданность к Трём Драгоценностям, сострадание ко всем существам и рассечение мысленных комплексов (Трекчо) посредством осознавания природы ума. Это сущность буддийской практики.
\\ \\ Сущность ума или природа Будды не является объектом медитации. Она имеется у каждого, и её нужно лишь осознать. Оставьте осознавание изначального состояния в неподдельном естестве. Осознайте то, что у вас есть, и оставьте это «так как есть». Это сущность всей практики. Не утруждайте себя аскетизмом.
\\ \\ Просто купите небольшой дом, дачу или ритритную хижину не слишком близко к деревне, — в одном, двух или пяти километрах от населённого пункта. Подберите место, где нет воров и бандитов. Устройтесь в нём и делайте несложную практику днём и ночью. Прекратите все занятия. Будьте дышащим трупом. Не двигайтесь со своего места медитации. Ваше тело должно сидеть прямо и раскрепощённо, подобно снопу сена с перерезанной обвязкой. Ваша речь не должна звучать, также как молчит гитара с порванными струнами. Ваш ум должен быть свободен от мыслей трёх времён. Как водяная мельница, в которую не поступает вода, мысли прекращаются сами собой. Ваше тело должно быть как труп, брошенный на кладбище. Ваш голос должен быть как у немого, который всегда молчит. Ваш ум должен быть как птица, пойманная в силок и неспособная взлететь в небо, — мысли бессильны возникать в таком уме.
\\ \\ До тех пор, пока вы не достигнете уровня, именуемого «определённостью в естественном состоянии», посещайте своего коренного Гуру раз в год, проводя с ним месяц или два. Всё остальное время оставайтесь в ритрите. Проведите в ритрите всю свою жизнь. Когда исчезнут все ваши сомнения в отношении практики, не выходите наружу вообще. Практикуя таким образом, вы не сможете избежать просветления. Вы не сможете не принести пользы существам. Не становитесь знаменитыми. Держитесь скромно. Скрывайте свои качества. Будьте самородком золота, завёрнутым в ветошь. Не будьте куском дерьма, завёрнутым в парчу. Если вы находитесь в затворничестве даже очень долго, но погружены в восемь мирских забот, то вы ничто иное, как дерьмо, завёрнутое в парчу. К сожалению, таково большинство современных практикующих в ритрите.
\subsection{Пожизненный затвор}
\\ \\ В прошлом было много практикующих, которые провели в затворничестве всю жизнь. Они начинали ритрит живыми, а завершали мёртвыми. Они жили в одном месте и никогда не выходили из ритритного дома, несмотря ни на что. Они не встречались ни с кем кроме избранного ритритного помощника. Некоторые находились в затворничестве в горах, а некоторые в лесу. Другие предпочитали жить рядом с монастырём или деревней. Они готовили себе пищу, доставляемую ритритным помощником. Некоторые йогины уходили в затворничество в молодости и проводили в нём оставшиеся сорок-пятьдесят лет. По случаю они приглашали к себе в ритрит коренного Гуру или других великих мастеров, чтобы испросить наставления по природе ума. Некоторые не делали даже и этого. Определившись в осознавании, они практиковали исключительно сами по себе. Множество практикующих в пожизненном ритрите развивали сверхъестественные способности, к примеру оставаться всё время в сидячем положении без сна, практически не питаться и так далее.
\\ \\ Те, кто уходят в пожизненное затворничество, не должны принимать завышенный обет на всю жизнь в ритрите. Они должны увеличивать свои обеты в небольших пропорциях, например по шесть месяцев, по году или трём. Настоящий практикующий не будет тратить ни минуты своей жизни. Он или она будет практиковать до последнего вздоха, приравнивая свою жизнь к практике.
\\ \\ Истинный мастер Ваджраяны вначале освободит свой собственный ум и достигнет просветления, прежде чем начинать учить других. Однако мастера Махаяны учат ещё до достижения собственного просветления. Если практик Ваджраяны стремит\-ся достичь полного просветления в этой жизни, проводя всю жизнь в затворничестве, то он это делает на благо всех существ. Поэтому практикующий в пожизненном ритрите не является эгоистом, стремящимся к личному покою. Наоборот, он альтруист, посвятивший себя благу всех живых существ самым совершенным образом, — посредством достижения полного просветления.
\\ \\ Практик Дзокчена никогда не оставит мотивацию сострадания. Он всегда должен следить, чтобы его мотивация в ритрите и в практике была направлена на благо других, а не наоборот. Каждая сессия должна начинаться с развития намерения практиковать осознавание природы ума на благо живых существ. Эта мотивация должна пронизывать вашу практику осознавания. Помимо этого, каждая сессия должна начинаться с напоминания о чистоте всего видимого, слышимого и осознаваемого в контексте тантрического восприятия. Сочетание практики осознавания с развитием правильной мотивации является очень глубоким методом. Само осознавание свобод\-но от всех эгоцентричных цепляний. Оно врождённо обладает мудростью, состраданием и потенциальной просветлённой активностью. Ригпа наделена собственным диапазоном чисто\-го видения, чистого слышания и чистого осознавания. Поэтому настоящий практик Дзокчена не может не иметь сострадания и чистого восприятия. Так как он не отклоняется от пробуждённости осознавания, он никогда не впадает в эгоистичную или собственническую мотивацию.
\\
\vspace{1cm}
\\
\scriptsize
Это учение было дано выдающимся Мастером Дзокчена 20 века Тулку Ургьен
Ринпоче (1920—1996) весной 1994 года в Наги Гомпа, Непал. Учение не
издавалось, и распростране\-ние манускрипта строго ограничено, поэтому
я опустил все разделы, на которых лежит ограничение. Учение давалось
одновременно с передачей цикла посвящений Лонгчен Ньингтик и посвящено
тому, как делать ритрит по Дзокчену в контексте практики тэрма Джигме Лингпы.
\normalsize
