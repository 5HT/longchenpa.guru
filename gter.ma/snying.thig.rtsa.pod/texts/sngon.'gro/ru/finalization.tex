\section{Заключение}
\\
\subsection{Молитва коренному учителю}
\\
\ti
དཔལ་ལྡན་རྩ་བའི་བླ་མ་རིན་པོ་ཆེ། \\
བདག་གི་སྙིང་གར་པདྨ་གདན་བཞུགས་ལ། \\
བཀའ་དྲིན་ཆེན་པོའི་སྒོ་ནས་རྗེས་བཟུངས་ཏེ། \\
སྐུ་གསུང་ཐུགས་ཀྱི་དངོས་གྲུབ་སྩལ་དུ་གསོལ།\\
\\
\ru
Славный милосердный коренной учитель,\\
Пожалуйста, воссядь на лотосовый трон в моем сердце.\\
Прими меня с безграничной добротой и благослови достижениями\\
Твоего просветленного тела, речи и ума.\\
\\
\ti
པལ་ལྡན་བླ་མའི་རྣམ་པར་ཐར་པ་ལ། \\
སྐད་ཅིག་ཙམ་ཡང་ཡང་ལོག་ལྟ་མི་སྐྱེ་ཞིང། \\
ཅི་མཛད་ལེགས་པར་མཐོང་བའི་མོས་གུས་ཀྱིས། \\
བླ་མའི་བྱིན་རླབས་སེམས་ལ་ལ་འཇུག་འཇུག་པར་ཤོག \\
\\
\ru
Да не возникнет у меня даже на мгновение ложный взгляд\\
На деяния святого Ваджрного Учителя.\\
Благодаря чистой преданности, —\\
Восприятия каждого его действия как учения, —\\
Да войдут высшие благословения Учителя в мой Ум.\\
\\
\ti
སྐྱེ་བ་ཀུན་ཏུ་ཡང་དག་བླ་མ་དང། \\
འབྲལ་མེད་ཆོས་ཀྱི་དཔལ་ལ་ལོངས་སྤྱོད་ནས། \\
ས་དང་ལམ་གྱི་ཡནོ ་ཏན་རབ་རྫོགས་ཏེ། \\
རྡོ་རྗེ་འཆང་གི་གོ་འཕང་མྱུར་ཐོབ་ཤོག \\
\\
\ru
Да не буду я разделен во всех рождениях со святым Учителем.\\
И, насладившись величием Дхармы,\\
Полностью завершу все пути Просветления,\\
Достигну быстро состояния Ваджрахары Великого Пробуждения!

\newpage
\subsection{Посвящение заслуг}
\\
\ti
དགེ་བ་འདི་ཡིས་སྐྱེ་བོ་ཀུན།\\
བསོད་ནམས་ཡེ་ཤེས་ཚོགས་རྫོགས་ཤིང༌།\\
བསོད་ནམས་ཡེ་ཤེས་ལས་བྱུང་བའི།\\
དམ་པ་སྐུ་གཉིས་ཐོབ་པར་ཤོག \\
འགྲོ་ཀུན་དགེ་བ་ཇི་སྙེད་ཡོད་པ་དང༌།\\
བྱས་དང་བྱེད་འགྱུར་དེ་བཞིན་བྱེད་པ་དག \\
བཟང་པོ་ཇི་བཞིན་དེ་འདྲའི་ས་དག་ལ།\\
ཀུན་ཀྱང་ཀུན་ནས་བཟང་པོར་རེག་གྱུར་ཅིག\\
འཇམ་དཔལ་དཔའ་བོས་ཇི་ལྟར་མཁྱེན་པ་དང༌། \\
ཀུན་ཏུ་བཟང་པོ་དེ་ཡང་དེ་བཞིན་ཏེ།\\
དེ་དག་ཀུན་གྱི་རྗེས་སུ་བདག་སློབ་ཅིང༌། \\
དགེ་བ་འདི་དག་ཐམས་ཅད་རབ་ཏུ་བསྔོ།\\
དུས་གསུམ་གཤེགས་པའི་རྒྱལ་བ་ཐམས་ཅད་ཀྱིས།\\
བསྔོ་བ་གང་ལ་མཆོག་ཏུ་བསྔགས་པ་སྟེ།\\
བདག་གི་དགེ་བའི་རྩ་བ་འདི་ཀུན་ཀྱང༌།\\
བཟང་པོ་སྤྱོད་ཕྱིར་རབ་ཏུ་བསྔོ་བར་བགྱི། \\
\\
\ru
Благодаря этой добродетели пусть все существа\\
Завершат накопления заслуг и мудрости,\\
Достижение двух высших тел!\\
Благодаря всей добродетели, какая только есть у существ,\\
И их благим деянием в прошлом, будущем и настоящем,\\
Пусть все они достигнут совершенства —\\
Реализации Самантабхадры!\\
\newpage
\subsection{Особые благопожелания}
\\
\ti
གང་དུ་སྐྱེས་པའི་སྐྱེ་བ་ཐམས་ཅད་དུ།\\
མཐོ་རིས་ཡོན་ཏན་བདུན་ལྡན་ཐོབ་པར་ཤོག\\
སྐྱེ་མ་ཐག་ཏུ་ཆོས་དང་འཕྲད་གྱུར་ཅིང༌།\\
ཚུལ་བཞིན་བསྒྲུབ་པའི་རང་དབང་ཡོད་པར་ཤོག\\
དེར་ཡང་བླ་མ་དམ་པ་མཉེས་བྱེད་ཅིང༌། \\
ཉིན་དང་མཚན་དུ་ཆོས་ལ་སྤྱོད་པར་ཤོག\\
ཆོས་རྟོགས་ནས་ནི་སྙིང་པོའི་དོན་བསྒྲུབ་སྟེ།\\
ཚེ་དེར་སྲིད་པའི་རྒྱ་མཚོ་བརྒལ་བར་ཤོག \\
སྲིད་པར་དམ་པའི་ཆོས་རབ་སྟོན་བྱེད་ཅིང༌།\\
གཞན་ཕན་བསྒྲུབ་ལ་སྐྱོ་ངལ་མེད་པར་ཤོག\\
རླབས་ཆེན་གཞན་དོན་ཕྱོགས་རིས་མེད་པ་ཡིས།\\
ཐམས་ཅད་ཕྱམ་གཅིག་སངས་རྒྱས་ཐོབ་པར་ཤོག\\
\\
\ru Во всех жизнях, где бы я не родился,\\
Пусть обрету семь качеств высшего состояния.\\
Сразу после рождения, повстречав Учение,\\
Пусть получу свободу осуществлять его подлинно.\\
Тогда, радуя святого Гуру,\\
Пусть я стану практиковать Учение день и ночь.\\
Постигнув Учение, а затем исполнив его сущностный смысл,\\
Пусть я пересеку океан существования в этой жизни,\\
Тщательно обясняя святую Дхарму в миру.\\
Пусть я никогда не устану работать на благо других.\\
Благодаря великому потоку безграничного\\
\indent и беспристрастного блага для других,\\
Пусть все, как один, обретут Просветление!\\
\\
\\
\\
\newpage
\ti\scriptsize ཅེས་རྫོགས་པ་ཆེན་པོ་ཀླ
ོང་ཆེན་སྙིང་ཐིག་གི་སྔོན་འགྲོའི་
ངག་འདོན་ཁྲིགས་སུ་སྡེབས་པ་རྣམ་མཁྱེན་ལམ་
བཟང་འདི་ཉིད་རིག་འཛིན་འཇིགས་མེད་གླིང་
པ་སོགས་དམ་པ་དུ་མས་བཀའ་དྲིན་གྱིས་
བསྐྱངས་ཤིང་དམ་ཚིག་ལ་མོས་པ་ཐོབ་པའི་
སྔགས་ཀྱི་རྣལ་འབྱོར་པ་ཆེན་པོ་འཇིགས་
མེད་ཕྲིན་ལས་འོད་ཟེར་གྱིས་བྲིས་པའི་
དགེ་བས་རྗེས་འཇུག་རྣམས་ཀྱིས་བླ་མ་སངས་
རྒྱས་སུ་མཐོང་འབྲས་ཀྱིས་རང་རིག་ཀུན་ཏུ་བཟང་
པོའི་རང་ཞལ་མངོན་དུ་གྱུར་ནས་འགྲོ་ཁམས་རྒྱ་མཚོ་
ལ་ཕན་པ་རྒྱུན་ཆད་མེད་པའི་རྒྱུར་གྱུར་ཅིག\\
\\
\ru\scriptsize\noindent
Это собрание предварительных практик Сердечной Сущности
Об\-ширного Пространства Великого Совершенства было записано
тантри\-ческим йогином Джигме Тринле Озером,
вскормленным добротой святых учителей: Ригдзином Джигме Лингпа
и достигшим твердости в самае. Благодаря накопленным
этим заслугам да будут ученики видеть в учителе самого Будду
и в результате этого да узрят они собственный лик
самоосознавания, Самантабхадру, для того, чтобы нести
нескончаемое благо океану существ.
\normalsize
\newpage
\section*{Комментарии на Нёндро}
\\
\begin{tabular}{ll}
1 & Патрул Ринпоче. \\
  & Слова моего Всеблагого Учителя.\\
  \\
2 & Патрул Ринпоче. \\
  & Краткое руководство по стадиям визуализации \\
  & в практике Лонгчен Ньингтик Нёндро.\\
    \\
3 & Дхонгсар Кхьенце Ринпоче. \\
  & Не счастья ради. Руководство по так называемым \\
  & предварительным практикам тибетского буддизма.\\
  \\
4 & Джигме Лингпа. \\
  & Лестница в Аканишту.\\
  \\
5 & Джамьнг Кьенце Вангпо. \\
  & Освещение Великолепного Пути к Всеведению.\\
\end{tabular}

\section*{Обращение с буддийской литературой}
\\
Дхарма — чудодейственное лекарство, помогающее вам
самим и всем живым существам избавится от страданий.
Ко всякой книге содержащей учение Будды (а даже одно
слово Дхармы или имя Будды делают ее таковой),  следует
относится уважи\-тельно, на каком бы языке она не была написана.
Избегайте класть книгу на пол или стул, не переступайте
через книгу и не ставьте на нее какие либо
предметы — даже изображения Будды или божеств.
Храните книги Дхармы на алтаре или в другом почетном,
чистом и предпочтительно высоком месте.\\
Не выбрасывайте, подобну мусору, старые или неисполь\-зуе\-мые
книги содержашие Учение Будды. Если возникла такая необходимость,
лучше сожгите их — предание огню считается уважительным
способом избавится от пришедшего в негод\-ность 
религиозного текста. Таким же образом рекомендуется
с почтением относится и к писаниям других духовных традиций.
