% HEVEA % Copyright (c) 2015 Synrc Research Center

%\usepackage{afterpage}
\usepackage[english,russian]{babel}
%\usepackage{graphicx}
%\usepackage{tocloft}
\usepackage{fontspec}
\usepackage{polyglossia}
\usepackage{hyphenat}
\usepackage{import}

%\usepackage{caption}
%\usepackage[usenames,dvipsnames]{color}
\usepackage[top=18mm, bottom=22.4mm,
            inner=15mm,outer=18mm,
            paperwidth=142mm, paperheight=200mm]{geometry}

\hyphenation{бес-чувстве-нен cуще-ствования
 буду-щем разно-образии Дхар-му наско-лько счастли-вого ниж-них
 сохрани-лось всеведую-щие свер-нуть неза-висимости}

\fontencoding{T1}

%\setlength{\cftsubsecnumwidth}{2.5em}
%\defaultfontfeatures{Ligatures=TeX}

% include image for HeVeA and LaTeX

%\makeatletter
%\def\@seccntformat#1{\llap{\csname the#1\endcsname\hskip0.7em\relax}}
%\makeatother

\newcommand{\includeimage}[2]
{\begin{figure}[h!]
\centering
\includegraphics[width=\textwidth]{#1}
\caption{#2}
\end{figure}
}

\newcommand*{\titlePRAYER}
{
\newfontfamily{\cyrillicfont}{Geometria}
\setdefaultlanguage{tibetan}
\setmainfont{Geometria}
    \begingroup
        \thispagestyle{empty}
        \hspace*{0.15\textwidth}
        \rule{1pt}{\textheight}
        \hspace*{0.05\textwidth}
        {
        \parbox[c][][s]{0.75\textwidth}
        {
             \noindent
             \vspace{-12cm}
             \textsc{
             \setdefaultlanguage{russian}
             \setmainfont{Geometria}
             \Large
             Сборник Практик\\ [0.3\baselineskip]
             Лонгчен Ньингтик\\ [0.3\baselineskip]
             \\
             \vspace*{3cm}
             \\
             \normalsize
             Устье, 2015}
        }}
    \endgroup
    \setdefaultlanguage{russian}
    \setmainfont{Geometria}
}



% define images store

\graphicspath{{./images/}}

% start each section from new page

%\let\stdsection\section
%\renewcommand\section{\newpage\stdsection}

% define style for code listings

\lefthyphenmin=1
\hyphenpenalty=100
\tolerance=3000

%\newcommand\blankpage{
%    \null
%    \thispagestyle{empty}
%    \newpage}

\newcommand{\Vspace}[1]{\vspace{#1}}
\newcommand{\Section}[1]{\section{#1}}
\newcommand{\SectionNo}[1]{\section*{#1}}
\newcommand{\SubSection}[1]{\subsection{#1}}
\newcommand{\SubSectionNo}[1]{\subsection*{#1}}
\newcommand{\SubSubSection}[1]{\subsubsection{#1}}
\newcommand{\SubSubSectionNo}[1]{\subsubsection*{#1}}

\newcommand{\ru}{
    \setdefaultlanguage{russian}
    \setmainfont{Geometria}
}

\newcommand{\ti}{
    \setdefaultlanguage{tibetan}
    \setmainfont{DDC Uchen}
}

\newtoggle{russian@scriptlangequal}
\newtoggle{tibetan@scriptlangequal}

\newlength\tindent
\setlength{\tindent}{\parindent}
\setlength{\parindent}{0pt}
\renewcommand{\indent}{\hspace*{\tindent}}

% HEVEA \begin{document}
% nyingma_author =Maxim Sokhatsky=
% HEVEA \title{Удивительный Океан Наставлений по Горной Дхарме}
\ru

\newpage
\Section{Ригдзин Джигме Лингпа.\\
Удивительный Океан\\
Наставлений по Горной Дхарме}
\\
\begin{verse}
Воплощение всех великолепных Будд,\\
Сострадательный владыка Падмасамбхава,\\
Воссядь на короне моих тёмно-синих локонов\\
И одели мой ум благословением.
\end{verse}

\\ \\ Слушайте, все верующие, кто соблюдает самаи и стремится к духовной практике из глубины своих сердец. В этой безначальной и бесконечной самсаре зёрна плохой кармы подчинили вас влиянию скверных обстоятельств. Все ваши мысли сводятся к переживаниям страха и страданий. Существа шести миров вынуждены переживать это непрерывно как узники, заточённые в темнице. Если у вас сейчас появляется болезнь, депрессия или нежелательная ситуация, вы начинаете паниковать, впадаете в отчаяние и паранойю, и весь мир уже не мил. А как вам понравится испытывать мучения трёх низших миров? Увы, единственный выход из этих страданий заключается в реализации абсолютной цели высшей Дхармы.
\\ \\ Вы можете заявлять, что «явления иллюзорны», и разъезжать на машинах (в оригинале «на конях»), пить пиво и предаваться развлечениям, а вечером принять религиозное обличив, прочистить дыхание как кузнец, продувающий меха, и позвенеть в свой колокольчик и барабанчик. Таким образом вы точно не достигнете Просветления.
\\ \\ Причина блуждания в самсаре — это цепляние за своё «Я». Как говорится в «Послании Другу»:
\\ \\ «Все пристрастия оборачиваются гибелью, подобно плоду Кимба, — сказал Всемогущий. Их следует оставить, ибо эти цепи сковывают всех существ в темнице самсары».
\newpage
\\ \\ Что касается цепляния за «Я», то вы откладываете практику Дхармы из-за привязанности к своей стране, дому, богатству и владениям. Когда вы находите иголку с ниткой, то славите бога, а если теряете ручку или шнурки, то тут же падаете духом. Таковы внешние проявления эгоизма. Ко внутренним проявлениям можно отнести восприятие мастеров своей традиции как божеств, а всех остальных как демонов, равно как и мысли типа «Чем я хуже Будды Шакьямуни?», являющиеся следствием отсутствия самокритики. К тайным формам эгоизма относятся вещественные цепляния во время стадии развития, концептуальное обрамление стадии завершения, предвзятость в практике сострадания или утверждения, что все вещи пустотны и лишены независимой природы, сопровождающиеся цеплянием за природу самой пустоты, аналогичным тому, как красавица с помутнённым восприятием одержима своим телом, равно как и мысли типа «Вряд ли кто достиг моего уровня медитации, поэтому я не должен ни с кем консультироваться...». Если так, то ваша жизнь пройдёт впустую. Я вам дам один совет, — если вы решительно отбросите привязанности к своей стране, богатству и имуществу, то половина Дхармы будет автоматически реализована.
\\ \\ Когда я вступил во врата абсолютного учения, моя способность отвергнуть эгоизм как плевок в пыли позволила захватить цитадель естественного состояния. Вокруг меня собралось много учеников, и я стал приносить пользу другим посредством моей мотивации и безграничных учений. При себе я держал лишь предметы первой необходимости и не оправдывался, что «мне понадобится это имущество чуть позже» или «мне это будет нужно, если я заболею или умру». Таким образом я не обременял себя заботами о средствах на пропитание в будущем, делая подношения Трём Драгоценностям, выкупая пойманных животных, помогая практикующим и подавая неимущим беднякам. Я не тратил подношения от живых и мёртвых людей на непристойные цели и не скапливал их подобно пчёлам в улье. Поскольку при мне никогда не было большого состояния, я не испытывал смущения перед посетителями.
\\ \\ Помните, мы все умрём. Так как Дхарма свободна от предвзятостей, поддерживайте чистое восприятие по отношению к каждому (практикующему). Если вы исследуете все традиции учения Будды, то все они глубоки по-своему. Меня же вполне устраивает взгляд Великого Завершения (Дзокчен), и все коренные падения исчезли в пространстве.
\\ \\ Создайте прочную основу из предварительных практик и не пренебрегайте ими, говоря, что всё пустотно, теряя поведение в воззрении. Что касается основной части практики, то её следует осуществлять в уединённом незнакомом месте, где вас будет сопровождать лишь осознавание и обет поддерживать непрерывный поток естественного состояния. Если до вас будут доходить хорошие или плохие новости, провоцирующие страхи и надежды, не относитесь к ним всерьёз, не отвергайте их и не принимайте за истину, — будьте как покойник, которому можно сказать что угодно.
\\ \\ Размышляйте о трудности обретения человеческой жизни, о редкости встречи с Дхармой и истинными Учителями. Думайте об уязвимости демонам, о смертности всех живущих, о муках и угнетённости обывателей. У вас должно появиться такое же отвращение к самсаре, какое больной желтухой испытывает к жирной пище. Если вы не будете помнить об этом, то с хорошей пищей, великодушным спонсором, тёплой одеждой, удобным местом и приятными беседами вы лишь подготовите себя к мирской жизни. Таким образом вы создадите себе препятствия ещё до начала истинной практики Дхармы. Как сказано: «Вы можете разглагольствовать на духовные темы высокой реализации с умным лицом, но если вы не покорили демона эгоизма и наслаждения, это всё равно проявится как в вашем поведении, так и во снах». Важно, чтобы вы это поняли.
\newpage
\\ \\ Также говорится, что принимая подношения из зарплаты опричников и чиновников, вы пожнёте пагубные плоды. Если вы поразмыслите, откуда берётся их богатство и состояние, то увидите, что ваша духовная практика вряд ли получит от этого пользу. Помимо этого сказано: «Чёрные подношения отсекают жизненную энергию подобно лезвиям. Одержимость едой перекрывает жизненный канал освобождения». В конечном счёте всё это станет мельничным жёрновом, утягивающим вас в глубины адов. Так что подумайте над этим внимательно, просите милостыню лишь на пропитание и не льстите другим.
\\ \\ Будды прошлого учили:
\begin{verse}«Питайтесь умеренно, \\ \indent
сбалансируйте продолжительность сна \\ \indent и поддерживайте ясность осознавания».
\end{verse}
Если вы будете объедаться, то ваши эмоции увеличатся автоматически. Если вы будете недоедать, то отправитесь побираться по деревням, стуча в свой барабан, бубня ритуалы и оглядываясь на прохожих в ожидании подаяния. Вы будете оправдывать себя словами «если я не сделаю этого, мне не хватит еды...», и в результате станете более алчными, чем беспризорная собака. Поэтому будьте осторожны с количеством потребляемой пищи. Алкоголь — это источник всех бед, так что не пейте больше рюмки. Если не можете быть вегетарианцами, то ешьте немного мяса в соответствии с практикой принятия пищи, описанной в моём тексте «ЧОД ЮЛ ЛАМ КХЬЕР».
\\ \\ Что касается ежедневной практики в ритрите, то трудно установить какой-то единый образец, ибо люди обладают как высшими, так средними и меньшими способностями. Тем не менее, я приведу в пример мой собственный ритрит в Палгьи Риво, длившийся три года и пять месяцев.
\newpage
\\ \\ Просыпался я задолго до рассвета, вставая очень проворно и прочищая своё дыхание девятью выдохами, чтобы разъединить чистые и нечистые части праны. Завершив предварительные практики, я молился так искренно, что слёзы текли из глаз. Затем в течение одной сессии, длившейся до середины утра, я делал практики с праной из особого цикла «Дрол Тик Ньен Гью». Вначале мне пришлось мужественно потерпеть боль, появляющуюся в результате этих упражнений с праной, но вскоре узлы (в каналах) развязались сами собой, и прана вошла в естественное русло. Контролируя 32 левых и 32 правых канала, я мог отслеживать сезонные изменения в продолжительности дня и ночи. Жизненная прана объединилась с нисходящими пранами, и мой живот стал отчётливо напоминать большой круглый сосуд типа дуршлага. Это послужило основой, на которой возникли обычные и особые знаки пути практики. Если же вы удерживаете дыхание лишь на короткий промежуток времени и не имеете ясной визуализации, то важно не хвастаться своей практикой.
\\ \\ После рассвета я выпивал чай или суп и делал огненное подношение. После этого я начинал сессию рецитации (мантр) Приближения и Достижения. Стадия развития подразумевает, что сущность божества — это свобода от цепляний, его проявление — это светоносная форма, а его энергия — ясная концентрация на излучении и возвращении света. Совершенство в стадиях развития и завершения достигается лишь силой такого осознавания. Некоторые нынешние практикующие медитируют чересчур расслабленно, без малейшего усердия, подобно старику, бубнящему «ОМ МАНИ ПЕМЕ ХУНГ». Это неправильно.
\newpage
\\ \\ Практикуя таким образом, я заканчивал эту сессию после полудня. Затем я подносил водяные торма, раскаивался в нарушениях самай, читал «Сампа Лхундрубма», «Цуктор Намгьял», «Йеше Кучок» и другие, равно как и дхарани, мантры и молитвы из сборника ежедневных практик. После завершения сессии я быстро писал около восьми страниц текста, если на то была необходимость. Если мой ум не предрасполагал к этому, я практиковал Тогал.
\\ \\ Во время обеда я обдувал мясо множеством особых мантр и дхарани, развивал сострадание и читал молитвы. Я практиковал йогу принятия пищи, представляя свои скандхи и элементы в виде божеств и читал сутру для очищения принятых подношений. После этого я делал двести-триста простираний и читал молитвы из сутр и тантр.
\\ \\ Потом я сразу же садился и усердно практиковал медитацию и рецитацию мантр моих Йидамов. Благодаря этому я смог реализовать много практик различных божеств. Ближе к вечеру я делал ганапуджу, подносил торма Защитникам и завершал практику этапом растворения (визуализации), относящимся к стадии завершения. Я искренне молился, чтобы осознать ясный свет (во сне), а также читал молитву «Сампа Лхундрубма» («Спонтанное Исполнение Желаний») как для себя, так и для всех существ без предпочтения. Следом за этим я делал сессию практики с праной, после чего начинал йогу сновидений.
\\ \\ В котором часу я бы ни просыпался, я не впадал в дрёму, а однонаправленно сосредотачивал своё внимание, благодаря чему смог добиться прогресса в практике. Вкратце, в течение этих трёх лет я всегда ел одинаковое количество пищи и укрывался лишь одной хлопковой накидкой. Через внутреннюю дверь не просочилось ни единого слова, и даже мои ритритные помощники не переступали через порог внутренней двери. В силу своего отречения, отвращения к самсаре и чёткого понимания непредсказуемости смерти я никогда не позволял себе сплетничать и пустословить.
\\ \\ Вы же, мои ученики, лишь вешаете на дверь затворнические таблички, в то время как ваши мысли бродят где угодно. Когда снаружи раздаются звуки, вы превращаетесь в сторожей и прислушиваетесь к каждому шороху. Если вы встречаете кого-то у внутренней двери, вы обсуждаете новости в Китае, Тибете, Монголии и где угодно. Ваши шесть чувств блуждают снаружи, а вы теряете весь эффект от своего ритрита. Вы следуете за внешними объектами и восприятием, тогда как ваши достижения исчезают снаружи, приглашая препятствия вовнутрь. Если вы будете идти на поводу у таких привычек, то время вашего затворничества истечёт, а ваш ум останется без изменений.
\\ \\ Вы не должны покидать ритрит такими же, какими были до него, будьте решительны в этом. Что бы вы ни делали в затворничестве, — медитацию стадии развития или завершения, читали молитвы, ежедневные практики, писали, болели, хворали или умирали, — вы должны всегда поддерживать невыразимую природу своего ума, лежащую за пределами концептуального рассудка. Не перенапрягаясь и не послабляясь, не медитируя и не отвлекаясь, не исправляя и не утверждая, не улучшая и не ухудшая, вы не должны разлучаться с этим присутствующим осознаванием. Если вы расстанетесь с ним, то у вас начнут появляться разные мысли, увеличивающие ваше тщеславие о «реализации». Вы будете думать: «Я знаю Дхарму и знаком с многими Ламами...». Вы начнёте обсуждать недостатки своих друзей по Дхарме, скапливать богатство, создавать неприятные ситуации, тратя своё время на множество вещей, из которых не будет ни одной верной. Некоторые безмозглые идиоты будут говорить: «Как велика его заслуга и польза живым существам!» Когда же вы начнёте поедать муку, предназначенную для подношения торма, это верный знак того, что вы уже одержимы демонами. Как сказал (Атиша): «Наметь свой ум на Дхарму. Наметь практику Дхармы на нищету. Наметь нищету вплоть до смерти. Наметь смерть в пустой пещере».
\newpage
\\ \\ Все практикующие обязаны почитать эти «четыре наметки» великих мастеров Кадампа как коронную драгоценность. Тогда вы будете неуязвимы для препятствий.
\\ \\ Помимо этого, если вы будете рассказывать о своих переживаниях, реализации, снах, практической информации и трудностях практики в затворничестве, а также о недостатках практикующих своей линии тем людям, кто не имеет тех же самай, то ваши достижения исчезнут, а недостатки наоборот обнаружатся. Поэтому будьте скромнее, поддерживайте со всеми гармоничные отношения, носите оборванную одежду и не озадачивайте себя мирскими амбициями. Внутри у вас не должно быть страха даже перед Владыкой смерти. Внешне вы должны создавать приятное впечатление, будучи более умиротворёнными, чем Царь Лебедей Юл Кхор Сунг.
\\ \\ Итак, практикующие Дхарму должны полагаться только на себя и не слушать никого, кроме истинных Учителей. Как бы искренны не были родительские советы, даже они могут оказаться неверными. Будьте как дикий зверь, вырвавшийся из западни. Когда вы делаете ритрит, никогда не нарушайте своих обетов, будьте как клин, прочно вбитый в землю. Если до вас дойдут плохие новости или произойдёт неприятная ситуация, то не паникуйте, а будьте как безрассудный сумасшедший. Когда находитесь среди множества людей, не теряйте осознанности из-за обычных вещей, вы должны тренироваться в безграничном чистом восприятии всех обусловленных явлений. Когда делаете практики с праной из стадии завершения, вы не должны терять сосредоточение подобно человеку, Вдевающему нитку в иголку. Даже если к вам неожиданно нагрянет смерть, у вас не должно быть тоски и сожалений как у парящего в небе орла, и в вашем уме не должно остаться ничего незавершённого. С этими семью важнейшими принципами вы сможете добиться абсолютного достижения Победоносных прошлого, а также осуществите мои пожелания. Таким образом ваша жизнь обретёт смысл, — войдя во врата высшего учения, вы достигнете конечного плода. А ЛА ЛА ХО!
\\ \\ 
\scriptsize
Я, Дзокченпа Лонгчен Намкхай Налджор, написал этот сердечный наказ на основе своего опыта для усердного практикующего «Чод» Джалу Дордже, чья вера и преданность сделали его превосходным преемником Тайной Мантры. Я прошу вас всех хранить этот текст рядом с местом медитации.
\\ \\ Этот текст из собрания сочинений Джигме Лингпы был любезно предоставлен
ваджрным и ритритным мастером Джигме Ринпоче из Дзокчен монастыря
Святейшего Чатрала Ринпоче.
\normalsize

% HEVEA \end{document}
